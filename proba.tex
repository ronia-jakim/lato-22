\documentclass{article}

\usepackage{notatka}

\usepackage[T1]{fontenc}
\usepackage[polish]{babel}
\usepackage[utf8]{inputenc}

\begin{document}\ttfamily
Przestrzeń liniowa $\R_4[X]$ z bazą $B=(1,\;X,\;X^2,\;X^3,\;X^4)$.\smallskip\\
Rozpiszmy pokolei jak wyglądają pochodne poszczególnych wektorów bazy:
$$\begin{matrix}
    1 && X && X^2 && X^3  && X^4\\
    0 && 1 && 2X  && 3X^2 && 4X^3\\
    0 && 0 && 2   && 6X   && 12X^2\\
    0 && 0 && 0   && 6    && 24X\\
    0 && 0 && 0   && 0    && 24
\end{matrix}$$
Skoro chcemy, żeby wektory bazy dualnej niezerowały się tylko dla jednego wektora z $B$, to zapiszmy je jako:
\begin{align*}
    b*_1&=P(0)\\
    b*_2&=P'(0)\\
    b*_3&=\frac12P''(0)\\
    b*_4&=\frac16P'''(0)\\
    b*_5&=\frac1{24}P^{(4)}(0)
\end{align*}
Wszystko powinno się zgadzać, moje wektory bazy dualnej wydają się być liniowo niezależne, ale nie potrafię za ich pomocą zapisać na przykład funkcji:
$$F=P'(4)+P'''(6)$$

\end{document}