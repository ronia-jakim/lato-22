\section{LICZBY KARDYNALNE}
Mamy kolekcję zbiorów, które wszystkie mają tę samą moc. Ale my byśmy chcieli wie-\\dzieć co to jest ta moc - liczby kardynalne pozwalają nam wybierać zbiory według ich \\mocy.\bigskip
\begin{center}\large
    {\color{def}LICZBA KARDYNALNA} to liczba porządkowa, \\która nie jest równoliczna z żadnym swoim elementem.\smallskip\\
    $Card(\alpha):=On(\alpha)\;\land\;(\forall\;\beta<\alpha)\;|\beta|<|\alpha|$
\end{center}\medskip
Zazwyczaj oznaczamy je $\kappa, \lambda$, chociaż kiedyś używało się gotyku.\bigskip\\
{\large\color{acc}Każda liczba kardynalna jest liczbą porządkową graniczną.}\medskip\\
$Card(0)$\smallskip\\
$Card(\omega)$, ale już $\neg\;Card(\omega+\omega)$, $\neg\;Card(\omega\cdot\omega)$ i $\neg\;Card(\omega^\omega)$.\smallskip\\
$(\forall\;n\in\omega)\;Card(n)$ - dowód później


\subsection{WŁASNOŚĆI}
\begin{center}\large
    Każdy zbiór jest równoliczny z pewną liczbą kardynalną.
\end{center}
\dowod
Ustalmy dowolny zbiór $X$. Wiemy, że $X$ można dobrze uporządkować przez $<$. Wtedy ist-\\nieje liczba porządkowa $\alpha$ z nim izomorficzna:
$$\varphi:X\xrightarrow[1-1]{izo}\alpha$$
W takim razie $\varphi$ jest bijekcją między $X$ a $\alpha$, więc
$$|X|=|\alpha|.$$
Niech
$$\kappa=\min\{\alpha\;:\;|\alpha|\geq|X|\}$$
Wtedy $\kappa\sim X$, a z minimalności $\kappa$ mamy $Card(\kappa)$.\medskip\\
Jeśli $|X|=|\kappa_1|$ i $|X|=|\kappa_2|$, to $|\kappa_1|=|\kappa_2|$.


NOWY WYKŁAD

\subsection{DZIAŁANIA NA LICZBACH KARDYNALNYCH}
\begin{center}\large
    Niech $\kappa$, $\lambda$ będą liczbami kardynalnymi, wtedy:\medskip\\
    $\kappa+\lambda = |(K\times\{0\})\cup(\lambda\times\{1\})|$\medskip\\
    $\kappa\cdot\lambda = |\kappa\times\lambda|$
\end{center}\bigskip
\podz{gr}\bigskip\\
\begin{center}\large
    Jeśli $\kappa\geq\omega$, to $\kappa\cdot\kappa = \kappa$
\end{center}
\dowod
Indukcja po liczbach kardynalnych lub po liczbach porządkowych - obie wersje będą pop-\\rawne.\medskip\\
\indent1. $\kappa=\omega$ $|\omega\times\omega|=|\omega|$\medskip\\
\indent 2. Przypśćmy, że dla nieskończonych liczb kardynalnych $<\kappa$ teza zachodzi.\smallskip\\
Na $\kappa\times\kappa$ definiujemy dobry porządek:
\begin{align*}
    \parl\alpha,\beta\parr\prec\parl\zeta,\xi,\parr\iff&\max\{\alpha,\beta\}<\max\{\zeta,\xi\}\lor\\
    &\lor(\max\{\alpha,\beta\}=\max\{\zeta,\xi\}\land\alpha<\zeta)\lor\\
    &\lor(\max\{\alpha,\beta\}=\max\{\zeta,\xi\}\land\alpha=\zeta\land\beta<\xi)
\end{align*}
Sprawdzanie, że to jest częściowy porządek zostaje na liście\medskip\\
Niech $\gamma=ot(\kappa\times\kappa,\prec)$. Niech $\parl\alpha,\beta\parr\in\kappa\times\kappa$ i niech $\delta = \max\{\alpha,\beta\}$. Wtedy
$$\parl\alpha,\beta\parr\preccurlyeq\parl\delta,\delta\parr$$
i mamy
$$pred(\kappa\times\kappa,\prec,\parl\alpha,\beta\parr)\subseteq pred(\kappa\times\kappa,\prec,\parl\delta,\delta\parr)\subseteq(\delta+1)\times(\delta+1)$$
Ale $\delta<\kappa$, więc
$$|pred(\kappa\times\kappa,\prec\parl\alpha,\beta\parr)|\leq|\delta+1|^2<\kappa$$
Jeśli wezmę dowolne $\eta<\gamma$, to $\eta$ jest odcinkiem początkowym $\gamma$, czyli $\eta<\kappa$. Wobec tego $\gamma\leq\kappa$. Ale $|\gamma|=|\kappa\times\kappa|$, zatem
$$\kappa\cdot\kappa=|\kappa\times\kappa|\leq\kappa.$$
Zdeciniujmy funckję 
$$f:\kappa\to\kappa\times\kappa$$
$$f(\alpha)=\parl\alpha,0\parr$$
któa jest inikcją, więc 
$$|\kappa|\leq|\kappa\times\kappa|$$
Czyli $\kappa\cdot\kappa=|\kappa\times\kappa|=\kappa$.
\kondow
\begin{center}\large
    Wniosek:\smallskip\\
$\color{def}\kappa,\lambda\leq\omega\implies \kappa+\lambda=\kappa\times\lambda=\max\{\kappa,\lambda\}$
\end{center}
\dowod
$$\max\{\kappa,\lambda\}\leq\kappa+\lambda\leq\kappa\cdot\lambda\leq\max\{\kappa,\lambda\}\cdot\max\{\kappa,\lambda\}=\max\{\kappa,\lambda\}$$
Wypadałoby pokazać, że $\kappa+\lambda\leq\kappa\cdot\lambda$, ale to się narysuje\bigskip
\kondow\bigskip
\podz{gr}\bigskip
\begin{center}\large
    Dla każdej liczby kardynalnej istnieje liczba kardynalna od niej większa.
\end{center}
\dowod
Ustalmy dowolne $\kappa$. Wtedy $|\Po{\kappa}|>\kappa$ z twierdzenia Cantora.\medskip\\
\dowod
Wersja bez aksjomatu wyboru:\medskip\\
Z twierdzenia Harcośtam: Dla każdego zbioru $X$ istnieje liczba porządkowa, z którą \\nie istnieje iniekcja z $\alpha$ w $X$.
$$X\mapsto H(X)=\min\{\alpha\;:\;\neg\;(\exists\;\varphi:\alpha\to X)\;\varphi\; to\; iniekcja\}$$
Utalmy $\kappa$. Wtedy $Card(H(\kappa))$ i $H(\kappa)>\kappa$.
\kondow
\newpage
\podz{gr}\bigskip
\begin{center}\large
    {\color{def}NASTĘPNIKIEM} liczby $\kappa$ nazywamy najmniejszą \\liczbę kardynalną od niej większą i oznaczamy ją\smallskip\\
    $\color{acc}\kappa^+$
\end{center}
Czyli $\kappa$ ma dwa następniki: kardynalny i porządkowy.\bigskip
\begin{center}\large
    Liczbę kardynalną $\kappa$ nazywamy {\color{def}NASTĘPNIKIEM}, \\jeśli $\kappa=\lambda^+$ dla pewnego $Card(\lambda)$.\medskip\\
    Liczbę kardynalną nazywamy {\color{def}GRANICZNĄ}, \\jeśli nie jest następnikiem.
\end{center}

\subsection{HIERARCHIA ALEFÓW}
Konstrukcja rekurencyjna:
\begin{align*}
    \aleph_0&=\omega\\
    \aleph_{\alpha+1}&=\aleph_\alpha^+\\
    \aleph_\gamma&=\bigcup\limits_{\xi<\gamma}\aleph_\xi\;Lim(\gamma)
\end{align*}
Alternatywny zapis to $\aleph_\alpha=\omega_\alpha$, ale używamy $\aleph$ żeby podkreślić kardynalny charakter bada-\\nego obiektu.\bigskip\\
\begin{center}\large
    $CARD=\omega\cup\{\aleph_\gamma\;:\;\gamma\in On\}$\smallskip\\
    Każda nieskończoa liczba kardynalna jest jakimś $\aleph$.
\end{center}
\dowod
Przypuśćmy nie wprost, że istnieje $\kappa\geq\omega$
$$(\forall\;\alpha\in ON)\;\kappa\neq\aleph_\alpha$$
Bez zmniejszenia ogólności $\kappa$ jest minimalna.\medskip\\
Rozważmy zbiór 
$$A=\{\xi\;:\;\aleph_\xi<\kappa\}\neq\emptyset.$$
Jest to zbiór niepusty, ponieważ $\kappa\neq\omega$, bo $\omega=\aleph_0$. W takim razie
$$\beta=\bigcup A\;\land\;On(\beta).$$
Są dwie możliwości:\medskip\\
\indent 1. $\beta\in A$, czyli $\beta$ jest największym elementem $A$. Ale wówczas istnieje największa \\liczba kardnalna mniejsza od $\kappa$: $\aleph_\beta$. Ale wtedy $\kappa=\aleph_{\beta+1}$.
\pmazidlo
    \draw[white, ultra thick] (0, 0)--(3, 0);
    \filldraw[def, thick] (1.8, 0) circle (0.1);
    \filldraw[gr, thick] (0.5, 0) circle (0.1);
    \node at (1.8, -0.3) {\color{def}$\kappa$};
    \node at (0.5, -0.45) {\color{gr}$\aleph_\beta$};
    \node at (1, -0.9) {\color{acc}$\aleph_{\beta+1}$};
    \node at (2.5, -0.9) {\color{acc}$\aleph_{\beta+1}$};
    \draw[acc, thick] (1, -0.7)--(1, 0);
    \draw[acc, thick] (2.5, -0.7)--(2.5, 0);
    \node at (1.6, 0.5) {\color{def}$\aleph_\beta+1$};
    \node at (-2, -1.7) {ale wtedy $\beta+1=\min A$};
    \draw[white, thick, <-] (1, -1.2) .. controls (1, -1.5) .. (0.6, -1.7);
    \node[align=left] at (5.8, -1.9) {ale wtedy $\kappa<\aleph_{\beta+1}$, \\czyli $\kappa=\aleph_\alpha$ takie, że $\alpha\in A$};
    \draw[white, thick, <-] (2.5, -1.2) ..controls (2.5, -1.5) .. (2.8, -1.7);
\kmazidlo
\indent 2. $\beta\notin A$, czyli $Lim(\beta)$. Wtedy $\kappa=\aleph_\beta$.
\kondow

\subsection{POTĘGOWANIE}
\begin{center}\large
    Hipoteza continuum
    Czym jest $\cont$?\smallskip\\
    $\cont>\aleph_0\implies\cont\geq\aleph_1$\smallskip\\
    $?\cont=\aleph_1?$
\end{center}

\begin{center}\large
    {\color{def}POTĘGOWANIE} liczb kardynalnych:\medskip\\
    $\kappa^\lambda:=|\kappa^\lambda|$\medskip\\
    Bierzemy zbiór funkcji z $\lambda$ w $\kappa$ i to jest moc tego zbioru.
\end{center}
$$2^\kappa >\kappa$$
$$\kappa\leq\lambda\implies\kappa^\mu \leq\lambda^\mu$$
$$\kappa^{\mu+\lambda}=\kappa^\mu\cdot\kappa^\lambda$$
$$(\kappa\cdot\lambda)^\mu=\kappa^\mu\cdot\lambda^\mu$$
$$(\kappa^\lambda )^\mu =\kappa^{\mu\cdot\lambda}$$

\begin{center}\large
    Niech $2\leq\kappa\leq\lambda$ oraz $\lambda\geq\omega$. Wtedy\smallskip\\
    $\kappa^\lambda=\lambda^\lambda=2^\lambda$.
\end{center}
\dowod
$$2^\lambda\leq\kappa^\lambda \leq\lambda^\lambda \leq |\Po{\lambda\times\lambda}|=|\Po{\lambda}| = 2^\lambda$$
\kondow

\subsection{UOGÓLNIONE OPERACJE NA LICZBACH KARDYNALNYCH}

\begin{center}\large
    Niech $\parl\kappa_i\;:\;i\in I\parr$ będzie indeksowaną rodziną liczb kardynalnych. \smallskip\\
    Wówczas dla tej rodziny definiujemy:\medskip\\
    {\color{def}sumę:} $\sum\limits_{i\in I}\kappa_i = |\bigcup\limits_{i\in I} \kappa_i\times\{i\}|$\medskip\\
    {\color{def}iloczyn}: $\prod\limits_{i\in I} \kappa_i:= |\prod\limits_{i\in I} \kappa_i|$, \\przy czym po prawej mamy uogólniony iloczyn kartezjański zbiorów
\end{center}

Niech $(\forall\;i\in I)\;\kappa_i=\kappa$. Wtedy
$$\sum\limits_{i\in I}\kappa_i=\bigcup\limits_{i\in I}\kappa\times\{i\} = \kappa\times\bigcup\limits_i\{i\}=\kappa\cdot|I|$$
$$\prod\limits_{i\in I}\kappa_i=\kappa^{|I|}$$

Za tydzień:
$$\parl\kappa_i\;:\;i\in I\parr,\parl\lambda_i\;:\;i\in I\parr\quad (\forall\;i\in I)\;\kappa_i<\lambda_i$$
wtedy
$$\sum\limits_{i\in I}\kappa_i<\prod\limits_{i\in I}\lambda_i$$

{\large\color{cyan}UZUPEŁNIĆ}\bigskip\\
{\large\color{acc}PRZYKŁAD}\medskip\\
\indent 1. $\aleph_0$ jest regularna, bo gdyby $cf(\aleph_0)<\aleph_0$, to $cf(\aleph_0)$ byłby skończony i wwtedy $\omega$ byłaby skończoną sumą skończonych zbiorów, a tak nie jest\medskip\\
\indent 2. $\aleph_1$ jest regularne\smallskip\\
Przypuśćmy, że $cf(\aleph_1)<\aleph_1\implies cf(\aleph_1)\leq \aleph_0$. W takim razie istnieje rodzina $\rodz A\subseteq \Po\aleph_1$ taka, że $(\forall\; A\in\rodz A)\;|A|\leq\aleph_0$ i $|\bigcup\rodz|=\aleph_1$, co daje sprzeczność, bo przeliczalna rodzina zbiorów przeliczalnych nie może dać nieprzeliczalnego zbioru.\medskip\\
\indent 3. $\kappa^+\;(\aleph_{\alpha+1})$ jest regularna (dowód na ćwiczeniach)\medskip\\
\indent 4. $\aleph_\omega$ jest singularny\smallskip\\
$$\aleph_\omega=\bigcup\limits{n<\omega}\aleph_n,$$
czyli $\aleph_\omega$ jest sumą przeliczalnie wiely zbiorówo o mocy mniejszej niż $\aleph_\omega$, czyli $cf(\aleph_\omega)\leq\aleph_0<\aleph_\omega$. Łatwo pokazać, że $cf(\aleph_\omega)=\aleph_0$.\medskip\\
\indent 5. $cf(\kappa)$ jest regularna, czyli $cf(cf(\kappa))=cf(\kappa)$, co zostanie pokazane na ćwiczeniach.

\subsection{WNIOSKI Z TWIERDZENIA KONIGA}
Jeśli $\parl\kappa_i\;:\;i\in I\parr$ oraz $\parl \lambda_i\;:\;i\in I\parr$ to są rodziny liczb kardynalnych takie, że
$$(\forall\;i\in I)\kappa_i<\lambda_i,$$
to wówczas
$$\sum\limits_{i\in I}\kappa_i<\prod\limits_{i\in I}\lambda_i$$
\podz{gr}\bigskip\\
1. Niech $\lambda\geq \omega$. Wtedy
$$\lambda^{cf(\lambda)}>\lambda.$$
\dowod
Z definicji kofinalności istnieje taka rodzina $\parl A_\alpha\;:\;\alpha<cf(\lambda)\parr$ podzbiorów $\lambda$ taka, że $|A_\alpha|<\lambda$ dla $\alpha\in cf(\lambda)$ i $|\bigcup A_\alpha|=\lambda.$ Wtedy
$$\lambda=|\bigcup\limits_{\alpha<cf(\lambda)}A_\alpha|\leq \sum\limits_{\alpha\in cf(\lambda)}|A_\alpha|<\prod\limits_{\alpha<cf(\lambda)}\lambda=\lambda^{cf(\lambda)}.$$
\kondow
2. Niech $\kappa\geq\omega$. Wtedy
$$cf(2^\kappa)>\kappa.$$
\dowod
Przypuśćmy, nie wprost, że $cf(2^\kappa)\leq\kappa$. Zastosujmy wniosek pierwszy do $\lambda=2^\kappa.$
$$2^\kappa<(2^\kappa)^{cf(2^\kappa)}\leq(2^\kappa)^\kappa=2^{\kappa\cdot\kappa}=2^\kappa$$
i mamy sprzeczność, bo $2^\kappa<2^\kappa$.
\kondow
Wnioskiem z tego jest
$$\cont=cf(2^{\aleph_0})>\aleph_0,$$
w szczególności
$$\cont=cf(2^{\aleph_0})\neq\aleph_\omega.$$
{\large\color{def}CZYM MOŻE BYĆ CONTINUUM?}\medskip\\
\indent 1. $2^{\aleph_0}=\aleph_1$ - hipoteza continuum\medskip\\
\indent 2. $2^{\aleph_0}=\aleph_2$ - Cohen\medskip\\
\indent 3. $2^{\aleph_0}=\aleph_n\quad 0<n<\omega$\medskip\\
\indent 4. $2^{\aleph_0}\neq\aleph_\omega$.

\subsection{TWIERDZENIE EASTONA (1970)}
\begin{center}\large
    Niech $F: REG\to CARD$, czyli przyporządkowanie funkcyjne z liczb regularnych w kardynalne, takie, że\smallskip\\
    \begin{tabular} { c }
    \makecell[l]{1. $\lambda\leq\lambda\implies F(\kappa)\leq F(\lambda)$  \\
    2. $cf(F(\kappa))>\kappa$.}
    \end{tabular}
    \medskip\\
    Wtedy $CONS(ZFC)\implies CONS(ZFC+(\forall\;\kappa\leq\omega)\;2^\kappa=F(\kappa))$, \\gdzie $CONS$ oznacza brak sprzeczności (czyli istnieje model).\medskip\\
    Przyporządkowanie $\kappa\mapsto 2^\kappa$ nzywamy {\color{def}FUNKCJĄ CONTINUUM}.
\end{center}

\subsection{TWIERDZENIE HAUSDORFFA}
\begin{center}\large
    Jeżeli $\kappa\geq \omega$ i $1\leq \lambda\leq\kappa$, to\smallskip\\
    $(\kappa^+)^\lambda=\kappa^+\cdot\kappa^\lambda$
\end{center}\bigskip
\dowod
$(\leq)$\medskip\\
$$(\kappa^+)^\lambda\leq\kappa^+$$
$$(\kappa^+)^\lambda\leq\kappa^\lambda$$
czyli
$$(\kappa^+)^\lambda\leq max\{\kappa^+, \\kappa^\lambda\}=\kappa^+\cdot\kappa^\lambda$$
($\leq$)\medskip\\
$$(\kappa^+)^\lambda=|(\kappa^+)^\lambda|,$$
ale $(\kappa^+)^\lambda$ to jest zbiórfunkcji z $\lambda$ w $\kappa^+$ i to się równa $(\kappa^+)^\lambda=\bigcup\limits_{\alpha<\kappa^+}\alpha^\lambda$
$(\subseteq)$
$$f\in (\kappa^+)^\lambda,$$
ale
$$\lambda<\kappa^+=cf(\kappa^+),$$
czyli każda fukcja z $\lambda$ w $\kappa^+$ jest ograniczona, czyli $f$ jest ograniczona w $\kappa^+$, a więc \\istnieje $\alpha<\kappa^+$
$$(\forall\;\beta<\lambda)\;f(\beta)<\alpha.$$
W takim razie $f\in \alpha^\kappa$, czyli $f\in\bigcup\limits_{???}\alpha^\lambda$
{\color{cyan}NIE WIEEEEEEEEM CO SIEĘ ZADZIAKOLSFAOHSKDJ FHALKSJD FHALKS}\bigskip\\
{\large\color{acc}PRZYKŁAD}
$$\aleph_3^{\aleph_2}=\aleph_2^{\aleph_2}\cdot\aleph_3=2^{\aleph_2}\cdot\aleph_3=2^{\aleph_2}$$

\subsection{TWIERDZENIE TARSKIEGO}
\begin{center}\large
    $\kappa^\lambda$\\
    $\lambda<cf(\kappa)$, gdzie $Lim(\kappa)$.
\end{center}

\subsection{BIG CARDINALS}
{\large\color{acc}Czy istnieje nieprzeliczalna liczba regularna graniczna?} To jest dobre py-\\tanie. Taka liczba nazywa się liczbą {\large\color{def}SŁABO NIEOSIĄGALNĄ}. Jeśli zmienilibyśmy wa-\\runek na liczbę silnie graniczną, czyli
$$(\forall\;\mu<\kappa)\;2^\mu<\kappa$$
to taka liczba nazywałaby się {\large\color{def}LICZBĄ SILNIE NIEOSIĄGALNĄ}, ale jej istnienie nie \\zostało ani udowodnione ani obalone.\bigskip\\
Problem nie polega na ich istnieniu, tylko na tym, że nie potrafimy udowodnić że one \\istnieją. Istnienie liczby silnie nieosiągalnej $\kappa$ oznacza, że $R_\kappa$ jest modelem teorii \\mnogości. A drugie twierdzenie G{\"o}dela mówi, że rzadna dostatecznie bogata teoria nie \\może dowodzić swojej niesprzeczności. \medskip\\
Matematykom, jak zwykle, to nie przeszkadza i badają sobie konsekwencje istnienia dużych liczb kardynalnych zamiast próbować dowodzić ich istnienia.