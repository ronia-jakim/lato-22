\documentclass{article}

\usepackage{../../notatka}

\begin{document}\ttfamily
surjekcja - funckja na
\section*{DOBRE PORZADKI, LICZBY PORZADKOWE}
    \begin{center}
        Suma przeliczalnie wielu przeliczalnych zbiorow jest przeliczalna:\smallskip\\
        $\aleph_0\geq\bigcup\limits_{n\in\N}A_n,\quad\forall\;n\in\N\quad|A_n|\leq\aleph_0$
    \end{center}
    \dowod
    Poniewaz $|A_n|\leq\aleph_0$ $n\in\N$, istnieje bijekcja
    $$f_n:\N\to A_n.$$
    Chcemy pokazac, ze istnieje rowniez bijekcja:
    $$f:\N\times\N\to\bigcup\limits_{n\in\N}A_n$$
    $$f(n, k)=f_n(k)\quad(\kawa)$$
    Musimy skorzystac z aksjomatu wyboru, poniewaz nie wystarczy nam tylko jeden element z $(f_n)$ - potrzebujemy znac wlasnosci wszystkich elementow $(f_n)$ jednoczesnie. Rozpatrujemy wiec zbior funkcji:
    $$F_n=\{\varphi\in S_n^\N\;:\;\varphi\texttt{ jest bijekcja}\}$$
    dla $n\in\N$, {\color{emp}gdzie $S_n^\N$ to wszystkie funckje $g:\N\to\N$ \emph{lub z $\N$ do podzbioru $A_n$}}. Niech $F$ bedzie funkcja wyboru dla rodziny $\{F_n\;:\;n\in\N\}$, czyli kazdej rodzinie przypisuje element tej rodziny:
    $$F(F_n)\in F_n.$$
    Przepiszmy wiec $(\kawa)$ w sposob bardziej formalny:
    $$f(n,k)=F(F_n)(k).$$
    Poniewaz $F(F_n)$ jest bijekcja, to rowniez $f$ jest bijekcja.
    \kondow
    \begin{center}
        {\large\color{def}LEMAT KURATOWSKIEGO-ZORNA:}\smallskip\\
        Jesli $\langle X, \leq\rangle$ jest zbiorem czesciowo uporzadkowanym, w ktorym \color{acc}kazdy lancuch \\
        jest ograniczony z gory\color{txt}, to w $X$ istnieje \color{emp}element maksymalny\color{txt}.
    \end{center}\bigskip
    \tw {dla dowolnych zbiorow $A, B$ zachodzi $|A|\leq|B|$ lub $|B|\leq|A|$}
    \dowod
    Musimy skonstruowac \emph{zbior czesciowo uporzadkowany $X$, do ktorego bedziemy mogli \color{acc}zastosowac LKZ}. Elementami tego zioru niech beda przyblizenia tego, co chcemy otrzymac:
    $$X=\{f\;:\;\funk{f}\;\land\;\dom{f}\subseteq A\;\land\;\rng{f}\subseteq B\;\land\; \texttt{f jest 1-1}\}.$$
    Bedziemy rozpatrywali $\langle X,\subseteq\rangle$. Chcemy zastosowac do niego LKZ, czyli musimy sprawdzic zalozenia.\smallskip\\
    Niech
    $$\mathcal{L}\subseteq X$$
    bedzie lancuchem w $X$. {\color{acc}Chcemy pokazac, ze ma on ograniczenie gorne}. Niech
    $$L=\bigcup\mathcal{L},$$
    wtedy $L$ jest ograniczeniem gornym $\mathcal{L}$, bo zawiera wszystkie elementy tego lancucha.\medskip\\
    Znalezlismy juz ograniczenie gorne lancucha $\mathcal{L}$, teraz musimy pokazac, ze $L$ jest elementem zbioru $X$ z zalozenia, czyli spelnia nastepujace warunki:\medskip\\

    \indent {\color{emp}1. $L$ jest zbiorem par uporzadkowanych}. Stwierdzenie to wynika bezposrednio z faktu, ze $L$ jest suma lancucha.\medskip\\
    \indent{\color{emp}2. $L$ jest funkcja}, gdyz elementami zbioru $X$ sa funkcje.\smallskip\\
    Chcemy pokazac, ze
        $$\forall\;x,y,z\quad \langle x,y\rangle \in L\;\land\;\langle x,z\rangle\in L\implies y=z,$$
    czyli $L$ jest zbiorem takich par uporzadkowanych, ze jesli dwie pary maja ten sam poprzednik, to maja tez ten sam nastepnik (def. funkcji).\smallskip\\
    Ustalmy dowolne $x,y,z$ takie, ze $\langle x,y\rangle\in L$ i $\langle x,z\rangle\in L$. Zatem istnieja $F,G\in\mathcal{L}$ takie, ze
    $$\langle x,y\rangle\in F\;\land\; \langle x,z\rangle\in G.$$
    Poniewaz $\mathcal{L}$ ma {\color{acc}ograniczenie gorne} (czyli jest zbior do ktorego naleza wszystkie pozostale) i jest {\color{acc}lancuchem}, wszystkie jego elementy mozemy porownac miedzy soba. Czyli, bez straty ogolnosci, mozemy zalozyc, ze $F\subseteq G$ i wowczas
    $$\langle x,y\rangle\in G\texttt{ i }\langle x,z\rangle\in G\implies y=z$$
    gdyz zbior $G$ jest funkcja ($\funk{G}$).\medskip\\
    \indent {\color{emp}3. $\dom{ L}\subseteq A$}\medskip\\
    \indent {\color{emp}4. $\rng{ L}\subseteq B$}\smallskip\\
    \emph{zalozenie 3. i 4. wynikaja bezposrednio z definicji zbioru $X$ oraz $L$}
    $$\dom{\bigcup\mathcal{L}}=\bigcup\limits_{F\in\mathcal{L}}\dom F$$
    $$\rng{\bigcup\mathcal{L}}=\bigcup\limits_{F\in\mathcal{L}}\rng F$$
    \indent{\color{emp}5. $L$ jest funkcja roznowartosciowa (iniekcja)}, czyli jesli $\langle x, y\rangle=\langle z, y\rangle$ to $x=z$.\smallskip\\
    Ustalmy dowolne $x,y,z$ takie, ze $\langle x,y\rangle\in L$ i $\langle z,y\rangle\in L$. Zatem istnieja $F,G\in \mathcal{L}$ takie, ze
    $$\langle x,y\rangle\in F\;\land\;\langle z,y\rangle\in G$$
    Poniewaz $\mathcal{L}$ jest lancuchem, to mozemy zalozyc, ze $F\subseteq G$, a poniewaz $\mathcal{L}\subseteq X$ i $X$ zawiera jedynie iniekcje, to
    $$\langle x,y\rangle\in G\;\land\;\langle z,y\rangle\in G\implies x = z.$$
    Poniewaz pokazalismy, ze dowolny lancuch $X$ jest ograniczony z gory, to na mocy \color{acc}w $X$ istnieje element maksymalny $\varphi\in X$\color{txt}. Rozpatrzmy trzy mozliwosci:\medskip\\
    \indent {\color{emp}1. $\dom \varphi=A$}. Wowczas z definicji zbioru $X$ otrzymujemy $\varphi:A\rarrow{1-1} B$, czyli $|A|\leq|B|$. \medskip\\
    \indent{\color{emp}2. $\rng\varphi=B$}. Wtedy $|B|\leq|A|$, bo
    $$\varphi:\dom\varphi\xrightarrow["na"]{1-1} B$$
    $$\varphi^{-1}:B\xrightarrow["na"]{1-1}\dom\varphi\subseteq A$$
    \indent{\color{emp}3. $\dom\varphi\neq A\;\land\;\rng\varphi\neq B$}. Czyli $\dom\varphi\subsetneq B$ i $\rng\varphi\subsetneq B$, zatem istnieja $s\in A\setminus\dom\varphi$ i $t\in B\setminus\rng\varphi$. W takim razie $\varphi$ moze byc rozszerzona do:
    $$\varphi'=\varphi\cup\{\langle s,t\rangle\}.$$
    $\varphi'\in X$ jest iniekcja, bo $t\notin\rng{\varphi}$. Dodatkowo,
    $$\varphi\subsetneq\varphi',$$
    czyli $\varphi$ nie jest elementem maksymalnym $X$, stad {\color{acc}zachodzi tylko 1 lub 2}.\kondow
\end{document}