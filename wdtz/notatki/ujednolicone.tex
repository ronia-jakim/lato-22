\documentclass{article}

\usepackage{../../notatka}
\usepackage[T1]{fontenc}
\usepackage[polish]{babel}

\title{\ttfamily {\color{tit}Wstęp do Teorii Zbiorów}\medskip\\ \normalsize {\color{dygresyja}notatki na podostawie wykładów J. Kraszewskiego}}
\author{\color{emp}Weronika Jakimowicz}
\date{}

\begin{document}\ttfamily
\maketitle\bigskip
\begin{center}
    {\color{acc}\emph{Ze wstępem do matematyki jest jak z uświadamianiem sekualnym dzieci - mówi \\im się prawdę, ale nie mówi im się wszystkiego.}}
\end{center}\bigskip
\begin{center}
    \tikz\randuck;
\end{center}
\newpage
\tableofcontents
\newpage
\section{JĘZYK LOGIKI}

\subsection{FUNKCJE}
\begin{center}\large
    {\color{def}FUNKCJA} - zbiór par uporządkowanych o właśności jednoznaczości,\\
    czyli nie ma dwóch par o tym samym poprzedniku i dwóch różnych następnikach.
\end{center}\bigskip
Teraz dziedzinę i przeciwdziedzinę określamy poza definicją funkcji - nie są na tym samym poziomie co sama funkcja:
\begin{align*}
    \dom{f}&=\{x\;:\;(\exists\;y)\;\langle x,y\rangle\in f\}\\
    \rng{f}&=\{y\;:\;(\exists\;x)\;\langle x,y\rangle\in f\}.
\end{align*}
Warto pamiętać, że {\color{acc}definicja funkcji} jako \emph{podzbioru $f\in X\times Y$ takiego, że dla każdego $x\in X$ istnieje dokładnie jeden $y\in Y$ takie, że $\langle x,y\rangle \in f$} jest tak samo poprawną definicją, tylko {\color{emp}kładzie nacisk na inny aspekt} funkcji.

\subsection{OPERACJE UOGÓLNIONE}
Dla {\color{def}rodziny indeksowanej} $\{A_i\;:\;i\in I\}$ definiujemy:\smallskip\\
    \indent - jej sumę: $\bigcup\limits_{i\in I}A_i = \{x\;:\;(\exists\;i\in I)\;x\in A_i\}$\smallskip\\
    \indent - jej przekrój: $\bigcap\limits_{i\in I}A_i=\{x\;:\;(\forall\;i\in I)\;x\in A_i\}$


\end{document}