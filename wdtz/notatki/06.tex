\documentclass{article}

\usepackage{../../notatka}

\begin{document}\ttfamily
\section*{ASKJOMAT WYBORU I KOLEDZY}
AC  - aksjomat wyboru (w wersji z selektorem lub z funkcja wyboru)\medskip\\
WO  - zasada dobrego uporzadkowania - kazdy zbior mozna dobrze uporzadkowac\medskip\\
LKZ - Lemat Kuratowskieg-Zorna - jesli $\langle X,\leq\rangle$ jest zbiorem czesciowo uporzadkowanym, w ktoym kazdy lancuch ma ograniczenie gorne, to w $X$ istnieje element maksymalny\bigskip\\
\podz{gr}\bigskip
\begin{center}\large
    Twierdzenie\smallskip\\
    $AC\iff WO\iff LKZ$
\end{center}\bigskip
\dowod
1) $WO\implies AC$\medskip\\
Niech $\mathcal{A}$ bedzie rozlaczna rodzina zbiorow niepustych. Chcemy pokazac, ze ta rodzina ma selektor.\\
Niech $\leq$ bedzie dobrym porzadkiem na $\bigcup \mathcal{A}$.\\
Jesli wszystko jest uporzadkowane, to w kazdym z tych zbiorow mozemy wziac element najmniejszy i to bedzie naszym selektorem\\
Niech $S=\{x\in \bigcup\mathcal{A}\;:\;\exists\;a\in\mathcal{A}\;x\;jest\;elementem\;min\;\mathcal{A}\}$. Wtedy $S$ jest selektorem rodziny $\mathcal{A}$. Trzeba pokazac, ze z kazdym z elementow rodziny $A$ selektor ma jednoelementowy przekroj. Ale jesl iwezme dowolny element rodziny $A$, to to jest niepusty podzbior sumy i i on ma element najmniejszy. W Zwiazku z tym przekros $S$ z $A$ jest niepusty i jednoelementowy, bo w kazdym elemencie z A mamy jeden element najmniejszy.\bigskip\\

2) $LKZ\implies AC$\medskip\\
Niehc $A$ bedzie rozlaczna rodzina zbiorow niepustych.\\
Niech $\mathcal {T}=\{T\subseteq\bigcup\mathcal{A}\;:\;\forall\;A\in\mathcal{A} \quad |T\cap A|\leq 1\}$ - czyli $\mathcal{T}$ jest zbiorem czesciowych sleketorow\\
Rozwazmy zbior up $\langle \mathcal{T},\subseteq\rangle$. Tezn zbior spelnia LKZ\\
Nich $\mathcal{L}\subseteq\mathcal{T}$ nendzie lacuchem. Niech $L=\bigcup\mathcal{L}$. Wtedy $L$ ogranicza od gory $\mathcal{L}$. Trzeba pokazac, ze $L\in\mathcal{T}$.\\
Chcemy pby dla kazdego $A\in\mathcal{A}$ miec $|A\cap L|\leq 1$. Przypuscmy nie wprost, ze istnieje $A\in\mathcal A$ takie, ze $|A\cap L|\geq2$. To znaczy, ze istnieja $x_1, x_2$ takie, ze $x_1\neq x_2$ i $x_1.x_2\in A\cap L$. Wtedy $x_1,x_2\in\bigcup \mathcal L$, czyli istnieja $L_1,L_2$takie, ze $x_1\in L_1$ o $x+2\in L_2$, ale $L$ jest lancuchem , wiec bez straty ogolnosci moge zalozyc, ze $L_1\subseteq L_2$, ale wted y$x_1,x_2\in L_2$, czyli $|A\cap L_2|\geq 2$, ale to jest sprzeczne bo $L_2\in\mathcal{T}$.\bigskip\\
Wobec tego na mocy LKZ w $\mathcal{T}$ istnieje element maksymalny $S$. Pozostaje zauwazyc, ze $S$ jest selektorem rodziny $\mathcal A$.

\end{document}