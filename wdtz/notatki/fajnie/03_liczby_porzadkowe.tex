\section{LICZBY PORZĄDKOWE}
\subsection{LEMAT KURATOWSKIEGO-ZORNA}
\begin{center}\large
    Suma przeliczalnie wielu przeliczalnych zbiorów jest przeliczalna:\smallskip\\
    $\quad(\forall\;n\in\N)\;|A_n|\leq\aleph{0}\implies \aleph_0\geq\bigcup\limits_{n\in\N}A_n$
\end{center}\bigskip

\dowod
Ponieważ $|A_n|\leq \aleph_0$, to istnieje bijekcja
$$f_n:\N\to A_n.$$
Chcemy pokazać, że istnieje też bijekcja:
$$f:\N\times\N\to\bigcup\limits_{n\in\N}A_n$$
$$f(n,k)=f_n(k)\quad(\kawa)$$

Musimy znać wszystkie elementy $(f_n)$ jednocześnie, więc skorzystamy z aksjomatu wyboru. \\Rozpatrzmy zbiór funkcji:
$$F_n=\{\varphi\in S^\N_n\;:\;\varphi\texttt{ jest bijekcją}\}$$
dla $n\in\N$, gdzie $S_n^\N$ oznacza wszstkie funkcje
$$g:\N\to A_n$$
Niech $F$ będzie funkcją wyboru dla rodziny 
$$\{F_n\;:\;n\in\N\},$$ 
czyli każdej rodzinie przypisujemy element tej rodziny:
$$F(F_n)\in F_n.$$
Opiszmy $(\kawa)$ korzystając z funkcji wyboru:
$$f(n, k)=F(F_n)(k).$$
Ponieważ $F(F_n)$ jest bijekcją, to również funkcja $f$ jest bijekcją.
\kondow

\begin{center}\large
    {\color{def}LEMAT KURATOWSKIEGO-ZORNA:}\medskip\\
    Jeśli $\parl X,\leq\parr$ jest zbiorem częściowo uporządkowanym, w którym {\color{acc}każdy łańcuch jest ograniczony z góry}, to w $X$ istnieje {\color{emp}element maksymalny}.
\end{center}\bigskip

\begin{center}\large
    Dla dowolnych zbiorów $A$, $B$ zachodzi\smallskip\\
    $|A|\leq|B|\;\lor\;|B|\leq|A|$
\end{center}
\dowod
Musimy skonstruować zbiór częściowoo uporządkowany $X$, do którego będziemy mogli zasto-\\sować LKZ. Elementami tego zioru niech będą przybliżenia tego, co chcemy otrzymac:
$$X=\{f\;:\;fnc(f)\;\land\;dom(f)\subseteq A\;\land\;rng(f)\subseteq B\;\land\; f\;jest\;1-1\}$$

