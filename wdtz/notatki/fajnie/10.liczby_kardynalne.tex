\section{LICZBY KARDYNALNE}
Mamy kolekcję zbiorów, które wszystkie mają tę samą moc. Ale my byśmy chcieli wie-\\dzieć co to jest ta moc - liczby kardynalne pozwalają nam wybierać zbiory według ich \\mocy.\bigskip
\begin{center}\large
    {\color{def}LICZBA KARDYNALNA} to liczba porządkowa, \\która nie jest równoliczna z żadnym swoim elementem.\smallskip\\
    $Card(\alpha):=On(\alpha)\;\land\;(\forall\;\beta<\alpha)\;|\beta|<|\alpha|$
\end{center}\medskip
Zazwyczaj oznaczamy je $\kappa, \lambda$, chociaż kiedyś używało się gotyku.\bigskip\\
{\large\color{acc}Każda liczba kardynalna jest liczbą porządkową graniczną.}\medskip\\
$Card(0)$\smallskip\\
$Card(\omega)$, ale już $\neg\;Card(\omega+\omega)$, $\neg\;Card(\omega\cdot\omega)$ i $\neg\;Card(\omega^\omega)$.\smallskip\\
$(\forall\;n\in\omega)\;Card(n)$ - dowód później\bigskip\\
\podz{gr}\bigskip\\
\begin{center}\large
    Każdy zbiór jest równoliczny z pewną liczbą kardynalną.
\end{center}
\dowod
Ustalmy dowolny zbiór $X$. Wiemy, że $X$ można dobrze uporządkować przez $<$. Wtedy ist-\\nieje liczba porządkowa $\alpha$ z nim izomorficzna:
$$\varphi:X\xrightarrow[1-1]{izo}\alpha$$
W takim razie $\varphi$ jest bijekcją między $X$ a $\alpha$, więc
$$|X|=|\alpha|.$$
Niech
$$\kappa=\min\{\alpha\;:\;|\alpha|\geq|X|\}$$
Wtedy $\kappa\sim X$, a z minimalności $\kappa$ mamy $Card(\kappa)$.\medskip\\
Jeśli $|X|=|\kappa_1|$ i $|X|=|\kappa_2|$, to $|\kappa_1|=|\kappa_2|$.


NOWY WYKŁAD

\subsection{DZIAŁANIA NA LICZBACH KARDYNALNYCH}
\begin{center}\large
    Niech $\kappa$, $\lambda$ będą liczbami kardynalnymi, wtedy:\medskip\\
    $\kappa+\lambda = |(K\times\{0\})\cup(\lambda\times\{1\})|$\medskip\\
    $\kappa\cdot\lambda = |\kappa\times\lambda|$
\end{center}\bigskip
\podz{gr}\bigskip\\
\begin{center}\large
    Jeśli $\kappa\leq\omega$, to $\kappa\cdot\kappa = \kappa$
\end{center}
\dowod
Indukcja po liczbach kardynalnych lub po liczbach porządkowych - obie wersje będą pop-\\rawne.