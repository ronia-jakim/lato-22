\documentclass{article}

\usepackage{../notatka}
\usepackage[utf8]{inputenc}
\usepackage[T1]{fontenc}
\usepackage[polish]{babel}
\usepackage{array}
\usepackage{microtype}
\usepackage{makecell}
\usepackage{showframe} 
\usetikzlibrary{automata,positioning}
%\usepackage[nomathsymbols, OT4]{polski}
\selectlanguage{polish}

\usepackage{courier}

\renewcommand*\ShowFrameColor{\color{gr}}

\title{\ttfamily {\color{tit}Wstęp do Teorii Zbiorów}\medskip\\ \normalsize {\color{dygresyja}notatki na podostawie wykładów J. Kraszewskiego}}
\author{\color{emp}Weronika Jakimowicz}
\date{}

\begin{document}\ttfamily
\maketitle\bigskip
\begin{center}
    {\color{acc}\emph{Ze wstępem do matematyki jest jak z uświadamianiem sekualnym dzieci - mówi im się prawdę, ale nie mówi im się wszystkiego.}}
\end{center}\bigskip
\begin{center}
    \tikz\randuck;
\end{center}
\newpage
\tableofcontents
\newpage

\section{JĘZYK LOGIKI}

\subsection{FUNKCJE}
\begin{center}\large
    {\color{def}FUNKCJA} - zbiór par uporządkowanych o właśności jednoznaczości,\\
    czyli nie ma dwóch par o tym samym poprzedniku i dwóch różnych następnikach.
\end{center}\bigskip
Teraz dziedzinę i przeciwdziedzinę określamy poza definicją funkcji - nie są na tym \\samym poziomie co sama funkcja:
\begin{align*}
    \dom{f}&=\{x\;:\;(\exists\;y)\;\langle x,y\rangle\in f\}\\
    \rng{f}&=\{y\;:\;(\exists\;x)\;\langle x,y\rangle\in f\}.
\end{align*}
Warto pamiętać, że {\color{acc}definicja funkcji} jako \emph{podzbioru $f\in X\times Y$ takiego, że dla każdego $x\in X$ istnieje dokładnie jeden $y\in Y$ takie, że $\langle x,y\rangle \in f$} jest tak samo poprawną defini-\\cją, tylko {\color{emp}kładzie nacisk na inny aspekt} funkcji.

\subsection{OPERACJE UOGÓLNIONE}
Dla {\color{def}rodziny indeksowanej} $\{A_i\;:\;i\in I\}$ definiujemy:\smallskip\\
    \indent - jej sumę: $\bigcup\limits_{i\in I}A_i = \{x\;:\;(\exists\;i\in I)\;x\in A_i\}$\smallskip\\
    \indent - jej przekrój: $\bigcap\limits_{i\in I}A_i=\{x\;:\;(\forall\;i\in I)\;x\in A_i\}$\medskip\\
Dla {\color{def}nieindeksowanej rodziny zbiorów} $\mathcal{A}$ definiujemy:\smallskip\\
    \indent - suma: $\bigcup\mathcal{A} = \{x\;:\;(\exists\;A\in\mathcal{A})\;x\in A\}$\smallskip\\
    \indent - przekrój: $\bigcap\mathcal{A}=\{x\;:\;(\forall\;A\in\mathcal{A})\;x\in A\}$\medskip\\
Formalnie, indeksowana rdzina zbiorów jest funkcją ze zbioru indeksów w rodzinę zbio-\\rów, więc powinna być zapisywana w nawiasach trójkątnych (para uporządkowana). Sto-\\sowany przez nas zapis w nawiasach klamrowych oznacza zbiór wartości takiej funkcji \\i nie ma znaczenia czy dany podzbiór pojawi się w nim wielokrotnie. Nie przeszkadza \\to więc w definiowaniu sumy czy przekroju.\bigskip\\

\podz{gr}\bigskip\\

{\large{\color{def} UOGÓLNIONY ILOCZYN KARTEZJAŃSKI} (uogólniony produkt) zbiorów:}\medskip\\
Dla dwóch i trzech zbiorów mamy odpowiednio:
$$A_1\times A_2=\{\parl x,y\parr\;:\;x\in A_1\land y\in A_2\}$$
$$A_1\times A_2\times A_3=\{\parl x,y,z\parr\;:\;x\in A_1\land y\in A_2\land z\in A_3\}.$$
Pierwszym pomysłem na definiowanie iloczynu kartezjańskiego trzech i wiecej zbiorów \\będzie definicja rekurencyjna:
$$A_1\times A_2\times A_3:=(A_1\times A_2)\times A_3.$$

Pojawia się problem formalny - {\color{emp}iloczyn kartezjański nie jest łączny}:
$$(A_1\times A_2)\times A_3\neq A_1\times (A_2\times A_3)$$
$$\parl\parl a_1, a_2\parr a_3\parr\neq\parl a_1,\parl a_2,a_3\parr\parr.$$
\emph{Mimo, że iloczyn kartezjański nie jest łączny, matematycy nie mają problemu uznawać, \\że jest łączny, gdyż {\color{acc}istnieje naturalna, kanoniczna bijekcja}, która lewej stronie \\przypisuje prawą stronę.}\medskip\\

Niech $\parl A_i\;:\;i\in I\parr$ będzie indeksowaną rodziną zbiorów, czyli
$$A:I\to\bigcup\limits_{i\in I}A_i$$
$$A(i)=A_i$$
Wyobraźmy sobie iloczyn kartezjański dwóch zbiorów nie jako punkt na płaszczyźnie, \\ale jako dwuelementowy ciąg:
\pmazidlo
\draw[white, thick] (0, 0) -- (0, 2);
\draw[white, thick] (2, 0) -- (2, 2);
\draw[acc, ultra thick] (0, 0.5) -- (2, 1.5);
\filldraw [color=acc, fill=back, thick] (0, 0.5) circle (0.1);
\filldraw [color=acc, fill=back, thick] (2, 1.5) circle (0.1);
\node at (-0.3, 0.5) {$a_1$};
\node at (2.3, 1.5) {$a_2$};
\node at (0, -0.3) {$A_1$};
\node at (2, -0.3) {$A_2$};
\kmazidlo
To przedstawienie łatwo jest przełożyć na nieskończenie długi iloczyn kartezjański, \\wystarczy dorysować kolejne osie z elementami kolejnego podzbioru rodziny:
\pmazidlo
\draw[white, thick] (0, 0) -- (0, 2);
\draw[white, thick] (2, 0) -- (2, 2);
\node at (3, 1) {...};
\draw[acc, ultra thick] (0, 0.5) -- (2, 1.5);
\filldraw [color=acc, fill=back, thick] (0, 0.5) circle (0.1);
\filldraw [color=acc, fill=back, thick] (2, 1.5) circle (0.1);
\node at (-0.3, 0.5) {$a_1$};
\node at (2.3, 1.5) {$a_2$};
\node at (0, -0.3) {$A_1$};
\node at (2, -0.3) {$A_2$};
\draw[acc, ultra thick] (2, 1.5)--(2.4, 1.2);
\draw[acc, ultra thick] (3.6, 1) -- (4, 0.6);
\draw[white, thick] (4, 0)--(4, 2);
\filldraw[color=acc, fill=back, thick] (4, 0.6) circle (0.1);
\node at (4.3, 0.6) {$a_n$};
\node at (4, -0.3) {$A_n$};
\kmazidlo
W ten sposób powstaje funkcja, która kolejnym indeksom przypisuje element z tego inde-\\ksu:
$$f:I\to \bigcup\limits_{i\in I} A_i$$
$$f(i)\in A_i.$$
Według tego, {\color{def}uogólniony iloczyn kartezjański to zbiór funkcji} ze zbioru indeksowego \\w rodzinę indeksowaną:
$$\prod\limits_{i\in I}A_i=\{f\in (\bigcup\limits_{i\in I}A_i)^I\;:\;(\forall\;i\in I)\;f(i)\in A_i\}$$
Jednak dla $I=\{1, 2\}$ nie zachodzi równość:
$$\prod\limits_{i\in I} A_i\neq A_1\times A_2$$
Po lewej mamy zbiór funkcji, a po prawej iloczyn kartezjański. Możemy pokazać natu-\\ralną bijekcję między lewą a prawą stroną, ale byty są róże. Wystarczy pamiętać, że \\mamy co innego i możemy się tym nie przejmować <3

\subsection{JĘZYK PIERWSZEGO RZĘDU}
{\color{emp}JĘZYK RZĘDU ZERO}, czyli rachunek zdań: $p, q, r, ..., \lor, \land, \neg, \implies, \iff$\bigskip\\
{\color{emp}JĘZYK PIERWSZEGO RZĘDU} jest nadzbiorem języka rzędu zero\medskip\\
{\color{acc}część logiczna:}\smallskip\\
    \indent 1. symbole zmiennych: $V=\{x_0, x_1, ...\}$\\
    \indent 2. symbole spójników logicznych: $\{\neg, \lor, \land, \implies, \iff\}$\\
    \indent 3. symbole kwantyfikatorów: $\{\forall, \exists\}$\\
    \indent 4. symbol równości: =\medskip\\
{\color{acc}część pozalogiczna:}\smallskip\\
    \indent 1. symbole funkcyjne: $F=\{f_i\;:\;i\in I\}$\\
    \indent 2. symbole relacyjne (predykaty): $R=\{r_j\;:\;j\in J\}$\\
    \indent 3. symbole stałe: $C=\{c_k\;:\; k\in K\}$\medskip\\
{\color{def}ARNOŚĆ} - odpowiada liczbie argumentów funkcji lub relacji. Każdy symbol ma swoją \\arność.\smallskip\\
{\color{def}SYGNATURA} - zawiera informację o tym, ile jest symboli funkcyjnych, relacyjnych lub \\stałych i jakiej są arności w danym języku. Sygnatura charakteryzuje język.

\subsection{SYNTAKTYKA vs SEMANTYKA}
\emph{Znała suma cała rzeka,\\ 
Więc raz przbył lin z daleka\\
I powiada: "Drogi panie,\\
Ja dla pana mam zadanie,\\
Jeśli pan tak liczyć umie,\\
Niech pan powie, panie sumie,\\
Czy pan zdoła w swym pojęciu,\\
Odjąć zero od dziesięciu?"\\
(...)\\
"To dopiero mam z tym biedę - \\
Może dziesięc? Może jeden?" }\medskip\\
Jak odjąc 0 od 10:\\
    \indent semantycznie: 10 - 0 = 10\\
    \indent syntaktycznie: od ciągu 1 i 0 odjęcie 0 to zostawienie tylko 1\bigskip\\

{\color{def}SEMANTYKA} - patrzy na znaczenie zapisów, nie sam napis.\\
{\color{def}SYNTAKTYKA} - interesuje ją tylko zapis, język, a znaczenia nie ma.

\subsection{KONSTRUOWANIE JĘZYKA}
\begin{center}\large
    {\color{def}TERMY} - bazowy zbiór termów to \\zbiór zmiennych i zbiór stałych:\smallskip\\
    $T_0=V\cup C$\smallskip\\
    {\normalsize Do ich budowy wykorzystujemy symbole funkcyjne ($F$)}
\end{center}
Załóżmy, że mamy skonstruowane termy aż do rzędu $n$ i chcemy skonstruować termy rzędu \\$n+1$. Jeśli mamy symbol funkcyjny arności $k$, to {\color{emp}termem jest zastosowanie tego symbolu do wczesniej skonstruowanych termów}, których mamy $k$:
$$f\in F\quad f\texttt{ -arności k}$$
$$F(t_1, ..., t_k)\quad t_1, ..., t_k\in \bigcup\limits_{i=0}^n T_i$$
Czylil jeśli mamy zbiór termów, to \emph{\color{emp}biorąc wszystkie dostępne symbole funkcyjne i sto-\\sując je na wszystkie możliwe sposoby do dotychczas skonstruowanych termów} tworzone \\są nowe termy.\medskip
\begin{center}Termy to potencjalne wartości funkcji\end{center}\bigskip
\begin{center}\large
    {\color{def}FORMUŁY} - budowane są rekurencyjnie, zaczynając \\od formuł atomowych:\smallskip\\
    $t=s,\quad t,s\in TM$\smallskip\\
    stosując wszystkie relacje równoważności termów\smallskip\\
    $r\in R\quad r(t_1, ..., t_k)$\smallskip\\
    {\normalsize zastosowanie symbolu relacyjnego na odpowiedniej ilości termów tworzy formułę}
\end{center}\medskip
Bazowym poziomem frmuł jest formuła atomowa:
$$F_{m_0}=\{\varphi\;:\;\varphi\texttt{ - formuła atomowa}\}$$
Jeśli mamy $F_{m_k}$ dla pewnego $k<n$, czyli wszystkie formuły poniżej $n$ zostały skon-\\struowane, to
$$F_{m_n}\;:\;\neg\;(\varphi),\;\varphi\lor\phi,\;\varphi\land\phi,...\quad \texttt{dla }\varphi,\phi\in\bigcup\limits_{k<n}F_{m_k},$$
czyli {\color{emp}używamy wszystkich spójników logicznych} dla poprzednich formuł
$$F_{m_n}\;:\;(\forall\;\varphi)\;(\exists\;x_i)\quad \texttt{dla }\varphi\in\bigcup\limits_{k<n}F_{m_k},\;x_i\in V$$
{\color{emp}kwantyfikujemy też po wszystkich możliwych zmiennych wszystkiemożliwe formuły}
$$FM = \bigcup\limits_{n=0}^\infty F_{m_n}$$
\subsection{JĘZYK TEORII MNOGOŚCI}
\begin{center}\large
    {\color{def}$L=\{\in\}$}\smallskip\\
    składa się z jednego binarnego predykatu, \\który nie jest jeszcze należeniem
\end{center}\bigskip
W racuhnku zdań przejście z syntaktyki do semantyki to nadanie symbolom wartości \\prawda lub fałsz.\bigskip
\begin{center}\large
    {\color{def}SYSTEM ALGEBRAICZNY:}\smallskip\\
    {\color{emp}$\rodz{A}=\parl A,\{F_i\;:\;i\in I\},\{R_j\;:\;j\in J\},\{C_k\;:\;k\in K\}\parr$}\smallskip\\
    {\normalsize odpowiednio: zbiór (uniwersum), funkcje na $A$, relacje na $A$, stałe w $A$}
\end{center}\medskip
przykłady: $\parl \Po\N,\subseteq\parr, \;\parl \R, +, \cdot, 0, 1\leq\parr$\bigskip\\
Język $L$ możemy interpretować w systemie $\rodz A$ o ile mają one tę samą sygnaturę.\bigskip
\begin{center}\large
    {\color{def}INTERPRETACJA} to funkcja ze zbioru wartości w uniwersum:\smallskip\\
    $i\;:\;V\to \rodz A,$\smallskip\\
    którą można rozszerzyć do funkcji ze zbioru termów w uniwersum:\smallskip\\
    $\overline i \;:\;TM\to \rodz A$\\
    $i\subseteq \overline i$
\end{center}\bigskip
Ponieważ sygnatury są takie same, to każdemu symbolowi funkcyjnemu możemy przypisać \\funkcję o dokładnie tej samej arności. \emph{Czyli jeśli dany symbol funkcyjny jest nakła-\\dany na termy, to odpowiadająca mu funkcja jest nakładana na wartości tych termów.}\bigskip
\begin{center}\large
    {\color{emp}W systemie $\rodz A$ formuła $\varphi$ jest spełniona przy interpretacji $i$:}\smallskip\\
    $\rodz A \models \varphi[i]$
\end{center}\bigskip
Zaczynamy od formuł atomowych, czyli:\medskip\\
\begin{tabular} { m{3cm} m{15cm} }
    {\color{acc}$\rodz A\models (t=s)[i]$} & wtedy i tylko wtedy, gdy mają tę samą interpretację (czyli $\overline i(t)=\overline i(s)$)\\
    {\color{acc}$\rodz A\models r_j(t_1,...,t_k)[i]$} & wtedy i tylko wtedy, gdy odpowiedająca temu predykatowi relacja zachodzi na wartościach termów (czyli $R_j(\overline i (t_1), ..., \overline i (t_k))$)\\
    {\color{acc}$\rodz A \models (\neg\;\varphi)[i]$} & \makecell[tl]{wtedy i tylko wtedy, gdy nieprawda, że $\rodz A \models \varphi[i]$, i tak ze wszy-\\stkimi spójnikami logicznymi}\\
    {\color{acc}$\rodz A\models (\forall\;x_m)\;\varphi[i]$} & \makecell[tl]{wtedy i tylko wtedy, gdy dla każdego $a\in \rodz A$ mamy $\rodz A\models \varphi[i({x_m\over a})]$ (spraw-\\dzamy dla konkretnego $a$ czy spełnia$\varphi$, a potem dla $x_m$ przypisujemy to \\$a$, natomiast inne wartości dostają podstawienie $({x_m\over a})$?)}
\end{tabular}

\newpage
\section{AKSJOMATY}
{\color{emp}Zbiór oraz należenie} uznajemy za {\color{emp}pojęcia pierwotne}, więc nie definiujemy ich tylko opi-\\sujemy ich własności.

\subsection{AKSJOMAT EKSTENSJONALNOŚĆI}
\begin{center}
    zbiór jest jednoznacznie wyznaczony przez swoje elementy\smallskip\\
    $(\forall\;x)\;(\forall\;y)\;(x=y\iff(\forall\;z)\;(z\in x\iff z\in y))$
\end{center}\medskip
Od tego momentu zakładamy, że \emph{\color{emp}istnieją wyłącznie zbiory}. Nie ma nie-zbiorów. Naszym \\celem jest budowanie uniwersum zbiorów i okazuje się, że w tym świecie można zinter-\\pretować całą matematykę.

\subsection{AKSJOMAT ZBIORU PUSTEGO}
\begin{center}
    istnieje zbiór pusty \O\smallskip\\
    $(\exists\;x)(\forall\;y)\neg\;y\in x$
\end{center}
Na podstawie {\color{def}aksjomatu ekstensjonalności} oraz {\color{def}aksjomaty zbioru pustego} można udowodnić, że istnieje {\color{emp}dokładnie jeden zbiór pusty}.\medskip\\
\begin{tabular} {m{3cm} m{15 cm}}
1. istnienie: & aksjomat zbioru pustego\\
 \\
\makecell[tl]{2. jedyność:} & \makecell[tl]{niech $P_1, P_2$ będą zbiorami pustymi. Wtedy dla dowolnego $z$ zachodzi \\$\neg\;z\in P_1\land \neg\;z\in P_2$, czyli $z\in P_1\iff z\in P_2$. Wobec tego, na mocy aksjomatu \\ekstensjonalności mamy $P_1=P_2$.}
\end{tabular}\bigskip\\
Przyjrzyjmy się następującemy systemowi algebraicznemu:
$$\rodz A_1=\parl\N\cap[10, +\infty),<\parr$$
W systemie spełnione są oba te aksjomaty:
$$\rodz A_1\models A_1+A_2$$
Ponieważ {\color{acc}nie mamy podanej interpretacji}, a nasze aksjomaty są spełnione, to spełnione \\są dla {\color{acc}dowolnej interpretacji}.

\subsection{AKSJOMAT PARY}
\begin{center}\large
    dla dowolnych zbiorów $x, y$ istnieje para $\{x, y\}$\smallskip\\
    $(\forall\;x,y)\;(\exists\;z)\;(\forall\;t)\;(t\in z\iff t=x\lor t=y)$
\end{center}
{\color{acc}Para nieuporządkowana jest jednoznacznie wyznaczona}. Aksjomat mówi tylko o istnieniu \\$z$, a można łatwo udowodnić, korzystając z aksjomatu ekstencjonalności, że takie $z$ is-\\tnieje tylko jedno.\medskip\\
Niech $P_1, P_2$ będa parami nieuporządkowanymi $x, y$. W takim razie jesli $t\in P_1$, to $t=x\lor t=y$. Tak samo $t\in P_2\iff t=x\lor t=y$. Czyli $P_1=P_2$ bo posiadają te same elementy. \bigskip\\
\podz{emp}\bigskip\\
{\color{def}SINGLETONEM} elementu $x$ nazywamy zbiór $\{x\}:=\{x, x\}$\bigskip
\begin{center}\large
    {\color{def}PARĄ UPORZĄDKOWANĄ} (wg. Kuratowskiego) \\elementów $x$ i $y$ nazyway zbiór:\smallskip\\
    $\parl x,y\parr := \{\{x\}, \{x,y\}\}$
\end{center}\medskip
\podz{gr}\medskip\\
Dla dowolnych elementów $a, b, c, d$ zachodzi:
$$\parl a, b\parr = \parl c,d\parr \iff a=c\land b=d$$
\dowod
Rozważmy dwa przypadki:\medskip\\
\indent 1. $a=b$\\
$$\parl a,a\parr = \{\{a\}, \{a, a\}\} = \{\{a\}\}$$
Czyli jeśli $x\in \{\{a\}\}$, to $x=\{a\}$. Z drugiej strony mamy 
$$\parl c, d\parr=\{\{c\}, \{c,d\}\}$$
A więc jeśli $x\in \{\{c\}, \{c,d\}\}$, to $x=\{c\}$ lub $x=\{c, d\}$. W takim razie mamy $\{a\}=\{c\}=\{c, d\}$, a więc z aksjomatu ekstensjonalności, $a=c=d$.\medskip\\
\indent 2. $a\neq b$
$$\parl a, b\parr = \{\{a\}, \{a, b\}\}$$
Jeśli więc $x\in \parl a, b\parr$, to $x=\{a\}$ lub $x=\{a, b\}$. Z drugiej strony mamy
$$\parl c, d\parr=\{\{c\}, \{c, d\}\}$$
Jeśli $x\in \parl c,d\parr$, to $x=\{c\}$ lub $x=\{c, d\}$. W takim razie otrzymujemy $\{c\}=\{a\}$ i $\{c, d\}=\{a, b\}$. Z aksjomatu ekstensjonalności mamy $a=c$ oraz $d=b$.
\kondow
\subsection{AKSJOMAT SUMY}
\begin{center}\large
    Dla dowolnego zbioru istnieje jego suma\smallskip\\
    $(\forall\;x)\;(\exists\;y)\;(\forall\;z)\;(z\in y\iff (\exists\;t)\;(t\in x\land z\in t))$
\end{center}\bigskip
Ponieważ wszystko w naszym świecie jest zbiorem, to \emph{\color{emp}każdy zbiór możemy postrzegać ja-\\ko rodzinę zbiorów} - jego elementy też są zbiorami. W takim razie suma tego zbioru to \\suma rodziny tego zbioru.\medskip\\
{\color{def}Suma jest określona jednoznacznie} i oznaczamy ją $\bigcup x$.\bigskip\\
\dowod
Załóżmy nie wprost, ze istnieją dwie sumy zbioru $x$: $S_1$ i $S_2$. Wtedy
$$(\forall\;z)(z\in S_1\iff (\exists\;t\in x) (z\in t))$$
$$(\forall\;z)(z\in S_2\iff (\exists\;t\in x) (z\in t))$$
Zauważamy, że
$$z\in S_1\iff (\exists\;t\in x)z\in t\iff z\in S_2$$
a więc $S_1$ i $S_2$ mają dokładnie te same elementy, więc z aksjomatu ekstencjonalności są \\tym samym zbiorem.
\kondow
Suma dwóch zbiorów:
$$x\cup y := \bigcup\{x, y\}$$
\dowod
Ustalmy dowolne $z$. Wtedy mamy
\begin{align*}
    z\in \bigcup\{z, y\}&\overset{4}\iff (\exists\;t)\;(t\in \{x, y\}\land z\in t)\overset{3}\iff (\exists\;t)((t=x\lor t=y)\land z\in t)\iff\\
    &\iff (\exists\;t)\;((t=x\land z\in t)\lor (t=y\land z\in t))\iff \\
    &\iff (exists\;t)(t=x\land z\in t)\lor(\exists\;t)(t=y\land z\in t)\implies\\
    &\implies (\exists\;t)(z\in x)\lor (\exists\;t)(z\in y\iff z\in x\lor z\in y)
\end{align*}
\kondow
\subsection{AKSJOMAT ZBIORU PUSTEGO}
\begin{center}\large
    dla każdego zbioru istnieje jego zbiór potęgowy\smallskip\\
    $(\forall\;x)(\exists\;y)(\forall\;z)z\in y\iff (\forall\;t\in z) t\in x$\smallskip\\
    $(\forall\;x)(\exists\;y)(\forall\;z) \;z\in y\iff z\subseteq x$
\end{center}\bigskip
Zbiór potęgowy jest wyznaczony jednoznacznie i oznaczamy go $\Po x$\medskip\\
\dowod
Załóżmy, nie wprost, że istnieją dwa różne zbiory potęgowe $P_1$ i $P_2$ dla pewnego zbioru \\$x$. Wówczas
$$(\forall\;z)\;z\in P_1\iff z\subseteq x$$
$$(\forall\;z)\;z\in P_2\iff z\subseteq x$$
Zauważamy, że
$$z\in P_1\iff z\subseteq x\iff z\in P_2,$$
czyli zbiory $P_1$ i $P_2$ mają dokładnie te same elementy, więc na mocy aksjomatu ekstencjo-\\nalności $P_1=P_2$
\kondow
\subsection{AKSJOMAT WYRÓŻNIANIA}
To tak naprawdę schemat aksjomatu, czyli nieskończona rodzina aksjomatów
\begin{center}\large
    {\color{def}SIMPLIFIED VERSION:} niech $\varphi(t)$ będzie formułą języka teorii mnogości. Wtedy dla tej formuły mamy $\color{tit}A_{6\varphi}$ dla każdego zbioru $x$ istnieje zbiór, którego elementy spełniają własność $\varphi$\smallskip\\
    $(\forall\;x)(\exists\;y)(\forall\;t)(t\in y\iff t\in x\land \varphi(t))$
\end{center}\bigskip
\begin{center}\large
    {\color{def}FULL VERSION:} niech $\varphi(t, z_0, ..., z_n)$ będzie formułą jezyka teorii mnogści. Wtedy pozostałe zmienne wolne będa parametrami (zapis skrócony $z_0, ..., z_n:= \overline z$)\smallskip\\
    Dla każdego układu parametrów i dla każdego $x$ istnieje $y$ taki, że dla każdego $t\in y$ $t$ należy do $x$ i $t$ spełnia formułę $\varphi$\smallskip\\
    $(\forall\;z_0)...(\forall\;z_n)(\forall\;x)(\exists\;y)(\forall\;t)(t\in y\iff t\in x\land \varphi(t, z_0, ..., z_n))$
\end{center}\bigskip
Weźmy półprostą otwartą:
$$(0, +\infty)=\{x\in\R\;:\;x>0\},$$
druga półprosta to
$$(1, +\infty)=\{x\in\R\;:\;x>1\}$$
i tak dalej. Czyli ogólna definicja półprostej to:
$$(a, +\infty)=\{x\in \R\;:\;x>a\}.$$
Dla każdej z tych półprostych trzeba wziąc inną formułę, które wszystkie są zdefinio-\\wane za pomocą formuły
$$\varphi(x, a)=(x>a),$$
gdzie $a$ funkcjonuje jako parametr.
\subsection{AKSJOMAT ZASTĘPOWANIA}
Ostatni aksjomat konstrukcyjny, jest to schemat rodziny aksjomatów\smallskip\\
\begin{center}\large
    {\color{def}SIMPLIFIED VERSION:} niech $\varphi(x, y)$ będzie formułą języka teorii mnogości taką, że:\smallskip\\
    $(\forall\;x)(\exists\;!\;y)\varphi(x, y).$\smallskip\\
    Wówczas dla każdego zbioru $x$ istnieje zbiór $\{z\;:\;(\exists\;t\in x)\;\varphi(t, z)\}$\smallskip\\
    $(\forall\;x)(\exists\;y)(\forall\;z)\;(z\in y\iff (\exists\;t\in x)\;\varphi(t, z))$
\end{center}\medskip
Czyli każdy zbiór można \emph{\color{acc}opisać za pomocą operacji}.\bigskip\\
\begin{center}\large
    {\color{def}FULL VERSION:} niech $\varphi(x, y, p_0, ..., p_n)$ będzie formułą języka teorii mnogości. \smallskip\\
    $(\forall\;p_0), ..., (\forall\;p_n)\;((\forall\;x)\;(\exists\;!y)\;\varphi(x, y, \overline p)\implies (\forall\;x)(\exists\;y)(\forall\;z)\;(z\in y\iff (\exists\;t\in x)\;\varphi(t, z, \overline p)))$
\end{center}

\subsection{KONSTRUKCJE NA ZBIORACH SKOŃCZONYCH}
Niech $x, y$ będą dowolnymi zbiorami. Wtedy definiujemy:\medskip\\
    \indent $x\cap y=\{t\in x\;:\;t\in y\}$\smallskip\\
    \indent $x\setminus y=\{t\in x\;:\; t\notin y\}$\smallskip\\
    \indent $x\times y=\{z\in \Po{\Po{x\cup y}}\;:\;(\exists\; s\in x)(\exists\;t\in y)\;z=\parl s, t\parr\}$\medskip\\
Formalnie stara definicja iloczynu kartezjańskiego nie działa w nowych warunkach, bo \\nie wiemy z czego wyróżnić tę parę uporządkowaną. Ponieważ $s, t\in x\cup y$, mamy
$$\{s\}, \{s, t\}\subseteq x\cup y,$$
a więc 
$$\{\{s\}, \{s, t\}\}\subseteq \Po{x\cup y}.$$
Czyli nasza para uporządkowana jest elementem zbioru potęgowego zbioru potęgowego sumy zbiorów.\medskip\\
    \indent $\bigcap x=\{z\in \bigcup x\;:\;(\forall\;y\in x)\;z\in y\}$ i wówczas $\bigcap\emptyset=\emptyset$\bigskip\\
\podz{gr}\bigskip\\
{\color{def}RELACJA} - definiujemy $\rel r$ jako dowolny zbiór par uporządkowanych:
$$\rel r :=(\exists\;x)(\exists\;y)\;r\subseteq x\times y$$
{\color{def}FUNKCJA} - relcja, która nie ma dwóch par o tym samym poprzedniku i różnych następni-\\kach:
$$\funk f := \rel f \land (\forall\;x)(\forall\;y)(\forall\;z)\;(\parl x,y\parr\in f\land \parl x, z\parr \in f)\implies y = x$$
Dziedzinę i zbiór wartości możemy wówczas zdefiniować jako:
$$\dom f = \{x\in \bigcup \bigcup f\;:\;(\exists\;y)\parl x,y\parr \in f\}$$
$$\rng f = \{y\in \bigcup \bigcup f\;:\;(\exists\;x)\parl x,y\parr \in f\},$$
ponieważ 
$$\{\{x\}, \{x, y\}\}\in f\implies \{x\}, \{x, y\}\in \bigcup f\implies x,y\in\bigcup\bigcup f$$
\emph{Dopóki działamy na zbiorach skończonych, wynikiem operacji zawsze będzie kolejny zbiór skończony - niemożliwe jest otrzymanie zbioru nieskończonego.}

\subsection{AKSJOMAT NIESKOŃCZONOŚCI}
\begin{center}\large
    Istnieje {\color{emp}zbiór induktywny}:\smallskip\\
    $(\exists\;x)\;(\emptyset\in x\land (\forall\;y\in x)\;(y\cup\{y\}\in x))$
\end{center}
Na początku do naszego zbioru $x$ dodajemy $\emptyset$. Potem, skoro $\emptyset$ należy do $x$, to należy też \\$\{\emptyset\}$. Ale skoro do $x$ należy $\emptyset\cup\{\emptyset\}$, to również $\{\emptyset\cup\{\emptyset\}\}$ jest jego elementem i tak dalej.\bigskip\\
\podz{def}\bigskip
\begin{center}\large
    {\color{def}TW.} Istnieje zbiór induktywny najmniejszy względem zawierania, czyli taki, który zawiera się w każdym innym zbiorze induktywnym.
\end{center}\bigskip
\dowod
Niech $x$ będzie zbiorem induktywnym, który istnieje z aksjomatu nieskończoności. Niech
$$\omega=\bigcap\{y\in\Po x\;:\;y \texttt{ jest zbiorem induktywnym}\}$$
Chcę pokazać, że $\omega$ jest zbiorem induktywnym, czyli $\emptyset\in\omega$.
$$\emptyset\in\omega\iff\emptyset\in y \texttt{ dla każdego zbioru induktywnego }y\subseteq x$$
Ponieważ każdy zbiór induktywny zawiera $\emptyset$, także $\omega$ zawiera $\emptyset$.\medskip\\
Pozostaje pokazać, że dla dowolnego $t\in\omega$ mamy
$$t\cup\{t\}\in \omega$$
Dla każdego zbioru induktywnego $y\subseteq x$ mamy $t\in y$. ale ponieważ $y$ jest zbiorem induktyw-\\nym, mamy 
$$t\cup\{t\}\in y.$$
Z definicji przekroju zbioru $x$ mamy
$$t\cup\{t\}\in \bigcap \{y\in \Po x\;:\;\texttt{ y jest zbiorem induktywnym}\}=\omega$$
Czyli istnieje zbiór induktywny $\omega$ będący przekrojem wszystkich innych zbiorów induktyw-\\nych. Pokażemy teraz, że jest to zbiór najmniejszy.\medskip\\
Niech $z$ będzie dowolnym zbiorem induktywnym. Wtedy $z\cap x$ jest zbiorem induktywnym i \\$z\cap x\subseteq x$. Czyli $z$ jest jednym z elementów rodziny, której przekrój daje $\omega$:
$$z\cap x\supseteq \{y\in\Po x\;:\; Y\texttt{ zb. ind.}\}=\omega$$
\kondow
\podz{gr}\bigskip\\
Każdy element $\emptyset,\;\{\emptyset\},\;\{\emptyset,\{\emptyset\}\}...$ możemy utoższamić z {\color{acc}kolejnymi liczbami naturalnymi}. W ta-\\kim razie ten najmniejszy zbiór induktywny będzie utożsamiany ze zbiorem liczb natural-\\nych. Konsekwencją tego jest \emph{\color{emp}zasada indukcji matematycznej}.\smallskip\\
Niech $\varphi(x)$ będzie formułą ozakresiie zmiennej $x\in\N$ takiej, że zachodzi $\varphi(0)$ oraz
$$(\forall\;n\in\N)\;\varphi(n)\implies\varphi(n+1).$$
Wówczas 
$$(\forall\;z\in\N)\;\varphi(n)$$
\dowod
Niech 
$$A=\{n\in\N\;:\;\varphi(n)\}.$$
Wtedy $A\in\N$ oraz $A$ jest induktywny. Kolejne zbiory należące do zbioru induktywnego \\utożsamialiśmy z $n\in\N$, więc skoro $\varphi(n)$ należy do tego zbioru induktywnego, to również \\$\varphi(n+1)$ należy do $A$. Skoro $A$ jest zbiorem induktywnym, to $\N\subseteq A$, więc $A=\N$.
\kondow

\subsection{AKSJOMAT REGULARNOŚCI}
Do tej pory poznaliśmy aksjomaty o instnieniu i serie aksjomatów konstrukcyjnych. Ak-\\sjomat regularności nie jest żadnym z nich.\bigskip
\begin{center}\large
    W każdym niepustym zbiorze istnieje element {\color{emp}$\in-$minimalny:}\smallskip\\
    $\color{acc}(\forall\;x)\;x\neq\emptyset\implies ((\exists\;y\in x)\;(\forall\;z\in x)\;\neg\;z\in y)$,\medskip\\
    a więc eliminowane są patologie jak np: $x\in x$, $y\in y\in x$.
\end{center}\bigskip

Antynomia Russlla,
$$\{x\;:\;x\notin x\},$$
jest eliminowana przez aksjomat regularności.

\subsection{AKSJOMAT WYBORU}
\begin{center}\large
    Dla każdej {\color{emp}rozłącznej rodziny parami rozłącznych \\zbiorów} niepustych {\color{def}istnieje SELEKTOR}\smallskip\\
    $(\forall\;x)\;{\color{acc}(}(\forall\;y,z\in x)\;{\color{tit}(}y\neq\emptyset\;\land\;(y\neq z\implies y\cap z=\emptyset){\color{tit})}\implies(\exists\;s)(\forall\;y\in x)(\exists\;!t)\;t\in s\cap y{\color{acc})}$
\end{center}\bigskip

Problematyczne nie jest znalezienie punktów, które są reprezentantami zbiorów naszej \\rodziny, a wskazanie zbioru, który je wszystkie zawiera. Dlatego w tym może nam pomóc akjomat wyboru. Wystarczy pokazać, że rozważamy rodzinę rozłącznych zbiorów i już z \\tego wiemy, że możemy wybrać selektor. Handy.\bigskip\\
{\large\color{emp}PARADOKS BANACHA-TARSKIEGO:}\medskip\\
Kulę możemy rozłożyć na {\color{acc}5 kawałków} i przesuwać je izometrycznie w taki sposób, żeby \\złożyć z nich {\color{acc}dwie identyczne kule} jak ta, którą mieliśmy na początku. Kawałki na \\które dzielimy są niemieżalne, nie mają objętości, są maksymalnie patologiczne, ale \\nadal możemy powiedzieć że istnieją korzystając z aksjomatu wyboru. Daje on nam tylko informację, że {\color{acc}istnieje selektor, a nie o tym jak on wygląda,} więc może być absurdalny i patologiczny jak tylko ma ochotę.\bigskip\\
\podz{tit}\bigskip\\
\begin{center}\large
    {\color{def}FUNKCJA WYBORU} - niech $\rodz A$ będzie rodziną zbiorów \\niepustych. Funkcją wyboru dla rodziny $\rodz A$ nazywamy \\wtedy {\color{acc}dowolną funkcję $f$:}\smallskip\\
    $f:\rodz A\to\bigcup \rodz A$\smallskip\\
    $(\forall\;A\in\rodz A)\;f(A)\in A$\medskip\\
    \normalsize Aksjomat wyboru jest równoważny temu, że dla każdej rozłącznej rodziny niepustych zbiorów istnieje funkcja wyboru (selektor).
\end{center}

\begin{center}\large
    Dla dowolnych dwóch zbiorów $A$, $B$ zachodzi\smallskip\\
    $|A|\leq|B|\lor|B|\leq|A|$
\end{center}\bigskip
\dowod

Musimy skonstruować zbiór częściowo uporządkowany $X$, do którego będziemy mogli zastosować LKZ. Elementami tego zbioru niech będą przybliżenia tego, co chcemy otrzymać:
$$X=\{f\;:\;fnc(f)\;\land\;dom(f)\subseteq A\;\land\;rng(f)\subseteq B\;\land\;f\;jest\;1-1\}$$
\newpage
\section{LICZBY PORZĄDKOWE}
\subsection{LEMAT KURATOWSKIEGO-ZORNA}
\begin{center}\large
    {\color{def}LEMAT KURATOWSKIEGO-ZORNA:}\medskip\\
    Jeśli $\parl X,\leq\parr$ jest zbiorem częściowo uporządkowanym, w którym {\color{acc}każdy łańcuch jest ograniczony z góry}, to w $X$ istnieje {\color{emp}element maksymalny}.
\end{center}\bigskip
\podz{gr}\bigskip

\begin{center}\large
    Suma przeliczalnie wielu przeliczalnych zbiorów jest przeliczalna:\smallskip\\
    $\quad(\forall\;n\in\N)\;|A_n|\leq\aleph{0}\implies \aleph_0\geq\bigcup\limits_{n\in\N}A_n$
\end{center}\bigskip

\dowod
Ponieważ $|A_n|\leq \aleph_0$, to istnieje bijekcja
$$f_n:\N\to A_n.$$
Chcemy pokazać, że istnieje też bijekcja:
$$f:\N\times\N\to\bigcup\limits_{n\in\N}A_n$$
$$f(n,k)=f_n(k)\quad(\kawa)$$

Musimy znać wszystkie elementy $(f_n)$ jednocześnie, więc skorzystamy z aksjomatu wyboru. \\Rozpatrzmy zbiór funkcji:
$$F_n=\{\varphi\in S^\N_n\;:\;\varphi\texttt{ jest bijekcją}\}$$
dla $n\in\N$, gdzie $S_n^\N$ oznacza wszstkie funkcje
$$g:\N\to A_n$$
Niech $F$ będzie funkcją wyboru dla rodziny 
$$\{F_n\;:\;n\in\N\},$$ 
czyli każdej rodzinie przypisujemy element tej rodziny:
$$F(F_n)\in F_n.$$
Opiszmy $(\kawa)$ korzystając z funkcji wyboru:
$$f(n, k)=F(F_n)(k).$$
Ponieważ $F(F_n)$ jest bijekcją, to również funkcja $f$ jest bijekcją.
\kondow

\podz{gr}\bigskip

\begin{center}\large
    Dla dowolnych zbiorów $A$, $B$ zachodzi\smallskip\\
    $|A|\leq|B|\;\lor\;|B|\leq|A|$
\end{center}
\dowod
Musimy skonstruować zbiór częściowo uporządkowany $X$, do którego będziemy mogli zasto-\\sować LKZ. Chcemy pokazać, że istnieje iniekcja lub suriekcja między tymi dwoma zbiora-\\mi, więc potrzebujemy zbioru zawierającego funkcje z jednego do drugiego:
$$X=\{f\;:\;fnc(f)\;\land\;dom(f)\subseteq A\;\land\;rng(f)\subseteq B\;\land\; f\;jest\;1-1\}.$$

Rozpatrzmy porządek $\parl X, \subseteq$. Aby zastosować do niego LKZ musimy sprawdzić założenia. \\Weźmy łańcuch $X$:
$$\rodz L\subseteq X.$$
Musimy pokazać, że ma on ograniczenie górne. Niech
$$L=\bigcup \rodz L.$$
Ponieważ każdy element $\rodz L\in L$, to $L$ jest ograniczeniem górnym $\rodz L$.\medskip\\
Należy teraz pokazać, że $L$ jest elementem zbioru $X$, czyli spełnia warunki:\smallskip\\
\indent 1. {\color{emp}$L$ jest zbiorem par uporządkowanych} - bezpośrednio z tego, że $L$ jest sumą łań-\\cucha $\rodz L\subseteq X$.\smallskip\\
\indent 2. {\color{emp}$L$ jest funckją,} czyli
$$(\forall\;x,y,z)\;(\parl x, z\parr\in L\;\land\;\parl x, z\parr\in L)\implies y=z.$$
Ustalmy dowolne $x, y, z$ takie, że $\parl x,y\parr\in L$ oraz $\parl x, z\parr\in L$. Zatem istnieją $F, G\in \rodz L$ takie, że
$$\parl x, y\parr\in F\land\parl x, z\parr\in G.$$
Ponieważ $\rodz L$ ma {\color{acc}ograniczenie górne i jest łańcuchem}, to wszystkie jego elementy mogą być między sobą porównywane. Bez straty ogólności możemy więc założyć, że $\color{acc}F\subseteq G$ i z tego wynika, że
$$(\parl x, y\parr\in G\;i\;\parl x, z\parr\in G)\implies y=z,$$
bo $fnc(G)$.\smallskip\\
\indent 3. {\color{emp}$dom(L)\subseteq A$} z tego, że $\rodz L\subseteq X$.\smallskip\\
\indent 4. {\color{emp}$rng(L)\subseteq A$} z tego, że $\rodz L\subseteq X$.\smallskip\\
\indent 5. {\color{emp}$L$ jest funkcją różnowartościową,} czyli $\parl x, y\parr=\parl z, y\parr\implies x=z$.\smallskip\\
Ustalmy dowolne $x, y, z$ takie, że 
$$\parl x,y\parr\in L \; i \; \parl z,y\parr\in L.$$
Zatem istnieją $F, G\in \rodz L$ takie, że
$$\parl x,y\parr\in F\;\land\;\parl z,y\parr\in G.$$
Ponieważ $\rodz L$ jest łańcuchem, to możemy założyć, że $F\subseteq G$, a ponieważ $\rodz L\subseteq X$ i $X$ zawiera jedynie iniekcje, to
$$\parl x,y\parr\in G\;\land\;\parl z,y\parr\in G\implies x=z.$$
Ponieważ pokazaliśmy, że {\color{acc}dowolny łańcuch $X$ jerst ograniczony z góry, to na mocy LKZ \\w $X$ istnieje element maksymalny}
$$\varphi\in X.$$
Rozpatrzmy trzy możliwości:
\indent 1. {\color{emp}$dom(\varphi)=A$:} wówczas z definicji zbioru $X$ otrzymujemy 
$$\varphi : A\xrightarrow[na]{1-1} B$$
a więc $|A|\leq|B|$.\smallskip\\
\indent 2. {\color{emp}$rng(\varphi)=B$:} wtedy $|B|\leq |A|$, bo
$$\varphi\;:\;dom(\varphi)\xrightarrow[na]{1-1} B$$
$$\varphi^{-1}\;:\;B\xrightarrow[na]{1-1} dom(\varphi)\subseteq A.$$
\indent 3. {\color{emp}$dom(\varphi)\neq A\;\land\;rng(\varphi)\neq B$:} czyli $dom(\varphi)\subsetneq A$ i $rng(\varphi)\subsetneq B$, zatem istnieją $s\in A\setminus dom(\varphi)$ oraz $t\in B\setminus rng(\varphi)$. W takim razie $\varphi$ może być rozszerzona do:
$$\varphi'=\varphi\cup\{\parl s,t\parr\}.$$
$$\varphi'\in X$$ nie jest iniekcją, bo $t\notin rng(\varphi)$. Dodatkowo,
$$\varphi\subsetneq \varphi',$$
czyli $\varphi$ nie jest elementem maksymalnym w $X$, stąd {\color{emp}zachodzi tylko 1 lub 2}, czyli $|A|\leq |B|$ lub $|B|\leq |A|$.
\kondow

\subsection{DOBRE PORZĄDKI}
{\color{acc}Dobry porządek }- w każdym niepustym podzbiorze $\parl X, \leq\parr$ istnieje element najmniejszy.
\begin{center}\large
    {\color{def}CZĘŚCIOWY LINIOWY DOBRY PORZĄDEK} $\parl X,\leq\parr,\;Lin(X)$???\smallskip\\
    $(\forall\;A\subseteq X)\;A\neq\emptyset\implies ((\exists\;a\in A)(\forall\;x\in A)\;x\leq A)$\smallskip\\
    $(\forall\;a,b\in A)\;a\leq b\;\lor\;b\leq a$\smallskip\\
    oraz $\leq$ jest zwrotny, przechodni i słabo antysymetryczny.
\end{center}\bigskip
Do tej pory ostry porządek < definiowaliśmy jako skrót
$$x<y\iff x\leq y\land x\neq y.$$
Teraz chcemy, żeby stał się on bytem. Seria twierdzeń z tym związanych:\medskip
\begin{itemize}
    \item relacja < jest przechodznia i silnie antysymetryczne
    \item jeśli < jest relacją przechodnią i silnie antysymetryczną, to relacja zadana warun-\\kiem $x\leq y \iff x< y\lor x=y$ jest częściowym porządkiem
    \item każdemu częściowemu porządkowi odpowiada tylko jeden osry porządek i każdemu os-\\tremu porządkowi odpowiada tylko jeden częściowy porządek.
\end{itemize}
\begin{center}
    {\color{def}SPÓJNOŚĆ} to warunek mówiący, że\smallskip\\
    $(\forall\;x,y)\;x\neq y\implies (xRy\;\lor\; yRx)$
\end{center}\bigskip
\podz{gr}\bigskip\\
{\color{acc}PRZYKŁADY} - dobry porządek\medskip\\
\indent 1. $\parl\N,\leq\parr$ 0 zasada minimum mówi, że w każdym niepustym podzbiorze $\N$ istnieje element najmnijszy, co jest róownoważne zasadzie indukcji matematycznej\smallskip\\
\indent 2. $\parl \{1-\frac1{n+1}\;:\;n\in\N\}, \leq\parr$ - izomorficzne ze zbiorem $\N$\smallskip\\
\indent 3. $\parl \{1-\frac 1{n+1}\}\cup\{1\}, \leq\parr$\smallskip\\
\indent 4. $\parl \{1-\frac 1{n+1}\}\cup\{2-\frac 1{n+1}\}, \leq\parr$\smallskip\\
\indent 5. $\parl n-\frac1m\;:\;n,m\in\N,\leq\parr$\bigskip\\
\podz{gr}\bigskip
\begin{center}\large
    {\color{def}ODCINEK POCZĄTKOWY} - niech $\parl X,\leq\parr$ będzie zbiorem \\z dobrym porządkiem $\leq$ i $a\in X$. Wówczas \\odcinkiem początkowym tego zbioru wyznaczonym \\przez $a$ jest zbiór\smallskip\\
    $pred(X, a,\leq)=\{x\in X\;:\;x<a\}$
\end{center}
W przykładach wyżej każdy zbiór jest odcinkiem początkowym dla zbioru następnego. \\'Krótsze porządki' są odcinkami początkowymi dla dłuższych porządków.\bigskip\\
{\large\color{acc}TWIERDZENIE:} dla dowolnego $a\in X$
$$pred(X, a, \leq)\not\simeq X$$
\dowod
Przypuśćmy, nie wprost, że dla pewnego $a\in X$ mamy
$$pred(X, a,\leq)\simeq X,$$
czyli isitnieje izomorfizm
$$f:X\to pred(X, a,\leq).$$
Wtedy $f(a)<a$, bo izomorfizm zachowuje porządek, i zbiór
$$A=\{x\in X\;:\;f(x)<x\}$$
jest niepusty. Niech $b=\min A$, ale wtedy
$$f(b)<b\implies f(f(b))<f(b),$$
czyli $b>f(b)\in A$, co jest sprzeczne z $b=\min A$.
\kondow
\podz{gr}\bigskip\\
Niech $\parl X, \leq_X\parr,\;\parl Y,\leq_Y\parr$ będą zbiorami dobrze uporządkowanymi. Wtedy zachodzi jedna z trzech możliwości:\smallskip\\
\indent 1. te dwa zbiory są {\color{acc}izomorficzne} $(X\simeq Y)$, czyli są tej samej długości\smallskip\\
\indent pierwszy jest dłuższy od drugiego:
$$(\exists\;a\in X)\;\parl pred(X, a, \leq_X), \leq\parr\simeq \parl Y,\leq_Y\parr$$
\indent 3. drugi jest dłuższy od pierwszego:
$$(\exists\;a\in Y)\;\parl pred (Y, a, \leq_Y), \leq\parr\simeq \parl X, \leq_X\parr$$
\emph{Wypadałoby to wszystko udowodnić, ale to jest przyjemny wykład i uznamy, że wszystko \\śmiga, żeby przejść do bardziej podniecających rzeczy, gdzie będziemy korzystać z po-\\prawności tego nieistniejącego dowodu :3}
\subsection{ZBIÓR TRANZYTYWNY}
\begin{center}
    \emph{Elementy moich elemntów są moimi elementami!}\medskip\\
    Zbiór $A$ nazywamy zbiorem {\color{acc}TRANZYTYWNYM}, gdy każdy jego element jest zarazem jego podzbiorem:\smallskip\\
    $(\forall\;x\in A)\;x\subseteq A$
\end{center}
$\emptyset$ jest zbiorem tranzytywym, bo nie ma elementów - ponieważ nie istnieją, to mogą mieć \\dowolne własności, w szczególnośći mogą być podzbiorami $\emptyset$. Tak jak wwierszy \emph{Na wyspach Bergamota}.\medskip\\
$\{\emptyset\}$ - jego jedyny element to zbiór pusty, który jest jednocześnie jego podzbiorem.\medskip\\
$Tran(\omega)$ - każda liczba naturalna jest zbiorem liczb od siebie mniejszych - dowód na \\liście zadanek :v\bigskip\\
\podz{gr}\bigskip\\
Jeżeli zbiór jest tranzytywny, to tranzytywna jest też jego {\color{def}suma, zbiór potęgowy i jego następnik:}
$$Tran(A)\implies Tran(\bigcup A)\implies Tran(\Po A)\implies Tran(A\cup \{A\})$$
\dowod
Udowodnimy, że $Tran(A)\implies Tran(A\cup \{A\})$\medskip\\
Ustalmy dowolne $x\in A\cup\{A\}$. Wtedy zachodzi jeden z dwóch przypadków:\medskip\\
\indent 1. $x\in A$, a ponieważ $Tran(A)$, to
$$(\forall\;y\in x)\;y\in A$$
\indent 2. $x\in\{A\}$, czyli $x=A$, a więc z $Tran(A)$ otrzymujemy, że $y\in x\implies y\in A\implies y\in \{A\}$.
\kondow
\subsection{LICZBY PORZĄDKOWE}
\begin{center}\large
    Zbiór tranzytywny $A$ nazywamy {\color{def}LICZBĄ PORZĄDKOWĄ}, \\jeśli spełnia warunek\smallskip\\
    $(\forall\;x,y\in A)\;x\in y\;\lor\;x=y\;\lor\; y\in x$\smallskip\\
    i używamy oznaczenia $On(A)$.
\end{center}
Jeśli $On(\alpha)$, to $\alpha$ jest dobrze uporządkowane przez $\in$, czyli każdy niepusty zbiór $A\subseteq \alpha$ ma element $\in$-minimalny:
$$(\forall\;A\subseteq \alpha)\;A\neq\emptyset\implies(\exists\;x\in A)(\forall\;y\in A)\;x=y\;\lor\;x\in y,$$
co wynika z aksjomatu regularności.\bigskip\\
\podz{gr}\bigskip\\

{\large\color{def}PODSTAWOWE WŁASNOŚCI LICZB PORZĄDKOWYCH:}\medskip\\
$\alpha, \beta$ - liczby porządkowe, $C$- zbiór liczb porządkowych\medskip\\
\indent {\color{acc}1. $(\forall\;x\in\alpha)\;On(\alpha)$} - elementy liczby porządkowej są liczbami porzadkowymi.\smallskip\\
Ustalmy dowolne $x\in\alpha.$ Ponieważ $Tran(\alpha)$, to
$$x\in\alpha.$$
Zatem $Lin(x)$, bo $Lin(\alpha)$. Ustalmy dowolne $y\in x$ i $x\in y$. Skoro $x\subseteq \alpha$, to $y\in\alpha$, czyli $y\subseteq \alpha$, zatem $z\in \alpha$. W takim razie $x, z$ są porównywalne jako elementy $\alpha$. Mamy trzy możliwości: $z\in x$, $x\in z$ (sprzeczne z aksj. regularności), $z=x$ (sprzeczne z aksj. regularności).\medskip\\
\indent {\color{acc}2. $\alpha\in \beta\iff\alpha\subset \beta$}\medskip\\
\indent {\color{acc}3. $\alpha\in \beta\;\lor\;\alpha=\beta\;\lor\;\beta\in\alpha$} - dowolne dwie liczby porządkowe są porównywalne.\smallskip\\
Niech $A=\alpha\cap\beta$. Wtedy $On(A)$. Przypuśćmy, że
$$A\neq \alpha\;\land\;A\neq\beta.$$
Wówczas $A$ jest prawdziwym podzbiorem zarówno $\alpha$ jak i $\beta$. Ale z 2 mamy
$$A\in \alpha\;\land\;A\in\beta,$$
czyli
$$A\in \alpha\cap\beta=A.$$
Jest to sprzeczne z aksjomatem regularności, więc $A=\alpha$ lub $A=\beta$, czyli $\alpha\subseteq \beta$ lub \\$\beta\subseteq\alpha$, co z 2 daje nam $\alpha\in \beta$ lub $\beta\in\alpha$.\medskip\\
\indent {\color{acc}4. $Tran(C)\implies On(C)$}\medskip\\
\indent {\color{acc}5. $C\neq\emptyset\implies(\exists\;\alpha\in C)(\forall\;\beta\in C)\;\alpha=\beta\;\lor\;\alpha\in\beta$.}\bigskip\\

Liczbę porządkową $\alpha$ utożsamiamy ze zbiorem dobrze uporządkowanym $\color{def}\parl \alpha, \in\parr$. Możemy w takim razie mówić o $\color{def}pred(\alpha,\in,\beta)$, ale skrócimy to do zapisu:
$$\color{acc}pred(\alpha,\in,\beta)=pred(\alpha,\beta)=\{x\in\alpha\;:\;x\in\beta\}=\beta,$$
czyli każda liczba porządkowa jest zbiorem liczb porządkowych od niej mniejszych.\bigskip\\
Jeśli $On(\alpha)$, to wtedy $\alpha\cup\{\alpha\}$ jest najmniejszą liczbą porządkową większą od $\alpha$ i nazywamy ją {\color{def}NASTĘPNIKIEM} porządkowym liczby $\alpha$
$$\alpha\cup\{\alpha\}:=\alpha+1$$
\begin{center}\large
    Nie istnieje zbiór wszystkich liczb porządkowych\medskip\\
    \normalsize\emph{paradoks Burali-Forti}
\end{center}
\dowod
Przypuścmy nie wprost, że $\color{acc}ON$ jest zbiorem wszystkich liczb porządkowych. Wtedy 
$$Tran(ON),$$ 
bo jeśli $\alpha\in ON$ i $\beta\in \alpha$, to $\beta\in ON$. Ponadto, $Lin(ON)$ z własności 3. Zatem 
$$On(ON),$$ 
czyli $ON\in ON$, co jest sprzeczne z aksjomatem regularności.
\kondow

\podz{def}\bigskip
\begin{center}\large
    Nich $\parl X,<\parr$ będzie zbiorem dobrze \\uporządkowanym. Wtedy istnieje dokładnie \\{\color{def}jedna liczba porządkowa $\alpha$ taka, że\smallskip\\
    $\parl X,<\parr\simeq \parl\alpha,\in\parr$}
\end{center}
Czyli każdy zbiór dobrze uporządkowany jest izomorficzny z jakąś liczbą porządkową.\bigskip\\
\dowod
{\large\color{def}1. JEDYNOŚĆ}\medskip\\
Przypuśćmy, nie wprost, że istnieją dwie różne liczby porządkowe $\alpha,\;\beta$ spełniające zależ-\\ność z twierdzenia. Wtedy
$$\alpha\simeq\beta,$$
co jest sprzeczne z ich różnością - któraś musi być mniejsza i wyznaczać odcinek począt-\\kowy w drugiej. Zbiór nie może być izomorficzny ze swoim odcinkiem poczatkowym.\bigskip\\
{\large\color{def}2. ISTNIENIE}\medskip\\
Zdefiniujmy zbiór
$$Y=\{a\in X\;:\;(\exists\;!\gamma_a)\;On(\gamma_a)\;\land\;\parl pred(X,a,<), <\parr\simeq\gamma_a\},$$
czyli wybieram podzbiory $X$, dla których twierdzenie zachodzi. Zauważmy, że $Y\neq \emptyset$, bo w $X$ istnieje element minimalny (z dobrego porządku). \medskip\\
Dla $a\in Y$ rozważmy izomorfizm
$$\varphi_a:pred(X,a,<)\to \gamma_a.$$
Niech $b\in Y$ o $b<a$. Wtedy
$$\varphi_a(b)\in\gamma_a$$
\pmazidlo
\draw[gr, very thick] (0, 0)--(5, 0);
\draw[gr, very thick] (0,2)--(5,2);
\draw[emp, ultra thick, ->] (2.5, 2)--(2.5, 0);
\node at (5.3, 2) {a};
\node at (5.3, 0) {$\gamma_a$};
\node at (2.5, 2.3) {b};
\node at (2.5, -0.4) {$\varphi_a(b)$};
\kmazidlo

W takim razie, $\varphi_{a \obet pred(X, b, <)}$ jest izomorfizmem pomiędzy $pred(X, b, <)$ i $\varphi_a(b)$. W takim razie $b\in Y,$ czyli $Y$ jest zamknięty w dół.\medskip\\
Stąd możemy wnioskować, że $X=Y$ lub $Y=pred(X, c, <)$. Załóżmy, że $Y=pred(X, c, <):$
$$X\neq Y\implies X\setminus Y\neq\emptyset.$$
Niech $c=\min(X\setminus Y)$, wówczas 
$$Y=pred(X, c, <).$$
Mam węc zbiór $Y$, z którego każdym elementem jest związana jakaś liczba porządkowa. \\Z aksjomatu zastępowania mogę stworzyć zbiór wszystkich tych liczb porządkowych.
$$f:Y\to ON$$
$$f(a)=\gamma_a$$
$$A=rng(f)=\{\gamma_a\;:\;a\in Y\}.$$
Wystarczy pokazać:\medskip\\
\indent 1. $Tran(A)\implies On(A)$ (z 4.):\smallskip\\
Ustalmy $\xi\in A$ oraz $\zeta\in \xi$. Skoro $\xi\in A$, to $\xi=\gamma_a$ dla pewnego $a\in Y$. Wtedy istnieje $b<a$ takie, że $\varphi_a(b)=\zeta$. Stąd wynika, że $\zeta=\gamma_b$, czyli $\zeta\in A$.\medskip\\
\indent 2. $f$ jest izomorfizmem porządkowym.\smallskip\\
Jest funkcją 1-11 z definicji zbioru $Y$, a funkcją "na" z definicji zbioru $A$. Zachowuje \\porządek, bo rozważamy odcinki początkowe.\medskip\\
\indent 3. $X=Y$ \smallskip\\
$Y=pred(X, c, <)$, a {\color{emp}pokazaliśmy, że $c\in Y$, bo $Y\simeq On(\alpha)$, więc jest dobrym porządkiem (ma elememnt najmniejszy)}. W takim razie tu byłaby sprzeczność.\\
Wyżej zakładaliśmy, że $X\neq Y\implies Y=pred(X, c, <)$. Ponieważ !?!?!?!?!?
\kondow
{\color{dygresyja}TWIERDZENIE NA BOCZKU}\smallskip\\
{\color{acc}TWIERDZENIE HARTOGSA} - Dla każdego zioru $X$ istnieje liczba porządkowa $\alpha$, dla której \\nie istnieje funkcja różnowartościowa w zbiór $X$\bigskip\\
\podz{def}\bigskip
\begin{center}\large
    {\color{def}TYPEM PORZDKOWYM} zbioru dobrze uporządkowanego \\nazywamy liczbę porządkową, z którą jest \\on homeomorficzny.\smallskip\\
    $ot(\N,\leq)=ot(\parl\{1-\frac1{n+1}\;:\;n\in\N\}\parr,\leq)=\omega$\smallskip\\
    $ot (\parl\{1-\frac1{n+1}\;:\;n\in\N\}\cup\{1\},\leq\parr)=\omega+1$
\end{center}

\subsection{DZIAŁANIA NA LICZBACH PORZĄDKOWYCH}
Niech $\alpha,\;\beta$ będą liczbami porządkowymi. Wówczas {\color{def}dodawanie definiujemy}:
{\large$$\alpha+\beta=ot(\alpha\times\{0\}\cup\beta\times\{1\}, \leq)$$}
\pmazidlo
\draw[acc, ultra thick] (0, 0) -- (2, 0);
\node at (1, -0.3) {$\color{acc}\alpha$};
\draw[def, ultra thick] (1.8, 1.3) -- (3.8, 1.3);
\node at (2.8, 1) {$\color{def}\beta$};
\node at (-0.3, 0) {0};
\node at (1.5, 1.3) {1};
\draw[gr, very thick] (-0.5, 1.8) rectangle (4.1, -0.5);
\node at (3.8, -0.8) {$\color{gr}\alpha+\beta$};
\kmazidlo
czyli najpierw rozdzielamy je, a potem sumujemy. Relację porządku na sumie liczb po-\\rządkowych definiujemy (porządek leksykograficzny):
{\large$$\parl\gamma, i\parr\leq_{lex}\parl\xi, j\parr\iff i<j\;\lor\; (i=j\;\land\;\gamma<\xi).$$}
{\color{def}Mnożenie liczb porządkowych} to z kolei typ porządkowy ich iloczynu z porządkiem leksy-\\kograficznym:
{\large$$\alpha\cdot\beta=ot(\beta\times\alpha,\leq_{lex})$$}
\pmazidlo
\draw[acc, ultra thick] (0, 0)--(0, 1.5);
\draw[acc, ultra thick] (0.4, 0)--(0.4, 1.5);
\draw[acc, ultra thick] (0.8, 0)--(0.8, 1.5);
\draw[acc, ultra thick] (1.2, 0)--(1.2, 1.5);
\draw[acc, ultra thick] (1.6, 0)--(1.6, 1.5);
\draw[acc, ultra thick] (2, 0)--(2, 1.5);
\draw[def, ultra thick] (0, 0)--(2, 0);
\node at (1, -0.4) {$\color{def}\beta$};
\node at (-0.3, 0.7) {$\color{acc}\alpha$};
\draw[gr, very thick] (-0.7, 1.8) rectangle (2.3, -0.6);
\node at (2, -0.9) {$\color{gr}\alpha\cdot\beta$};
\kmazidlo
czyli bierzemy $\beta$ kopii $\alpha$ - wygodniej na to patrzeć jak na takiego jerzyka z iloczynu \\kartezjańskiego.\medskip\\
Kilka przykładów:\smallskip\\
\indent $\omega+\omega=ot(\{1-\frac1{n+1}\;:\;n\in\N\}\cup\{2-\frac1{n+1}\;:\;n\in\N\}, \leq)$\smallskip\\
\indent $\omega+\omega+1=ot(\{1-\frac1{n+1}\;:\;n\in\N\}\cup\{2-\frac1{n+1}\;:\;n\in\N\}\cup\{3\}, \leq)$\smallskip\\
\indent $\omega\cdot\omega=ot(\{m-\frac1n\;:\;n,m\in\N\},\leq)$\bigskip\\
{\large\color{acc}WŁASNOŚCI DZIAŁAŃ NA LICZBACH PORZĄDKOWYCH}\medskip\\
\indent - dodawanie i mnożenie są \emph{łączne}\smallskip\\
\indent - nie są przemienne - \emph{kolejność jest ważna}
$$\omega+1\neq1+\omega=\omega$$
\indent - mnożenie jest \emph{rozdzielne }względem dodawania\medskip\\
\podz{dygresyja}\medskip
\begin{center}\large
    {\color{def}NASTĘPNIKIEM }liczby porządkowej $\alpha$ nazywamy liczbę porządkową $\alpha\cup\{\alpha\}=\alpha+1=\beta$:\smallskip\\
    $Succ(\beta)\iff(\exists\;\alpha)\;On(\alpha)\;\land\;\beta=\alpha+1$\medskip\\
    {\color{def}LICZBĄ GRANICZNĄ} nazywamy liczbę porządkową $Lim(\beta)$, \\jeśli nie jest ona następnikiem innej liczby.\medskip\\
    \normalsize Najmniejszą liczbą graniczną jest 0, kolejną jest $\omega$, a wszytkie liczby \\naturalne są następnikami.
\end{center}\bigskip

{\large
$$Lim(\alpha)\iff\alpha=\bigcup\alpha$$}
\dowod
$\implies$\medskip\\
Wiem, że $Lim(\alpha)$, czyli
$$\neg\;(\exists\;\beta)\;\alpha=\beta\cup\{\beta\}.$$
Jeśli założymy, że \bigskip\\
$\impliedby$\medskip\\
Ponieważ $Tran(\alpha)$, to również $Tran(\bigcup\alpha)$. Załóżmy, nie wprost, że $Succ(\alpha)$, czyli
$$(\exists\;\beta)\;\alpha=\beta\cup\{\beta\}.$$
Wtedy
$$\bigcup\alpha=\bigcup(\beta\cup\{\beta\})=\beta,$$
ale wówczas
$$\beta\cup\{\beta\}=\beta,$$
czyli wówczas $\beta\in\beta\cup\{\beta\}=\beta$, co daje nam sprzeczność.

\subsection{INDUKCJA POZASKOŃCZONA}
\begin{center}\large
    Niech $\varphi(n)$ będzie formułą języka teorii mnogości taką, że\smallskip\\
    $(\forall\;\beta)(\forall\;\alpha<\beta)\;\varphi(\alpha)\implies\varphi(\beta)$\smallskip\\
    Wtedy $(\forall\;\alpha)\varphi(\alpha)$.\medskip\\
    Jest to {\color{def}TWIERDZENIE O INDUKCJI POZASKOŃCZONEJ}
\end{center}
\dowod
Przypuśćmy, nie wprost, że
$$(\exists\;\alpha)\neg\;\varphi(\alpha).$$
Wtedy zbiór
$$C=\{\gamma\in\alpha\cup\{\alpha\}\;:\;\varphi(\gamma)\}$$
jest niepustym zbiorem liczb porządkowych. Wtedy w $C$ istnieje element najmniejszy $\xi$. \\Jego minimalność oznacza, że
$$(\forall\;\varepsilon<\xi)\;\varphi(\varepsilon).$$
Z założenia, że
$$(\forall\;\alpha)(\forall\;\beta<\alpha)\;\varphi(\beta)\implies\varphi(\alpha)$$
wynika, że $\varphi(\xi)$, czyli mamy sprzeczność z $\xi\in C$.
\kondow
{\color{def}Struktura indukcji}:\medskip\\
\indent 1. krok bazowy - sprawdzamy dla najmniejszej możliwej liczby\smallskip\\
\indent 2. krok indukcyjny:\smallskip\\
\indent\indent - krok następnikowy\smallskip\\
\indent\indent - krok graniczny

\subsection{REKURSJA POZASKOŃCZONA}
Od twierdzenia o indukcji różni się swoją istotą - indukcja służy dowodzeniu, a re-\\kursja - tworzeniu konstrukcji.\bigskip
\begin{center}\large
    Niech $\varphi(x,y)$ będzie formułą języka teorii mnogości taką, że\smallskip\\
    $(\forall\;x)(\exists\;!y)\;\varphi(x,y).$\smallskip\\
    Wówczas dla każdej liczby porządkowej $\alpha$ istnieje funkcja $f$ taka, że\smallskip\\
    $dom(f)=\alpha$\smallskip\\
    i spełniony jest warunek\smallskip\\
    $(\forall\;\beta<\alpha)\;\varphi(f\obet\beta, f(\beta))\quad(\kawa)$
\end{center}
Tworzymy pozaskończony ciąg indeksowany liczbami porządkowymi, gdzie kolejny krok wynika z tego co juz mamy.\bigskip\\
\dowod
{\large\color{def}JEDYNOŚĆ}\medskip\\
Przypuśćmy, że dla pewnego $\alpha$ istnieją dwie różne funkcje $f_1,\;f_2$ o dziedzinie $\alpha$ spełnia-\\jące $(\kawa)$. Wtedy zbiór jest niepusty
$$\{\beta\in\alpha\;:\;f_1(\beta)\neq f_2(\beta)\}\neq\emptyset.$$
Niech $\beta_0$ będzie najmniejszym elementem tego zbioru. Wtedy dla $\varepsilon<\beta_0$ mamy
$$f_1(\varepsilon)=f_2(\varepsilon),$$
czyli $f_1\obet\beta_0=f_2\obet\beta_0$, czyli z $(\kawa)$ i $fnc(\varphi)$
$$f_1(\beta_0)=f_2(\beta_0),$$
co daje sprzeczność.\bigskip\\
{\large\color{def}ISTNIENIE}\medskip\\
Indukcja po $\alpha$\medskip\\
1. $\alpha=0$ OK\medskip\\
2. Krok indukcyjny\smallskip\\
Ustalmy $\alpha$ takie, że dla $\gamma<\alpha$ istnieje funkcja taka, że $dom(f)_\gamma=\gamma$ i spełnia $(\kawa)$.\smallskip\\
\indent krok następnikowy $\alpha=\beta+1$\smallskip\\
Wtedy istnieje $f_\beta$ jak powyżej. Wiemy, że istnieje dokładnie jedno $y$ takie, że zachodzi
$$\varphi(f_\beta, y).$$
Niech 
$$f_\alpha=f_\beta\cup\{\parl\beta, y\parr\}.$$
Wtedy $fnc(f_\alpha)$ oraz 
$$dom(f_\alpha)=dom(f_\beta)\cup\{\beta\}=\beta\cup\{\beta\}=\beta+1=\alpha.$$
Wystarczy pokazać, że $f_\alpha$ spełnia $(\kawa)$. Trzeba ustalić jakieś 
$$\eta<\alpha=\beta+1.$$
Więc jeśli $\eta<\beta$, to $f_\alpha\obet\eta=f_\beta\obet\eta$ oraz $f_\alpha(\eta)=f_\beta(\eta)$. Czyli spełnia z założenia indukcyj-\\nego.. \\
Jeśli $\eta=\beta$, to mamy $\varphi(f_\alpha\obet\beta, f_\alpha(\beta))$, bo $f_\alpha(\beta)=y$, co również jest prawdziwe.\medskip\\
\indent krok graniczny $Lim(\alpha)$\smallskip\\
Wiemy, że
$$Lim(\alpha)\iff \alpha=\bigcup \alpha.$$

\newpage
\section{LICZBY KARDYNALNE}
Mamy kolekcję zbiorów, które wszystkie mają tę samą moc. Ale my byśmy chcieli wie-\\dzieć co to jest ta moc - liczby kardynalne pozwalają nam wybierać zbiory według ich \\mocy.\bigskip
\begin{center}\large
    {\color{def}LICZBA KARDYNALNA} to liczba porządkowa, \\która nie jest równoliczna z żadnym swoim elementem.\smallskip\\
    $Card(\alpha):=On(\alpha)\;\land\;(\forall\;\beta<\alpha)\;|\beta|<|\alpha|$
\end{center}\medskip
Zazwyczaj oznaczamy je $\kappa, \lambda$, chociaż kiedyś używało się gotyku.\bigskip\\
{\large\color{acc}Każda liczba kardynalna jest liczbą porządkową graniczną.}\medskip\\
$Card(0)$\smallskip\\
$Card(\omega)$, ale już $\neg\;Card(\omega+\omega)$, $\neg\;Card(\omega\cdot\omega)$ i $\neg\;Card(\omega^\omega)$.\smallskip\\
$(\forall\;n\in\omega)\;Card(n)$ - dowód później\bigskip\\
\podz{gr}\bigskip\\
\begin{center}\large
    Każdy zbiór jest równoliczny z pewną liczbą kardynalną.
\end{center}
\dowod
Ustalmy dowolny zbiór $X$. Wiemy, że $X$ można dobrze uporządkować przez $<$. Wtedy ist-\\nieje liczba porządkowa $\alpha$ z nim izomorficzna:
$$\varphi:X\xrightarrow[1-1]{izo}\alpha$$
W takim razie $\varphi$ jest bijekcją między $X$ a $\alpha$, więc
$$|X|=|\alpha|.$$
Niech
$$\kappa=\min\{\alpha\;:\;|\alpha|\geq|X|\}$$
Wtedy $\kappa\sim X$, a z minimalności $\kappa$ mamy $Card(\kappa)$.\medskip\\
Jeśli $|X|=|\kappa_1|$ i $|X|=|\kappa_2|$, to $|\kappa_1|=|\kappa_2|$.


NOWY WYKŁAD

\subsection{DZIAŁANIA NA LICZBACH KARDYNALNYCH}
\begin{center}\large
    Niech $\kappa$, $\lambda$ będą liczbami kardynalnymi, wtedy:\medskip\\
    $\kappa+\lambda = |(K\times\{0\})\cup(\lambda\times\{1\})|$\medskip\\
    $\kappa\cdot\lambda = |\kappa\times\lambda|$
\end{center}\bigskip
\podz{gr}\bigskip\\
\begin{center}\large
    Jeśli $\kappa\leq\omega$, to $\kappa\cdot\kappa = \kappa$
\end{center}
\dowod
Indukcja po liczbach kardynalnych lub po liczbach porządkowych - obie wersje będą pop-\\rawne.
\end{document}