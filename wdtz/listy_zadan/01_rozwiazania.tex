\documentclass{article}

\usepackage{../../notatka}

\begin{document}\ttfamily
\subsection*{Zad 1. Sprawdz, ze $\langle a, b\rangle = \langle c,d\rangle\implies a=c\;\land\;b=d$}
    Z definicji pary uporzadkowanej wg. Kuratowskiego:
    $$\langle a,b\rangle = \{\{a\}, \{a,b\}\}$$
    DOWOD:\medskip\\
    Ustalmy dowolne $abcd$ takie, ze $\langle a,b\rangle =\langle c,d\rangle$. Wowczas
    $$\{\{a\},\{a,b\}\}=\{\{c\}, \{c,d\}\}$$
    Rozpatrzmy pzypadki:\medskip\\
    {\color{acc}1. $a=b$}\smallskip\\
        Wtedy mamy 
        $$\{\{a\}, \{a,a\}\}=\{\{a\}\}=\{\{c\}, \{c, d\}\}$$
        i wtedy z aksjomatu ekstencjonalnosci
        $$\{a\}=\{c\}=\{c,d\}$$
        wiec $a=c=d$, czyli $a=c\;\land\;b=d$.\medskip\\
    {\color{acc}2. $a\neq b$}\smallskip\\
        Wtedy $\{a\}\neq \{a,b\}$, stad wnioskujemy 
        $$\{c\}=\{a\},$$ 
        wiec $c=a$.\smallskip\\
        Dalej zauwazamy, ze $\{a,b\}\neq\{c\}$, bo $c=a\neq b$, wiec
        $$\{a,b\}=\{c,d\}=\{a,d\}$$
        i poniewaz $a\neq b$, to $b=d$.
\subsection*{Zad 2. Udowodnij, ze $\bigcup\Po(A)=A$.}
    DOWOD:\medskip\\
    {\color{acc}1. $\bigcup\Po(A)\supseteq A$}\smallskip\\
    Ustalmy dowolne $x\in A$. \emph{Chcemy pokazac, ze $x\in \bigcup\Po(A)$.} Zauwazmy, ze 
    $$A\in\Po(A),$$ 
    wiec z definicji sumy otrzymujemy
    $$x\in\bigcup\Po(A).\medskip$$
    {\color{acc}2. $\bigcup\Po(A)\subseteq A$}\smallskip\\
    Ustalmy dowolne $x\in\bigcup\Po(A)$. Wowczas istnieje $B\in \Po(A)$ takie, ze 
    $$x\in B.$$ 
    Z definicji zbioru potegowego 
    $$B\subseteq A,$$ 
    zatem z definicji zawierania $x\in A$.
\subsection*{Zad 3. Niech $A$ bedzie zbiorem niepustym. Ktore z ponizszych twierdzen sa prawdziwe?}
    {\large\color{tit}Jesli $A=\bigcup A$, to $\emptyset\in A$.}\medskip\\
    Teza $A=\bigcup A\implies\emptyset\in A$.\medskip\\
    Z aksjomatu regularnosci wiemy, ze istnieje $x\in A$ taki, ze
    $$(\heartsuit)\forall\;a\in A\quad\neg\;(y\in x).$$
    Gdyby $\emptyset\neq x$, to istnialoby $z\in x$. Poniewaz $z\in x$ i $x\in A$, to $$z\in\bigcup A,$$
    czyli z zalozenia mamy $z\in A$, co jest sprzeczne z $(\heartsuit)$. Wobec tego $x=\emptyset\in A$.\bigskip\\
    {\large\color{tit}Jesli $\emptyset\in A$, to $A=\bigcup A$.}\medskip\\
    NIE: Niech $A=\{\emptyset\}$. Wowczas $\emptyset\in \{\emptyset\}$ i $\bigcup A=\emptyset\neq\{\emptyset\}=A$\bigskip\\
    {\large\color{tit}Jesli $\bigcup A=\bigcap A$, to $A=\{x\}$ dla pewnego $x$}\medskip\\
    Teza: $\bigcup A=\bigcap A \implies \exists\;x\quad A=\{x\}$\medskip\\
    Niech $x\in A$. Zalozmy nie wprost, ze istnieje $y\in A$ takie, ze $y\neq x$. Bez straty ogoolnosci mozemy zalozyc, ze istnieje $t\in x$ i $t\notin y$.\smallskip\\
    Z definicji sumy $t\in \bigcup A$, a z drugiej strony, z definicji przekroju, $t\notin \bigcap A$. Czyli $\bigcap A\neq \bigcup A$ i otrzymujemy sprzecznosc z zalozeniem.
\subsection*{Zad 4. Ktora z ponizszych rownosci zachodzi dla dowolnego zbioru $A$?}
    {\large\color{tit}$\bigcap\{\Po(B)\;:\;B\subseteq A\}=\{\bigcap\Po(B)\;:\;B\subseteq A\}$}\medskip\\
    Po lewej szukamy wspolnego elementu wszystkich podzbiorow zbioru $A$ - jest to \O. Z prawej strony mam rodzine wszystkich przekrojow. Czyli zeby byc podzbiorem wszystkich podzbiorow zbioru $A$ trzeba byc \O\bigskip\\
    {\large\color{tit}$\bigcap\{\Po(B)\;:\;B\subseteq A\}=\{\bigcap\Po(B)\;:\;B\subseteq A\}$}
    \medskip\\
    $\Po(A)$
\subsection*{Zad 5. Udowodnij, ze aksjomat pary wynika z pozostalych aksjomatow teorii $ZF_0$.}
    Bierzemy zbior induktywny i pzetlaczamy go przez odpowiednia funkcje.\medskip\\
    Ustalmy dwa dowoolne $x,y$. Rozwazmy zbior
    $$p=\{z\;:\;\exists\;t\in\omega\quad(t=\emptyset\land z=x)\lor(t\neq \emptyset\land z=y)\}$$
    na mocy aksjomatu zastepowania.\smallskip\\
    Ustalmy dowolne $z$. Mamy 
    \begin{align*}
        z\in p&\iff\exists\;t\in\omega\quad(t=\emptyset\land z=x)\lor(t\neq \emptyset\land z=y)\iff\\
        &\iff\exists\;t\in\omega\quad(t=\emptyset\land z=x)\lor\exists\;t\in \omega\quad (t\neq \emptyset\land z=y)\\
        &dokonczyc\;przeksztalcanie\\
        &\iff z=x\lor z=y
    \end{align*}
    Musimy wziac zbior min 2-el i otrzymac pare jako el tego zbioru.\smallskip\\
    !!nie ma w jezyku \O, czyli musimy to zastapic nieuzywajac znaczka
\subsection*{Zad 6.}
    Mamyzbior $A$ i formule jezyka TM $\varphi$. Rozwazmy dwa przypadki\medskip\\
    \indent 1.$\forall\;x\in A\quad \neg\;\varphi(x)$. Wtedy
    $$\{x\in A\;:\;\varphi(x)\}=\emptyset$$
    \indent 2. $\exists\;x\in A\quad \varphi(x)$. Niech $x$ bedzie tym istniejacym elementem $A$, wtedy
    $$\psi(t,z,p)=(\varphi(t)\land z=t)\lor(\neg\;\varphi(t)\land z=p).$$
    Mamy formule i mamy parametr - teraz bedziemy stosowac te formule do tego parametru.\medskip\\
    Niech $b$ bedzie zbiorem istniejacym na mocy aksjomatu zastepowania.
    \begin{align*}
        z\in b&\iff (\exists\;t\in a\quad (\varphi(t)\land z=t)\lor(\neg\;\varphi(t)\land z=p))\iff\\
        &\iff\exists\;t\in A\quad (\varphi(t)\land z=t)\lor(\exists\;t\in A\quad(\neg\;\varphi(t)\land z=x))\iff\\
        &\iff(\exists\;t\quad (z\in A\land \varphi(z)))\lor(\exists\;t\quad t\in A\land\neg\;\varphi(t)\land z=x)\iff\\
        &\iff (z\in A\land\varphi(z))\lor z=x\iff\\
        &\iff z\in A\land \varphi(z)
    \end{align*}
\subsection*{Zad 9}
Latwiej jest zfunkcji wyboru (FW) w selektor (AC).\bigskip\\
{\color{acc}FW $\implies$ AC}\medskip\\
Niech $\mathcal{A}$ bedzie niepusta, rozlaczna rodzina zbiorow niepustych. Chcemy dla tej rodziny znalezc slektor. Wiemy, ze istniejedla niej funkcja.\smallskip\\
Niech 
$$F:\mathcal{A}\to\bigcup\mathcal{A}$$
bedzie funkcja wyboru rodziny $\mathcal{A}$, czyli
$$\forall\;A\in\mathcal{A}\quad F(A)\in A.$$
Niech $S = \rng F$. $S$ jest selekorem, bo\medskip\\
\indent 1. dla dowolnego $A\in\mathcal{A}$ mamy $|A\cap S\geq 1$ (bo $F(A)\in A\rng F$)\smallskip\\
\indent 2. dla dowolnego $A\in\mathcal{A}$ mamy $|A\cap S|\leq 1$, bo $\mathcal{A}$ jest rozlaczne (gdyby $F(A_1)\in A\cap S$ dla pewnego $A_1\in\mathcal{A}$, to poniewaz $F(A_1)\in A_1$, wiec $A\cap A_1\neq\emptyset$ - sprzecznosc).\bigskip\\
{\color{acc}AC$\implies$FW}\medskip\\
Ustalmy dowolna rodzine zbiorow niepustych $\mathcal{A}$. \smallskip\\
Rozwazmy rodzine
$$\mathcal{A}'=\{\{A\}\times A\;:\;A\in\mathcal{A}\}$$
i ta rodzina jest parami rozlaczna - kazdy ze zbiorow zostaje wysuniety na inny poziom. Do tej rodziny mozemy teraz zastosowac aksjomat wyboru, czyli istnieje dla niej selektor $S$.\smallskip\\
Okazuje sie, ze $S$ sam w sobie jest funkcja wyboru rodziny $\mathcal{A}$:
$$|S\cap (\{A\}\times A)|=1,$$
wiec $S$ jest zbiorem par $\{A\}\times A$, czyli funkcja gdzie $\dom S= \mathcal{A}$, a $\rng S=\bigcup\mathcal{A}$.
\end{document}