\documentclass{article}

\usepackage{../../../notatka}

\begin{document}\ttfamily
\subsection*{2. Udowodnij, ze $\bigcup\Po{\rodz A} = \rodz A$}
$\supseteq$\smallskip\\
Wezmy dowolny $x\in\rodz A$. Z aksjomatu zbioru potegowego wiemy, ze
$$(\exists\;y\in\Po{\rodz A})\;x\in y\subseteq \rodz A$$
Dalej, na mocy aksjomatu sumy wiemy, ze
$$x\in\bigcup\{y\}\subseteq \bigcup\Po{\rodz A},$$
czyli
$$x\in\bigcup\Po{\rodz A}$$
$\subseteq$\smallskip\\
Wezmy dowolny $x\in\bigcup\Po{\rodz A}$. Z aksojatu sumy wiemy, ze
$$(\exists\;y\in\Po{\rodz A})\;x\in y\in\Po{\rodz A},$$
ale z aksjomatu zbioru potegowego wiemy, ze
$$y\in\Po{\rodz A}\iff y\subseteq\rodz A.$$
W takim razie
$$x\in y\subseteq\rodz A\implies x\in\rodz A$$
\kondow

\subsection*{3. Niech $A$ bedzie zbiorem niepustym. Ktore z ponizszych twierdzen sa prawdziwe?}
\indent (a) Jesli $A=\bigcup A$, to $\emptyset\in A$\smallskip\\
Nie, poniewaz
$$A=\bigcup A$$
na przyklad jesli 
$$A=\emptyset.$$
A z aksjomatu ekstensjonalnosci wiemy, ze $\emptyset\notin\emptyset$.

\indent (b) Jesli $\emptyset\in A$, to $A=\bigcup A$\smallskip\\
Nie, wezmy na przyklad
$$A=\{\emptyset, \{7\}\}.$$
Wtedy
$$\bigcup A=\{7\}\supseteq\emptyset,$$
ale
$$\emptyset\notin\{7\}=\bigcup A.$$

\indent (c) Jesli $\bigcup A=\bigcap A$, to $A=\{x\}$ dla pewnego $x$.
TAK:\smallskip\\
\begin{align*}
    x\in \bigcup A&\iff(\exists\;y\subseteq A)\;x\in y\\
    x\in \bigcap A&\iff(\forall\;y\subseteq A)\;x\in y\\
    U&=\bigcup A=\bigcap A\\
    ((x\in U\iff(\exists\;y\subseteq A)\;x\in y)&\iff(x\in U\iff(\forall\;y\subseteq A)\;x\in A))\implies (\exists\;x)\;A=\{x\}
\end{align*}

\subsection*{4. Ktora z ponizszych rownosci zachodzi dla dowolnego zbioru $A$?}
\indent (a) $\bigcap\{\Po{B}\;:\;B\subseteq A\}=\{\bigcap\Po{B}\;:\;B\subseteq A\}$\smallskip\\
NIE:\smallskip\\
$$A=\{1\}$$
$$\bigcap \{\Po{B}\;:\;B\subseteq A\}=\bigcap\{\{1\}, \emptyset\}=\emptyset$$
$$\{\bigcap\Po{B}\;:\;B\subseteq A\}=\{\bigcap\{\{1\}, \emptyset\}, \bigcap\{\emptyset\}\}=\{\emptyset, \emptyset\}=\{\emptyset\}$$
$$\emptyset\neq\{\emptyset\}$$

\indent (b) $\bigcup\{\Po{B}\;:\;B\subseteq A\}=\{\bigcup\Po{B}\;:\;B\subseteq A\}$\smallskip\\
TAK:\smallskip\\
Pokazemy, ze $L= \Po{A}= P$\smallskip\\
$L=\Po{A}$\smallskip\\
Z aksjomatu sumy wiemy, ze
\begin{align*}
    x\in \bigcup \{\Po{B}\;:\;B\subseteq A\}&\iff((\exists\;t\in\{\Po{B}\;:\;B\subseteq A\})\;x\in t)\iff\\
    &\iff((\exists\;t\subseteq A)\;x\in t)\iff\\
    &\iff x\in \Po{A}
\end{align*}
$P=\Po{A}$
\begin{align*}
    x\in\{\bigcup\Po{B}\;:\;B\subseteq A\}&\iff x\in \{y\;:\;((\exists\;z\subseteq \Po{B})\;y\in z)\land B\subseteq A\}\iff\\
    &\iff x\in \{y\;:\;y\in\Po{A}\}\iff\\
    &\iff x\in\Po{A}
\end{align*}
\kondow

\subsection*{4.5 Sprawdzic, ze para nieuporzadkowana, suma i zbior potegowy sa zdefiniowane jednoznacznie.}
\indent para nieuporzadkowana jest zdefiniowana jednoznacznie\smallskip\\
Zalozmy niewprost, ze istnieja pary
$$\{a, b\}=\{c, d\}$$
\begin{align*}
    &((x\in\{a,b\}\iff (x=a\lor x=b))\iff (x\in \{c, d\}\iff (x=c\lor x=d)))\iff\\
    \iff&((x=a\lor x=b)\iff x=c\lor x=d)\iff ((a=c\land b=d)\lor (a=d\land b=c))
\end{align*}
i cyraneczka komutuje
\kondow

\subsection*{5. Udowodnij, ze aksjomat pary wynika z pozostalych aksjomatow teorii ZF$_0$.}

Z aksjomatu zbioru pustego i zbioru potegowego mozemy skonstruowac zbior
$$P =\Po{\Po{\emptyset}} = \Po{\{\emptyset\}} = \{\emptyset, \{\emptyset\}\}$$

W wersji uproszczonej chcemy napisac formule $\varphi(t, z, a, b)$, ktora 
$$(\forall\;t)(\exists\;!y)\;\varphi(t, z, a, b)$$
$$(t=\emptyset\land z = a)\lor (t=\{\emptyset\}\land z = b)\lor (t\neq \emptyset\land t\neq\{\emptyset\} \land x = \emptyset)$$
Wersja krotsza
$$(t=\emptyset\land z=a)\lor(t\neq\emptyset\land z=b)$$

Czyli z aksjomatu zastepowania nasza funkcja produkuje zbior $y=\{a, b\}$
$$(\forall\;a)(\forall\;b)(\forall\;x)(\exists\;y)(\forall\;z)(z\in y\iff (\exists\;t\in x)\;\varphi(t,z, a, b))$$

\subsection*{6. Udowodnij, ze aksjomat wyrozniania wynika z pozostalych aksjomatow teorii ZF$_0$}

Wezmy dowolny zbior $x$ i dowolna formule $\varphi(t, \overline a)$.\bigskip\\

Rozpatrzmy dwa przypadki:\medskip\\
\indent 1. Nie istnieje $y\in x$ takie, ze $\varphi(y, \overline a) == true$. Wtedy po przefiltrowaniu mam $\emptyset$.\medskip\\
\indent 2. Istnieje $y\in x$ takie, ze $\varphi(y,\overline z) == true$. Wowczas konstrukcja mojego przefiltrowanego zbioru uzywajaca aksjomat zastepowania i formule $\sigma(t, z, \overline a)$
$$(t= z\;\land\; \varphi(t, \overline a))\lor(y= z\;\land\; \neg\;\varphi(t, \overline a))$$

\subsection*{7. Udowodnij (w teori ZF), ze $\neg\;(\exists\;x_1, ..., x_n)\;x_1\in x_2\in...\in x_n\in x_1$}

Poniewaz jestesmy ekstra upierdliwi, to konstruujemy sobie zbiorek $s=\{x_1, ..., x_n\}$. Wezmy pare
$$\{x_1, x_1\}$$
i jego sume $\bigcup \{x_1, x_1\}$. Dalej $\bigcup\{\{x_1, x_1\}, \{x_2, x_2\}\}$ i znowu sume tego. Analogicznie dalej.\bigskip\\

Z aksjomatu regularnosci wiemy, ze istnieje takie $k$, ze $x_k$ jest elementem $\in$-minimalnym utworzonego wyzej zbioru. W takim razie
$$(\forall\;y\in s)\;y\notin x_k.$$
Rozwazmy dwa przypadki:\medskip\\
\indent 1. $k=1$\smallskip\\
Wtedy $x_n\notin x_1$ i mamy sprzecznosc. \medskip\\
\indent 2. $k\neq 1$\smallskip\\
Wtedy $x_{k-1}\notin x_k$ i mamy sprzecznosc.

\subsection*{8. Udowodnij (w teorii ZF), ze \\$\neg(\exists\;f)(fnc(f)\;\land\;dom(f)=\omega\;\land\;(\forall\;n\in\omega) f(n+1)\in f(n))$}

Zalozmy, ze $f$ jest funckja, dla ktorej ta formula nie smiga.\smallskip\\

$\triangle$ sa affiniczne <3 \bigskip\\

Zbiorek $p_{ysio}=\{f(0), f(1), ...\}$ tworzymy zastepujac przy pomocy aksjomatu zastepowania elementy dziedziny $dom(f)$ przez odpowiadajace im elementy obrazu, gdzie formula bylaby tak naprawde relacja rownowazna naszej funkcyji ($\psi(t, z, f)$)
$$z=f(t)$$

W takim razie istnieje $k$ takie, ze $f(k)$ jest elementem $\in$-minimalnym zbioru $p_{ysio}$. Ale w takim razie nie moze zajsc $f(k+1)\in f(k)$. Wiec mamy sprzecznosc <3

\subsection*{9. Udowodnij, ze aksjomat wyboru jest rownowazny zdaniu\\
$(\forall\;x)(((\forall\;y\in x)\;y\neq\emptyset)\implies (\exists\;f)\;fnc(f)\;\land\;dom(f)=x\;\land\;(\forall\;y\in x)f(y)\in y)$}

AKSJOMAT $\iff$ FUNCKJA\medskip\\
$\impliedby$\smallskip\\
Czyli potrzebuje skonstruowac majac funckje zbior wartosci tejze funkcji. Robie to zastepujac elementy dziedziny przez ich wartosci na mocy aksjomatu zastepowania. Wystarczy pokazac, ze to rzeczywiscie jest selektor. 
$$(\forall\;y\in x)(\exists\;!t)\;t\in s\cap y$$

Wezmy dowolne $y\in x$. Z definicji funkcji wiemy, ze $f(y)\in y$. W takim razie $f(y)\cap y = f(y)$. \smallskip\\
Dlaczego jest to jedyne? Jesli istnialyby dwa $t_1, t_2$ bedace w selektorze i w $y$, to wowczas $f^{-1}[t_1]\neq f^{-1}[t_2]$, bo $f$ jest funkcja. Ale poniewaz $x$ jest rozlaczna rodzina zbiorow, to nie moze byc, ze sie pokryja dwa zbiory zeby oba mialy $t_1$ i $t_2$. sprzecznosc.\medskip

$\implies$\smallskip\\
Na mocy aksjomatu wyboru biere selektor $s$ z rodziny $x$. Teraz chce zastapic jego elementy parami uporzadkowanymi gdzie poprzednik to zbior do ktorego dany element nalezy, a nastepnik to on sam. Piszemy formuly $\theta(t, z, x, s)$
$$(t\notin s\; \land \;z=\emptyset)\lor((\exists\;y\in x) \;t\in y\;\land\;z=\parl t, y\parr)$$
alternatywnie, biore $x\times s$ i filtruje na mocy aksjomatu wyrozniania przy pomocy formuly $\rho(t, x, s)$
$$s\cap\bigcup x \in x\cap t$$
\kondow

\subsection*{10. Udowodnij, ze aksjomat wyboru jest rownowazny faktowi, ze jesli $\parl X_i\;:\;i\in I\parr$ jest niepusta rodzina zbiorow niepustych, to iloczyn kartezjanski $\prod\limits_{i\in I}X_i$ jest niepusty. Sformuluj powyzszy fakt bez uzycia pojecia rodziny indeksowanej.}

AKSJOMAT WYBORU $\iff$ NIEPUSTE

$\impliedby$\smallskip\\
Wezmy dowolna rozlaczna rodzine zbiorow niepustych $\rodz X=\{X_i\;:\;i\in I\}$\smallskip\\
Wiemy, ze to jest niepusty zbior funkcji z indeksow w sumy zbiorow $X_i$, wiec moge wziac z niego pewna funkcje, $m$. Do mojego selektora chce wrzucic po jednym elemencie z kazdego elementu mojej rodziny, wiec zrobie to wrzucajac do niego wartosci $m$ dla poszczegolnych $i\in I$, bo mam aksjomat zastepowania (zastepuje $m$ przez $m(i)$ - funkcje jej obrazem).\smallskip\\
Pokazemy, ze jest to faktycznie selektor.\\
Wezmy dowolny element $o\in\rodz X$. Chce pokazac, ze $o\cap m(i)$ ma tylko jeden element. Istnieje co najmniej jeden, bo istnieje $i\in I$ takie, ze
$$m(i)\in o.$$
Z drugiej strony, jesli istnialoby wiecej niz jeden, to istnialyby $i, j\in I$ takie, ze $i\neq j$ oraz
$$m(i)\in o\;m(j)\in o,$$
to wowczas elementy $\rodz X$ nie bylyby rozlaczne, tzn $X_i\cap X_j\neq \emptyset$.\medskip

$\implies$\smallskip\\
Mam podana niepusta indeksowana rodzine niepustych zbiorow.\smallskip\\
Z 9 mamy, ze aksjomat wyboru jest ronowazny funkcji wyboru, wiec polecmy z funkcja wyboru.\smallskip\\
Wiem, ze istnieje funkcja wyboru z $\rodz X$. Nalezy do iloczynu kartezjanskiego, tj jej domena jest $I$, a zbiorem wartosci jest suma $\rodz X$. Czyli moge zrobic druga funkcje, ktora dla $i\in I$ daje mi konkretnie element z $X_i$. Jest to element zbioru kartezjanskiego. Korzystam z aksjomatu zastepowania.

\subsection*{11. Rozwazmy indeksowana rodzine zbiorow niepustych $\parl A_i\;:\;i\in I\parr$, taka, ze $\bigcap\limits_{i\in I}A_i\neq\emptyset$. Czy do niepustosci $\prod\limits_{i\in I}A_i$ potrzebujemy aksjomatu wyboru?}



\end{document}