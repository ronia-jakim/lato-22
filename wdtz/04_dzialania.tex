\section{DZIAŁANIA NA LICZBACH PORZĄDKOWYCH}
\subsection{DEFINICJA DODAWANIA I MNOŻENIA}
Niech $\alpha,\;\beta$ będą liczbami porządkowymi. Wówczas {\color{def}dodawanie definiujemy}:
{\large$$\alpha+\beta=ot(\alpha\times\{0\}\cup\beta\times\{1\}, \leq)$$}
\pmazidlo
\draw[acc, ultra thick] (0, 0) -- (2, 0);
\node at (1, -0.3) {$\color{acc}\alpha$};
\draw[def, ultra thick] (1.8, 1.3) -- (3.8, 1.3);
\node at (2.8, 1) {$\color{def}\beta$};
\node at (-0.3, 0) {0};
\node at (1.5, 1.3) {1};
\draw[gr, very thick] (-0.5, 1.8) rectangle (4.1, -0.5);
\node at (3.8, -0.8) {$\color{gr}\alpha+\beta$};
\kmazidlo
czyli najpierw rozdzielamy je, a potem sumujemy. Relację porządku na sumie liczb po-\\rządkowych definiujemy (porządek leksykograficzny):
{\large$$\parl\gamma, i\parr\leq_{lex}\parl\xi, j\parr\iff i<j\;\lor\; (i=j\;\land\;\gamma<\xi).$$}
{\color{def}Mnożenie liczb porządkowych} to z kolei typ porządkowy ich iloczynu z porządkiem leksy-\\kograficznym:
{\large$$\alpha\cdot\beta=ot(\beta\times\alpha,\leq_{lex})$$}
\pmazidlo
\draw[acc, ultra thick] (0, 0)--(0, 1.5);
\draw[acc, ultra thick] (0.4, 0)--(0.4, 1.5);
\draw[acc, ultra thick] (0.8, 0)--(0.8, 1.5);
\draw[acc, ultra thick] (1.2, 0)--(1.2, 1.5);
\draw[acc, ultra thick] (1.6, 0)--(1.6, 1.5);
\draw[acc, ultra thick] (2, 0)--(2, 1.5);
\draw[def, ultra thick] (0, 0)--(2, 0);
\node at (1, -0.4) {$\color{def}\beta$};
\node at (-0.3, 0.7) {$\color{acc}\alpha$};
\draw[gr, very thick] (-0.7, 1.8) rectangle (2.3, -0.6);
\node at (2, -0.9) {$\color{gr}\alpha\cdot\beta$};
\kmazidlo
czyli bierzemy $\beta$ kopii $\alpha$ - wygodniej na to patrzeć jak na takiego jerzyka z iloczynu \\kartezjańskiego.\medskip\\
Kilka przykładów:\smallskip\\
\indent $\omega+\omega=ot(\{1-\frac1{n+1}\;:\;n\in\N\}\cup\{2-\frac1{n+1}\;:\;n\in\N\}, \leq)$\smallskip\\
\indent $\omega+\omega+1=ot(\{1-\frac1{n+1}\;:\;n\in\N\}\cup\{2-\frac1{n+1}\;:\;n\in\N\}\cup\{3\}, \leq)$\smallskip\\
\indent $\omega\cdot\omega=ot(\{m-\frac1n\;:\;n,m\in\N\},\leq)$\bigskip\\
{\large\color{acc}WŁASNOŚCI DZIAŁAŃ NA LICZBACH PORZĄDKOWYCH}\medskip\\
\indent - dodawanie i mnożenie są \emph{łączne}\smallskip\\
\indent - nie są przemienne - \emph{kolejność jest ważna}
$$\omega+1\neq1+\omega=\omega$$
\indent - mnożenie jest \emph{rozdzielne }względem dodawania\medskip\\
\podz{dygresyja}\medskip
\begin{center}\large
    {\color{def}NASTĘPNIKIEM }liczby porządkowej $\alpha$ nazywamy liczbę porządkową $\alpha\cup\{\alpha\}=\alpha+1=\beta$:\smallskip\\
    $Succ(\beta)\iff(\exists\;\alpha)\;On(\alpha)\;\land\;\beta=\alpha+1$\medskip\\
    {\color{def}LICZBĄ GRANICZNĄ} nazywamy liczbę porządkową $Lim(\beta)$, \\jeśli nie jest ona następnikiem innej liczby.\medskip\\
    \normalsize Najmniejszą liczbą graniczną jest 0, kolejną jest $\omega$, a wszytkie liczby \\naturalne są następnikami.
\end{center}\bigskip

{\large
$$Lim(\alpha)\iff\alpha=\bigcup\alpha$$}
\dowod
$\implies$\medskip\\
Wiem, że $Lim(\alpha)$, czyli
$$\neg\;(\exists\;\beta)\;\alpha=\beta\cup\{\beta\}.$$
Jeśli założymy, że \bigskip\\
$\impliedby$\medskip\\
Ponieważ $Tran(\alpha)$, to również $Tran(\bigcup\alpha)$. Załóżmy, nie wprost, że $Succ(\alpha)$, czyli
$$(\exists\;\beta)\;\alpha=\beta\cup\{\beta\}.$$
Wtedy
$$\bigcup\alpha=\bigcup(\beta\cup\{\beta\})=\beta,$$
ale wówczas
$$\beta\cup\{\beta\}=\beta,$$
czyli wówczas $\beta\in\beta\cup\{\beta\}=\beta$, co daje nam sprzeczność.

\subsection{INDUKCJA POZASKOŃCZONA}
\begin{center}\large
    Niech $\varphi(n)$ będzie formułą języka teorii mnogości taką, że\smallskip\\
    $(\forall\;\beta)(\forall\;\alpha<\beta)\;\varphi(\alpha)\implies\varphi(\beta)$\smallskip\\
    Wtedy $(\forall\;\alpha)\varphi(\alpha)$.\medskip\\
    Jest to {\color{def}TWIERDZENIE O INDUKCJI POZASKOŃCZONEJ}
\end{center}
\dowod
Przypuśćmy, nie wprost, że
$$(\exists\;\alpha)\neg\;\varphi(\alpha).$$
Wtedy zbiór
$$C=\{\gamma\in\alpha\cup\{\alpha\}\;:\;\varphi(\gamma)\}$$
jest niepustym zbiorem liczb porządkowych. Wtedy w $C$ istnieje element najmniejszy $\xi$. \\Jego minimalność oznacza, że
$$(\forall\;\varepsilon<\xi)\;\varphi(\varepsilon).$$
Z założenia, że
$$(\forall\;\alpha)(\forall\;\beta<\alpha)\;\varphi(\beta)\implies\varphi(\alpha)$$
wynika, że $\varphi(\xi)$, czyli mamy sprzeczność z $\xi\in C$.
\kondow
{\color{def}Struktura indukcji}:\medskip\\
\indent 1. krok bazowy - sprawdzamy dla najmniejszej możliwej liczby\smallskip\\
\indent 2. krok indukcyjny:\smallskip\\
\indent\indent - krok następnikowy\smallskip\\
\indent\indent - krok graniczny

\subsection{REKURSJA POZASKOŃCZONA}
Od twierdzenia o indukcji różni się swoją istotą - indukcja służy dowodzeniu, a re-\\kursja - tworzeniu konstrukcji.\bigskip
\begin{center}\large
    Niech $\varphi(x,y)$ będzie formułą języka teorii mnogości taką, że\smallskip\\
    $(\forall\;x)(\exists\;!y)\;\varphi(x,y).$\smallskip\\
    Wówczas dla każdej liczby porządkowej $\alpha$ istnieje funkcja $f$ taka, że\smallskip\\
    $dom(f)=\alpha$\smallskip\\
    i spełniony jest warunek\smallskip\\
    $(\forall\;\beta<\alpha)\;\varphi(f\obet\beta, f(\beta))\quad(\kawa)$
\end{center}
Tworzymy pozaskończony ciąg indeksowany liczbami porządkowymi, gdzie kolejny krok wynika z tego co juz mamy.\bigskip\\
\dowod
{\large\color{def}JEDYNOŚĆ}\medskip\\
Przypuśćmy, że dla pewnego $\alpha$ istnieją dwie różne funkcje $f_1,\;f_2$ o dziedzinie $\alpha$ spełnia-\\jące $(\kawa)$. Wtedy zbiór jest niepusty
$$\{\beta\in\alpha\;:\;f_1(\beta)\neq f_2(\beta)\}\neq\emptyset.$$
Niech $\beta_0$ będzie najmniejszym elementem tego zbioru. Wtedy dla $\varepsilon<\beta_0$ mamy
$$f_1(\varepsilon)=f_2(\varepsilon),$$
czyli $f_1\obet\beta_0=f_2\obet\beta_0$, czyli z $(\kawa)$ i $fnc(\varphi)$
$$f_1(\beta_0)=f_2(\beta_0),$$
co daje sprzeczność.\bigskip\\
{\large\color{def}ISTNIENIE}\medskip\\
Indukcja po $\alpha$\medskip\\
1. $\alpha=0$ OK\medskip\\
2. Krok indukcyjny\smallskip\\
Ustalmy $\alpha$ takie, że dla $\gamma<\alpha$ istnieje funkcja taka, że $dom(f)_\gamma=\gamma$ i spełnia $(\kawa)$.\smallskip\\
\indent {\color{acc}krok następnikowy} $\alpha=\beta+1$\smallskip\\
Wtedy istnieje $f_\beta$ jak powyżej. Wiemy, że istnieje dokładnie jedno $y$ takie, że zachodzi
$$\varphi(f_\beta, y).$$
Niech 
$$f_\alpha=f_\beta\cup\{\parl\beta, y\parr\}.$$
Wtedy $fnc(f_\alpha)$ oraz 
$$dom(f_\alpha)=dom(f_\beta)\cup\{\beta\}=\beta\cup\{\beta\}=\beta+1=\alpha.$$
Wystarczy pokazać, że $f_\alpha$ spełnia $(\kawa)$. Trzeba ustalić jakieś 
$$\eta<\alpha=\beta+1.$$
Więc jeśli $\eta<\beta$, to $f_\alpha\obet\eta=f_\beta\obet\eta$ oraz $f_\alpha(\eta)=f_\beta(\eta)$. Czyli spełnia z założenia indukcyj-\\nego.. \\
Jeśli $\eta=\beta$, to mamy $\varphi(f_\alpha\obet\beta, f_\alpha(\beta))$, bo $f_\alpha(\beta)=y$, co również jest prawdziwe.\medskip\\
\indent {\color{acc}krok graniczny} $Lim(\alpha)$\smallskip\\
Wiemy, że
$$Lim(\alpha)\iff \alpha=\bigcup \alpha.$$
Czyli
$$Lim(\alpha)\iff\neg\;(\exists\;\beta)\;\alpha=\beta+1\iff(\forall\;\beta<\alpha)\;\beta+1<\alpha$$
{\color{cyan}COOO?}\medskip\\
Zauważmy, że jeśli $\gamma_!,\gamma_1<\alpha$ i $\gamma_1\subseteq\gamma_2$, to $f\obet\gamma_1\subseteq f\obet\gamma_2$ - \emph{krótsze funkcje są wydłużane przez funkcje dłuższe}. \medskip\\
Niech $f=\bigcup f_\gamma$. Wtedy $f$ jest funkcją oraz 
$$dom \;f=\bigcup\limits_{\gamma<\alpha}dom\;f_\gamma=\bigcup\limits_{\gamma<\alpha}\gamma=\alpha.$$
Spełnia (\kawa): ustalmy dowolne $\beta<\alpha$. Skoro $\beta<\alpha$, to $f\obet\beta=f_{\beta+1}\obet\beta$. Ponadto $f(\beta)=f_{\beta+1}(\beta).$ Ale $\beta+1$ też jest jedną z liczb występujących w $\bigcup\limits_{\gamma<\alpha}\gamma$, a dla nich zachodzi (\kawa).{\color{cyan}CO JA TUTAJ NAPISAŁAM?}

\subsection{REKURENCYJNA DEFINICJA DODAWANIA I MNOŻENIA}
{\large\color{def}DODAWANIE}:
\begin{align*}
    \alpha+0 &= \alpha\\
    \alpha+(\beta+1)&=(\alpha+\beta)+1\\
    \alpha+\gamma&=\bigcup\limits_{\xi<\gamma}(\alpha+\xi)\quad Lim(\gamma)
\end{align*}
\dowod
Udowodnimy, że jest to definicja równoważna z definicją iteracyjną korzystając z indu-\\kcji po $\beta$.
1. $\beta = 0$\smallskip\\
rekursyjnie: $\alpha+0=\alpha$\smallskip\\
iteracyjnie: $\alpha+0=ot(\alpha\times\{0\}\cup\emptyset\times\{1\})=\alpha$\medskip\\
2. Krok indukcyjny\medskip\\
\indent {\color{acc}krok następnikowy} $\beta=\gamma+1$\smallskip\\
Rekurencyjnie wiemy, że 
$$\alpha+\beta=\alpha+(\gamma+1)=(\alpha+\gamma)+1=(\alpha+\gamma)\cup\{\alpha+\gamma\}.$$
Iteracyjna definicja daje nam z kolei
\begin{align*}
    \alpha+\beta&=ot(\alpha\times\{0\}\cup\beta\times\{1\})=ot(\alpha\times\{0\}\cup(\gamma+1)\times\{1\}).
\end{align*}
Z założenia indukcyjnego wiem, że
$$\varphi+\gamma=ot(\alpha\times\{0\}\cup\gamma\times\{1\}),$$
czyli istnieje izomorfizm 
$$\varphi_\gamma:ot(\alpha\times\{0\}\cup\gamma\times\{1\})\to \alpha+\gamma.$$
Napiszmy izomorfizm między tymi dwoma zbiorami. 
$$\varphi_{\gamma+1}:ot(\alpha\times\{0\}\cup\gamma\times\{1\}\cup\{\gamma\}\times\{1\})\to(\alpha+\gamma)+1=(\alpha+\gamma)\cup\{\alpha+\gamma\}$$
$$\varphi_{\gamma+1}(\xi, n)=\begin{cases}\{\alpha+\gamma\}\quad (\xi, n) = (\gamma, 1)\\ \varphi_\gamma(\xi, n)\end{cases}$$
{\color{acc}krok graniczny}\smallskip\\
Rekurencyjna definicja dodawania liczb porządkowych daje mi:
$$\alpha+\beta=\bigcup\limits_{\xi<\beta}(\alpha+\xi)$$
Z normalnej definicji mamy
$$\alpha+\beta=ot(\alpha\times\{0\}\cup\beta\{1\}).$$
Z zalozenia indukcyjnego wiem, ze
$$(\forall\;\gamma<\beta)\;(\exists\;\varphi_\gamma:ot(\alpha\times\{0\}\cup\gamma\times\{1\})\to\alpha+\gamma)\;\varphi_\gamma\;-\;\texttt{izomorfizm}$$
Funkcja
$$\varphi:ot(\alpha\times\{0\}\cup\beta\times\{1\})\to\alpha+\beta=\bigcup\limits_{\gamma<\beta}\alpha+\gamma$$
$$\varphi(\xi, n)= \varphi_{\xi+1}(\xi, n)$$
jest izomorfizmem.\bigskip  

{\large\color{def}MNOŻENIE}:
\begin{align*}
    \alpha\cdot 0 &= 0\\
    \alpha\cdot(\beta+1)&=\alpha\cdot\beta+\alpha\\
    \alpha\cdot\gamma&=\bigcup\limits_{\xi<\gamma}(\alpha\cdot\xi)\quad Lim(\gamma)
\end{align*}

\subsection{HIERARCHIA $R_\alpha$}
\begin{center}\large
    {\color{def}HIERARCHIĘ} na liczbach porządkowych \\definiujemy rekurencyjnie:\smallskip\\
    $R_0=\emptyset$\smallskip\\
    $R_{\alpha+1}=\Po{R_\alpha}$\smallskip\\
    $R_\gamma=\bigcup\limits_{\xi<\gamma}R_\gamma\quad Lim(\gamma)$
\end{center}
{\large\color{acc}WŁASNOŚCI HIERARCHII}\medskip\\
\indent {\color{def}1.} $(\forall\;\alpha)\;Tran(r_\alpha)$\smallskip\\
Indukcja po $\alpha$\smallskip\\
Krok bazowy\smallskip\\
$\alpha= 0$ $Tran(\emptyset)$\medskip\\
\indent Krok indukcyjny\smallskip\\
\indent {\color{acc}krok następnikowy}\smallskip\\
Zakładamy, że $Tran(R_\alpha)$, pokażemy, że wówczas $Tran(R_{\alpha+1})$. Z definicji $R_{\alpha+1}=\Po R_\alpha$ \\mamy, że $Tran\Po{R_\alpha}$, więc $Tran(R_{\alpha+1})$. \medskip\\
\indent {\color{acc}krok graniczny}\smallskip\\
Zakładamy, że 
$$(\forall\;\gamma<\alpha)\;Tran(R_\gamma).$$
Ale skoro $Lim(\alpha)$, to
$$R_\alpha=\bigcup\limits_{\gamma<\alpha}R_\gamma,$$
a więc mamy sumę zbiorów tranzytywnych, więc $Tran(\bigcup\limits_{\gamma<\alpha}\gamma)$ i $Tran(R_\alpha)$\bigskip\\
\indent {\color{def}2.} $\alpha\leq \beta\implies R_\alpha\subseteq R_\beta$\bigskip\\
\indent{\color{def}3.} $R_\alpha\cup ON=\alpha$\bigskip\\


\subsection{DOMKNIĘCIE TRANZYTYWNE}
\begin{center}\large
    {\color{def}TRANZYTYWNE DOMKNIĘCIE ZBIORU} $x$ to najmniejszy zbiór tranzytywny zawierający zbiór $x$.\smallskip\\
    $tcl(x)=x\cup\bigcup x\cup\bigcup\bigcup x\cup \bigcup\bigcup\bigcup x\cup...$
\end{center}
{\color{acc}\large KONSTRUKCJA REKURSYWNA}
\begin{align*}
    a_0^x&=x\\
    a_{n+1}^x&=\bigcup a_n^x\\
    tcl(x)&=\bigcup\limits_{i\in\omega}a_i^x
\end{align*}
\podz{def}
\begin{center}\large
    Każdy zbiór jest w jakiejś hierarchii:\smallskip\\
    $(\forall\;x)(\exists\;\alpha)\;x\in R_\alpha$\medskip\\\normalsize
    lub równoważnie\smallskip\\
    $\bigcup\limits_{\alpha\in ON}R_\alpha=V$
\end{center}
\dowod
Przypuśćmy, nie wprost, że istnieje zbiór $x$ taki, że
$$(\forall\;\alpha)\;x\notin R_\alpha.$$
Rozważmy zbiór
$$Y=\{y\in tcl(x)\cup \{x\}\;:\;y\notin \bigcup\limits_{\alpha\in ON}R_\alpha\}\neq\emptyset.$$
Z aksjomatu regularności wiemy, że w $Y$ istnieje element $\in$-minimalny, czyli istnieje \\$y_0\in Y$ taki, że
$$(\forall\;t\in Y)\;t\notin y_0.$$
To znaczy, że dla każdego $z\in y_0$ mamy $z\notin Y$ (z minimalności $y_0$), czyli
$$z\notin tcl(x)\cup\{x\}.$$
W takim razie $z\in \bigcup\limits_{\alpha\in ON}R_\alpha$, a więc
$$(\forall\;z\in y_0)(\exists\;\alpha\in ON)\;z\in R_\alpha.$$
Mamy zatem funkcję
$$f:y_0\to ON$$
$$f(z)=min\{\alpha\;:\;z\in R_\alpha\}.$$
Na mocy aksjomatu zastępowania istnieje wtedy $rng(f)$. Niech
$$\beta=\bigcup rng(f)\in ON$$
Mamy 
$$(\forall\;z\in y_0)\;z\in R_\alpha$$
Wiemy, że suma liczb porządkowych jest majmniejszą liczbą porządkową większą od wszy-\\stkich elementów tej liczby.
$$f(z)\leq R_\beta,$$ co prowadzi nas do $y_0\in R_\beta$, co jest sprzeczne z $y_0\in Y$.