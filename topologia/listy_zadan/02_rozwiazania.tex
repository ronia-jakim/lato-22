\documentclass{article}

\usepackage{../../notatka}

\begin{document}\ttfamily
\subsection*{Zad 1.}
    Jesli zrobimy brutalny manewr i wezmiemy $A$, to bedzie jednoczesnie otwarte i domkniete \kotecek\medskip\\
    W takim razie wezmy $Y=[0, \infty)$
\subsection*{Zad 2. }
\subsection*{Zad 3.}
    {\Large\color{tit}a. Jesli $A$ otwarty w $Y$, to $A$ otwarty w $X$?}\medskip\\
    NIET.\\
    $(\R^2, d_{euklid})=X$\\
    $(\R_\alpha, d_{euklid})=Y$\\
    odcineczek w $Y$ \bigskip\\
    {\Large\color{tit}b.}\medskip\\
    TAK.\\
    $$U':=\{Y\cap U\;:\;U\in \mathcal{U}\}$$
    no to skoro $A$ jest otwarte w $X$ i ten przekroj jest otwarty w $Y$, to $A$ jest otwarte w $Y$?? bedzie robiona lista\\\kondow\bigskip
    {\Large\color{tit}c. jesli $A$ jest gesty w $Y$ i $Y$ jest gesty w $X$, to $A$ jest gesty w $X$?}
    $$B_\frac{r}2(x)\cap Y\neq\emptyset$$
    niech $y\in Y$
    $$B_\frac{r}2(y)\cap A\neq\emptyset$$
    czyli
    $$a\in B_\frac{r}2(y)$$
    chcemy pokazac, ze $a\in B_r(X)$\\
    Z nierownosci trojkata:
    $$d(a,x)< r$$
    czyli
    $$B_r(x)\cap A\neq \emptyset$$
    \kondow
    chociaz to nie przejdzie w topologii
\subsection*{Zad 4}
\subsection*{Zad 5.}
zaroweczka i cien na kuli?

\end{document}