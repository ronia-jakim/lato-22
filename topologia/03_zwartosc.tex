\section{ZWARTOŚĆ, SPÓJNOŚĆ}

\subsection{PRZESTRZEŃ ZWARTA}
\begin{center}\large
    {\color{def}POKRYCIE} - rodzina zbiorów otwartych \\sumująca się do $X$\smallskip\\
    $\bigcup \rodz U = X$\bigskip\\
    Przestrzeń topologiczna $X$ jest {\color{def}ZWARTA}, \\gdy z każdego pokrycia można wybrać podpokrycie skończone.
\end{center}\bigskip

$(0,1)$ w metryce euklidesowej nie jest zbiorem zwartym. Kontrprzykładem są coraz to \\mniejsze w średnicy przedziały otwarte:
\pmazidlo
    \draw[gray, thick] (-1, 0) -- (4, 0);
    \draw[gray, ultra thick] (0,0)--(3, 0);
    \filldraw[color=tit, fill=back, ultra thick] (1.5, 0) circle (0.06);
    \filldraw[color=tit, fill=back, ultra thick] (0,0) circle (0.06);
    \filldraw[color=acc, fill=back, ultra thick] (1.2, 0) circle (0.06);
    \filldraw[color=acc, fill=back, ultra thick] (2.5, 0) circle (0.06);
    \filldraw[color=gr, fill=back, ultra thick] (2.3, 0) circle (0.06);
    \filldraw[color=gr, fill=back, ultra thick] (2.8, 0) circle (0.06);
\kmazidlo
Ich prawe granice zbiegają do 1, więc wyrzucenie nawet jednego zbioru nie da nam po-\\krycia.\medskip\\
$[0,1]$ w metryce euklidesowej jest zbiorem zwartym. Jeśli znowu podzielimy na coraz to \\mniejsze przedziały, to zawsze zostaje ten malutki, który musi sie sumować do 1. Wys-\\tarczy że go wybierzemy, a resztę tych maleństw wyrzucimy i w ten sposób otrzymamy pod-\\pokrycie skończone.\bigskip\\
\podz{gr}\bigskip
\begin{center}\large
    Przestrzeń metryczna jest {\color{def}ZWARTA} wtedy \\i tylko wtedy, gdy z {\color{emp}każdego ciągu możemy wybrać podciąg zbieżny.}
\end{center}\bigskip
\dowod
Chcemy pokazać $\iff$.\bigskip\\
$\color{def}\implies$\medskip\\
Niech $(x_n)$ będzie dowolnym ciągiem w przestrzeni zwartej $X$. Wtedy może zajść jedna z \\dwóch możliwość:\medskip\\
\indent 1. Niech $A=\{a\;:\;(\exists\;k)\;a=x_k\}$, wtedy $(\exists\;x\in X)(\forall\;\varepsilon>0)\; B_\varepsilon(x)\cap A$ jest nieskończony, czyli nieskończenie wiele wyrazów zawiera się w kuli o dowolnym promieniu $\varepsilon>0$ i środku w $x$ (czli $x$ jest PUNKTEM SKUPIENIA CIĄGU).\smallskip\\
Wybieżmy rosnący ciąg $(n_k)$ taki, że
$$x_{n_k}\in B_\frac1k (x).$$
Ale taki ciąg musi spełniać $(x_{n_k})\to x$, czyli mamy podciąg zbieżny.\medskip\\
\indent 2. Załóżmy, że $(x_n)$ nie ma punktu skupienia.\smallskip\\
Weźmy dowolne $x\in X$. Wówczas istnieje $B_r(x)$, czyli kula o środku $x$, która zawiera skoń-\\czenie wiele wyrazów ciągu $(x_n)$. Rozważmy zbiór takich kul
$$\{B_r(x)\;:\;x\in X\}$$
że są one pokryciem $X$. Ponieważ $X$ jest zwarte, to istnieje takie $F$, że
$$B_r(x)\;:\;x\in F$$
jest skończonym pokryciem $X$. Ale wtedy ciąg tych $x$ jest ciągiem skończonym i mamy \\sprzeczność ({\color{cyan}?}).\bigskip\\
$\color{def}\impliedby$\medskip\\
{\large\color{cyan}ZROBIĆ DOWOD}

\podz{gr}\bigskip
\begin{center}\large
    $(X, d)$ jest przestrzenią metryczną, $X\subseteq Y$\smallskip\\
    {\color{def}Jeżeli $X$ jest zwarta, to}\medskip\\
    1. $X$ jest ograniczona\smallskip\\
    2. $X$ jest domknięty w $Y$.
\end{center}\bigskip
Jeśli mamy metrykę euklidesową i $\R^n$, to implikacja zamienia się w równoważność, tzn $X\subseteq \R^n$ jest zwarty $\iff$ jest domknięty i ograniczony.\bigskip\\
\dowod
\indent 1. Wiemy, że przestrzeń $X$ jest ograniczony wtw gdy jej średnica jest skończona:
$$X\;\texttt{jest ograniczona}\iff diam(X)<\infty$$
$$diam(X) = \sup\{d(x,y)\;:\;x,y\in X\}.$$
Załóżmy, że $X$ jest nieograniczona. Wskażemy wówczas ciąg, który nie ma podciągu zbież-\\nego.
\begin{align*}
    x_0&\in X\\
    x_1&\quad d(x_0,x_1)>1\\
    x_2&\quad d(x_0, x_2)>1\;\land\;d(x_1, x_2)>1\\
    ...&
\end{align*}
$(x_n)$ nie ma podciągu zbieżnego, bo wszystkie jego elmenty są odległe od siebie o więcej niż 1.\bigskip\\
\indent2. Załóżmy, że $X$ nie jest domknięty, czyli istnieje ciąg, który jest zbieżny w $Y$, \\ale nie jest zbieżny w $X$. 
$$(x_n)\quad ((\forall\;n)\;x_n\in X)\;\land\;((\exists\;y\in Y)\;(x_n)\to y)$$
Ale to jest sprzeczne z warunkiem zwartości, bo każdy podciąg $(x_n)$ jest zbieżny do tego samego $y\in Y$, więc żaden nie jest zbieżny w $X$.
\kondow\bigskip
\podz{gr}\bigskip

\begin{center}\large
    Jeśli istnieje $f:X\xrightarrow[ciagla]{na} Y$, to zachodzi\smallskip\\
    $X\;\texttt{zwarta}\implies Y\;\texttt{zwarta}$\bigskip

    \emph{\normalsize{\color{def}Zbieżność jest przechodnia przez funckje ciągłe i na}.}
\end{center}
\dowod
Na $Y$ wyróżniamy pewne pokrycie i chcemy pokazać, że możemy wybrać z niego podpokrycie \\skończone.\medskip\\
Rozważmy rodzinę $\rodz U$ otwartych zbiorów w $Y$ taką, że
$$\bigcup\rodz U=Y$$
oraz przeciwobrazy zbiorów tej rodzny:
$$\rodz B=\{f^{-1}[U]\;U\in\rodz U\}.$$
Ponieważ $f$ jest funkcją ciągłą, to $f^{-1}[U]$ są zbiorami otwartymi. Łatwo zauważyć, że
$$\bigcup \rodz B=X.$$
Ponieważ $X$ jest przestrzenią zwartą, możemy wybrać skończone podpokrycie z $\rodz B$: 
$$\rodz B_1=\{f^{-1}[U]\;:\;\rodz U_1\}.$$
Ale funkcja $f$ jest na, więc
$$\bigcup\rodz U_1=Y,$$
a z ciągłości wiemy, że każdy zbiór w $\rodz U_1$ jest zbiorem otwartym. W takim razie dowolne pokrycie $Y$ ma podpokrycie skończone.
\kondow
{\large\color{acc}WNIOSKI:}\medskip\\
\indent 1. $X\cong Y$ i $X$ jest zwarta, to $Y$ też musi być zwarta\smallskip\\
\indent 2. funkcja ciągła $f:[a,b]\to \R$ na przedziale domkniętym jest ograniczone i przyjmuje swoje kresy.\bigskip\\
\podz{gr}\bigskip

\begin{center}\large
    Zwartość przenosi się na podzbiory domknięte
\end{center}\bigskip
\dowod
Weźmy dowolne $\rodz U$ pokrycie $X\subseteq Y$. Żeby dostać pokrycie $Y$, wystarczy do niego dodać $X^c$, które jest otwarte, bo $X$ jest zbiorem domkniętym. Czyli otrzymujemy pokrycie zwartego zbioru $Y$:
$$\rodz U\cup X^c.$$
Możemy z niego wybrać podpokrycie skończone, bo $Y$ jest przestrzenią zwartą. Dostajemy
$$\rodz U_1\cup X^c,$$
ale $\rodz U_1$ jest skończonym podzbiorem $\rodz U$ i pokrywa $X$, dostajemy więc skończone podpokrycie \\$X$, więc jest on zwarty.
\kondow

\begin{center}\large
    Jeśli $X$ jest przestrzenią zwartą \\oraz $X\subseteq Y$, $Y$ jest przestrzenią Hausdorffa, \\to wtedy $X$ jest domknięty w $Y$.
\end{center}\bigskip
\dowod
Chcemy pokazać, że $Y\setminus X$ jest zbiorem otwartym.\medskip\\
Weźmy dosolny $y\in Y\setminus X$. Chcemy znaleźć zbiór otwarty, który oddzieli nas od $X$. Weźmy dowolne $x\in X$.\medskip\\
Z przestrzeni Hausdorffa wiemy, że istnieje $U_x\ni x$ oraz $V_x\ni y$, które są rozłączne
$$U_x\cap V_x=\emptyset.$$
Dla każdego punktu $x\in X$ możemy wybrać taki zbiór, więc rodzina
$$\rodz A=\{U_x\cap X\;:\;x\in X\}$$
jest pokryciem $X$. Ze zwartości $X$ wiemy, że istnieje skończony podzbiór $X_1\subseteq X$ takie, że
$$\rodz A_1=\{U_x\cap X\;:\;x\in X_1\}$$
jest skończonym pokryciem $X$. Weźmy przekrój
$$\bigcap\limits_{x\in X_1}V_x.$$
Jest on zbiorem otwartym, bo jest przekrojem skończenie wielu zbiorów otwartych. \\Z własności przestrzeni Hausdorffa wiemy, że
$$\bigcap\limits_{x\in X_0}V_x\cap U_x=\emptyset,$$
a ponieważ $U_x$ należało do pokrycia $X$, to również
$$\bigcap\limits_{x\in X_0}V_x\cap X=\emptyset.$$
Wobec dowolności $y\in Y\setminus X$ mamy $Y\setminus X$ jest zbiorem otwartym, więc $X$ jest domknięte w $Y$.
\kondow

\begin{center}\large
    Jeżeli $f:X\xrightarrow{ciagla} Y$ jest bijekcją, to jeśli $X$ \\jest przestrzenią zwartą, {\color{def}$f$ jest homeomorfizmem}
\end{center}
\dowod
Wystarczy pokazać, że $f^{-1}$ jest funkcją ciągłą, tzn $f[D]$ jest domknięty dla każdego \\$D\underset{dom}\subseteq X$. \medskip\\
Ponieważ $D$ jest domkniętym podzbiorem zwartej przestrzeni, to $D$ również jest zwarte. \\W takim razie $f[D]$ jest zwartym podzbiorem $Y$, a więc jest jego domkniętym podzbiorem.
\kondow
\begin{center}\large
    Jeśli $X$ jest zwartą przestrzenią metryczną, \\to $X$ jest całkowicie ograniczone.
\end{center}
{\large\color{cyan}COO?}