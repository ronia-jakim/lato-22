\documentclass{article}

\usepackage{../../notatka}
\usepackage{tikz}
\usetikzlibrary{fadings}

\newcommand\fadingtext[3][]{%
  \begin{tikzfadingfrompicture}[name=fading letter]
    \node[text=transparent!0,inner xsep=0pt,outer xsep=0pt,#1] {#3};
  \end{tikzfadingfrompicture}%
  \begin{tikzpicture}[baseline=(textnode.base)]
    \node[inner sep=0pt,outer sep=0pt,#1](textnode){\phantom{#3}}; 
    \shade[path fading=fading letter,#2,fit fading=false]
    (textnode.south west) rectangle (textnode.north east);% 
  \end{tikzpicture}% 
}

\begin{document}\ttfamily
    {\color{emp}\large STRZALKA} (Sorgenfrey line), przyklad w $\R$\smallskip\\
    $$B=\{[a,b)\;:\;a<b\}$$ 
    sa zbiorami otwartymi (otwarto-domknietymi, tak jak $\R$ czy $\emptyset$ w $\R$).
    \pmazidlo
    \draw[gray] (0, 0)--(5, 0);
    \node at (4.8, 0.3) {$\R$};
    \node at (1.5, -0.4) {0};
    \node at (3, -0.4) {1};
    \draw[def, ultra thick] (1.5, 0)--(3, 0);
    \filldraw[color=def, fill=def, thick](1.5, 0) circle (0.1);
    \filldraw[color=def, fill=back, thick](3, 0) circle (0.1);
    \kmazidlo
    \begin{center}\large
        {\color{def}BAZA} dla topologii to taka \emph{\color{emp}rodzina zbiorow otwartych}, \\ze kazdy niepusty i otwarty podzbior tej przestrzeni \\mozna wysumowac przy pomocy pewnych elementow bazy.
    \end{center}\bigskip
    Topologia strzalki jest bogatsza (silniejsza, wieksza) niz topologia euklidesowa - kazdy otwarty zbior w sensie euklidesowym jest tez otwarty w sensie strzalki\medskip\\
    Strzalka jest przestrzenia {\color{acc}Handsdorffa}\medskip\\
    Jak wygladaja ciagi zbiezne w strzalce? 
    $$\left({1\over n}\right)_{n\in\N}\to 0$$
    $$\left({a\over n}\right)\texttt{ nie jest zbiezny, bo wszystkiw wyrazy sa poza przedzialem}$$
    \emph{nie jest to przestrzen metryzowalna}\bigskip\\
    \podz{gr}\bigskip\\
    {\color{emp}\large UZWARCENIE ALEKSANDROWA} (aka przestrzen z gruszka) na $\R$, ale moze byc to dowolna przestrzen\medskip
    \begin{flushright}
        {\color{acc}przestrzen zwarta \color{dygresyja}- przestrzen topologiczna, \\ze z dowolnego jej pokrycia zbiorami mozna wybrac \\podpokrycie skonczone}\smallskip\\
        {\color{acc}uzwarcenie \color{dygresyja}- rozszerzenie \\danej przestrzeni topologicznej tak, \\by byla ona przestrzenia zwarta}\\
        \color{acc}otoczenie \color{dygresyja}- dowolny zbior, \\ktory zawiera zbior otwarty \\zawierajacy dany punkt
    \end{flushright}\medskip
    \pmazidlo
    \draw[white] (0,0)--(5,0);
    \node at (3, 0.5) {\color{acc}\kotecek};
    \kmazidlo
    Mamy $\R$ i mamy jakiegos {\color{acc}\kotecek}. Otoczenia wszystkich liczb $\R$ to
    $$r:\{r\},$$ 
    czyli signletony liczb rzeczywitych sa otwarte. Otoczeniem {\color{acc}\kotecek} sa 
    $${\color{acc}\kotecek}:\{{\color{acc}\kotecek}\}\cup A,$$ 
    takie, ze $A\subseteq \R$ i $ \R\setminus A$ jest skonczony.\medskip\\
    Topologie mozemy w uzwarceniu Aleeksandrowa zdefiniowac w dowolny sposob, musi tylko jasno wynikac, co jest zbiorem otwartym, a co zamknietym.\medskip\\
    Uzwarcenie Aleksandrowa jest przestrzenia {\color{acc}Hansdorffa}\smallskip\\
    Jak wygladaja ciagi zbiezne?
    $$\left({1\over n}\right)\to \kotecek,$$
    bo tylko skonczenie wiele punktow moze byc zignorowanych przez otoczenie \kotecek. W takim razie mozemy powiedziec, ze jesli mamy dowolny $(x_n)$ roznowartosciowy, to
    $$\lim x_n=\kotecek$$
    bo $\kotecek\underset{otw}\in U$ i istnieje skonczenie wiele $n$ takich, ze $x_n\notin U$.\bigskip\\
    \podz{def}\bigskip\\

\subsection*{PRZESTRZEN OSRODKOWA}\bigskip
    Ciag zbiezny - byl definiowany jako ciag, ktorego wszystkie elementy leza w kuli o coraz to mniejszym promieniu\\
    $$\texttt{Int}A= \{x\in A\;:\;\exists\;x\underset{otw}\in U\quad U\subseteq A\}$$
    natomiast zbiorem domknietym byly dopelnienia otwartych:
    $$\overline{A}=\{x\in X\;:\;\forall\;x\underset{otw}\in U\quad U\cap A\neq \emptyset\}.\bigskip$$
    \podz{tit}\bigskip
    \begin{center}\large
        $X$ - przestrzen topologiczna\medskip\\
        Zbior $A\subseteq X$ jest {\color{def}GESTY} (dense), jezeli\smallskip\\
        $\forall\; U\underset{otw}\neq\emptyset\quad U\cap A\neq \emptyset\iff \overline{A}=X$\medskip\normalsize\\
        \emph{jest to zbior otwarty, ktory \\{\color{acc}kroi sie niepusto z kazdym zbiorem otwartym} \\(lub dopelnia sie do calej przestrzeni)}\bigskip\large\\
        Przestrzen $X$ jest {\color{def}OSRODKOWA}, \\jesli istnieje w niej \emph{\color{emp}przeliczalny zbior gesty}
    \end{center}\bigskip
    \podz{gr}\bigskip\\
    {\large\color{emp}PRZYKLADY} - OSRODKOWA\medskip\\
    \indent$\R$ z metryka euklidesowa - osrodkowy (separable) bo \indent$\Q\subseteq \R$\medskip\\
    \indent$\R^2$ z metryka euklidesowa: $\Q\times \Q$ jest gesty\medskip\\
    \indent$\R^2$ z metryka miasto: $\Q^2$ bo zbiory otwarte w miescie sa takie same jak w euklidesie\medskip\\
    \indent kostka Cantora ($\{0,1\}^\N$) - bierzemy wszystkie skonczone ciagi stale od pewnego miesjca (czyli skonczone, ale sztucznie przedluzone do nieskonczonosci) - jest ich przeliczalnie wiele, a ich zbior jest gesty. Wezmy kule $B_r(x)$ o promieniu $r>\frac1{2^n}$
    $$y(i)=x(i)\quad i\leq n+1$$
    $$y(i)=0 \quad i>n+1\bigskip$$
    {\large\color{emp}ANTYPRZYKLADY}: \medskip\\
    \indent$(\R, d_{dyskretna})$. Zbior gesty $A\subseteq \R$ musi kroic sie niepusto z kazdym singletonem, wiec
    $$\forall\;x\quad A\cap \{x\}\neq\emptyset\iff A=\R$$
    czyli zbior gesty nie jest przeliczalny.\medskip\\
    \indent$(\R^2, d_{centrum})$. Intuicja podpowiada, ze $\Q\times\Q$ byloby geste i wtedy to bylby przeliczalny. Jednak, jesli kula lezy na prostej o wyrazach niewymiernych, na przyklad
    $$y=\pi x,$$
    to tnie sie pusto ze zbiorem $\Q\times\Q$\medskip\\
    \podz{gr}\medskip
    \begin{center}
        {\large\color{def}FAKT}: W przestrzeni metrycznej $\langle X,d\rangle$ {\color{acc}zbior $A\subseteq X$ jest gesty}, \\wtedy i tylko wtedy, gdy {\color{acc}dla kazdej kuli $B_r(x)$ istnieje $a\in A$ blizej $x$ niz kula}:\smallskip\\
        $A\texttt{ zb. gesty}\iff\forall\;x\in X\;\forall\;\varepsilon>0\;\exists\;a\in A\quad d(x,a)<\varepsilon$
    \end{center}
    \dowod
    $\color{acc}\implies$\smallskip\\
    Zalozmy, ze twierdzenie jest nieprawdziwe, czyli dla zbioru gestego $A$ i przestrzeni metrycznej $\langle X,d\rangle$ istnieje kula o promieniu $\varepsilon$ i srodku $x\in X$ taka, ze nie zawiera elementow z $A$:
    $$\exists\;x\quad B_\varepsilon(x)\cap A=\emptyset$$
    W takim razie $A$ tnie sie pusto ze zbiorem otwartym $B_\varepsilon(x)$, czyli nie jest zbiorem gestym.\medskip\\
    $\color{acc}\impliedby$\smallskip\\
    Wezmy jakis zbior otwarty 
    $$U\underset{otw}\subseteq X,$$ 
    czyli mozemy zalozyc, ze istnieje kula:
    $$\exists\;B_r(x)\subseteq U.$$
    Czyli kula $B_r(x)$ zawiera sie otwartym zbiorze $U$, wiec istnieje w $U$ punkt ktory lezy w tej kuli:
    $$\exists\;u\in U\quad d(x,u)<r,$$
    a wiec kula tnie sie niepusto ze zbiorem $U$:
    $$U\cap B_r(x)\neq\emptyset$$
    \kondow
    {\color{emp}NA CO TO BYL PRZYKLAD?}
    POWROT DO METRYKI CENTRUM\\
    Rozwazmy okrag i robimy kule promieniscie i jest ich $\cont$ wiele
    $$S^1=\{x\;:\;d(x, \langle0,0\rangle =1)\}$$
    \pmazidlo
        \draw[gray] (2, 0)--(2, 4);
        \draw[gray](0, 2)--(4, 2);
        \draw[emp] (2, 2) circle (1);
        \draw [def] (3, 3)--(2.5,2.5);
    \kmazidlo
    Przestrzen supremum jest osrodkowa, bo wielomiany tworza ciag gesty.\bigskip\\
    \podz{def}\bigskip
    \begin{center}\large
        Jesli istnieje $\color{def}f:X\to Y$ ktora jest ciagla i na, to jezeli \\{\color{def}$X$ jest przestrzenia osrodkowa, to $Y$ tez} jest przestrzenia osrodkowa\smallskip\\
        \emph{\normalsize\color{acc}osrodkowoosc przenosi sie przez ciagle suriekcje}
    \end{center}
\dowod
Celem dowodu jest zdefiniowanie przeliczalnego zbioru gestego w $Y$.\smallskip\\
Niech $A\subseteq$ bedzie przeliczalnym zbiorem gestym w $X$. Wtedy zbiorem gestym w $Y$ bedzie obraz $A$ przez funkcje $f$
$$B=f[A].$$
Poniewaz $B$ jest obrazem zbioru przeliczalnego przez ciagla suriekcje, to jest on zbiorem przeliczalnym. Pozostaje udowdnic, ze $B$ jest zbiorem gestym.\smallskip\\
Wezmy dowolny zbior otwarty w $Y$: $U\underset{otw}\subseteq Y$. Wtedy $f^{-1}[U]\subseteq X$, poniewaz $f$ jest funkcja "na".\medskip\\

No to w takim przypadku zbiorem gestym w $Y$ bedzie $f[A]$. Jest to zbior przeliczalny, bo jest obrazem zbioru przeliczalnego, a czy jest gesty?\\
Bierzemy dowolny zbior otwarty w $U\subseteq Y$, to wtenczas $f^{-1}[U]\subseteq X$
$$\exists\; a\in A\quad a\in f^{-1}[U]\quad f(a)\in U\cap f[A]\neq\emptyset$$
\subsection*{ZBIOR CANTORA <3}
$$C\subseteq[0,1]$$
C jest przekrojem zbiorow domknietych, wiec sam tez jest zbioreom dokmnietym.\bigskip\\
ZBIOR CANTORA  jest homeomorficzny z kostka Cantora
$$Cant\underset{home}\simeq {0,1}$$
DOWODZIK:
\pmazidlo
    \node (a) at (5, 5) {};
    \node (b) at (4, 4) {};
    \node (b2) at (6, 4) {};
    \draw[white, thick] (a)--(b);
    \draw[white, thick] (a)--(b2);
    \node (c) at (3.2, 3) {};
    \node (c2) at (4.8, 3) {};
    \draw[white, thick] (b)--(c);
    \draw[white, thick] (b)--(c2);
    \node (d) at (5.2, 3) {};
    \node (d2) at (6.8, 3) {};
    \draw[white, thick] (b2)--(d2);
    \draw[white, thick] (b2)--(d);
\kmazidlo
$$f:\{0,1\}^\N\to Cant$$
$s$ - skonczony ciag 0,1. Wowczas $C_s$ to jest ciag, ktory w zbiorze Cantora pokolei przyjmuje lewy lub prawy podbior poprzedniego zbioru (skaczemy lew-prawa)
$$f(x)=y \quad\bigcap \limits_{s-odc\; pocz\; x}D_s = \{y\}$$
Co nas czeka: \\
\indent zobaczenie ze to $D_s$ jest niepuste\\
\indent ze to jest 1-1 i na\\
1-1 bo mamy dwa rozne ciagi, to one sie nam rozjeda i nie ma opcji zeby sie znowu pozniej spotkaly\\
bo zawsze dojdziemy d odowolnego $x$\\
\indent dowod ciaglosci i ciaglosci $f^{-1}$\\
\kondow
\setlength\fboxsep{0pt}

\fadingtext{left color = emp, right color = def}{teczowy homeomorfizm $\simeq$} 
\end{document}