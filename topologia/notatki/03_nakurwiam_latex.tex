\documentclass{article}

\usepackage{../../notatka}

\begin{document}\ttfamily
\tikz\duck;\\
    PRZYKLAD\medskip\\
    Strzalka (Sorgenfrey line), przyklad w $\R$\smallskip\\
    Baza: $\{[a,b)\;:\;a<b\}$ staja sie zbiorami otwartymi
    \pmazidlo
    \draw[gray] (0, 0)--(5, 0);
    \node at (4.8, 0.3) {$\R$};
    \draw[def, ultra thick] (1.5, 0)--(3, 0);
    \filldraw[color=def, fill=def, thick](1.5, 0) circle (0.1);
    \filldraw[color=def, fill=back, thick](3, 0) circle (0.1);
    \kmazidlo
    Baza dla topologii to taka rodzina, ze kazda ??? jest suma zbiorow otwartych?\bigskip\\
    topologia strzalki jest bogatsz niz topologia euklidesowa - kazdy otwarty zbior w sensie euklidesowym jest tez otwarty w sensie strzalki\medskip\\
    strzalka jest handsdorffa\medskip\\
    Jak wygladaja ciagi zbiezne w strzalce? 
    $$\left({1\over n}\right)_{n\in\N}\to 0$$
    $$\left({a\over n}\right)\texttt{ nie jest zbiezny, bo wszystkiw wyrazy sa poza przedzialem}$$
    nie jest to przestrzen metryzowalna\bigskip\\
    UZWARCENIE ALEKSANDROWA (aka przestrzen z gruszka)\medskip\\
    znowu przestrzen to $\R$, ale moze byc dowolne
    \pmazidlo
    \draw[gray] (0,0)--(5,0);
    \node at (3, 0.5) {\color{acc}\kotecek};
    \kmazidlo
    Mamy $\R$ i mamy jakiegos kota. Otoczenia $r:\{r\}$ - signletony liczb rzeczywitych sa otwarte (no to wszystko jest otwarte). Otoczeniem {\color{acc}\kotecek} sa $\kotecek:\{\kotecek\}\cup A$, takie, ze $A\subseteq \R$ i $ \R\setminus A$ jest skonczony\medskip\\
    Topologie definujemy jak nam sie podoba, tylko musi jasno wynikac, co jest otwarte, a co jest zamkniete.\medskip\\
    Jest to przestrzen Hansdorffa\smallskip\\
    Jak wygladaja ciagi zbiezne?
    $$\left({1\over n}\right)\to \kotecek$$
    po tylko skonczenie wiele punktow moze byc zignorowanych przez otoczenie \kotecek\smallskip\\
    czyli ogolem, jesli mamy dowolny $(x_n)$ roznowartosciowy, to
    $$\lim x_n=\kotecek$$
    bo $\kotecek\underset{otw}\in U$ bo istnieje skonczenie wiele $n$ takich, ze $x_n\notin U$
\subsection*{COS}
    Ciag zbiezny - byl definiowany\\
    $\texttt{Int}A= \{x\in A\;:\;\exists\;x\underset{otw}\in U\quad U\subseteq A\}$\\
    $\overline{A}=\{x\in X\;:\;\forall\;x\underset{otw}\in U\quad U\cap A\neq \emptyset\}$\\
    zbiory domkniete = dopelnienia otwartych
    \begin{center}
        $X$ - przestrzen topologiczna\medskip\\
        $A\subseteq X$ jest GESTY, jezeli\smallskip\\
        $\forall\; U\underset{otw}\neq\emptyset\quad U\cap A\neq \emptyset\iff \overline{A}=X$\smallskip\\
        czyli zb otwarty, ktory kroi sie niepusto z kazdym zbiorem otwartym (lub dopelnia sie do calej przestrzeni)\bigskip\\
        Przestrzen $X$ jest OSRODKOWA, jesli istnieje w niej przeliczalny zbior gesty
    \end{center}\bigskip
    PRZYKLADY - OSRODKOWA\medskip\\
    $\R$ z metryka euklidesowa - osrodkowy bo $\Q\subseteq \R$\medskip\\
    $\R^2$ z metryka euklidesowa: $\Q\times \Q$ jest gesty\medskip\\
    $\R^2$ z metryka miasto: $\Q^2$ bo zbiory otwarte w miescie sa takie same jak w euklidesie\medskip\\
    kostka Cantora ($\{0,1\}^\N$) - bierzemy wszystkie skonczone ciagi stale od pewnego miesjca (czyli skonczone, ale sztucznie przedluzone do nieskonczonosci) - jest ich przeliczalnie wiele i to jest geste\\
    Wezmy kule $B_r(x)$ o promieniu $r>\frac1{2^n}$
    $$y(i)=x(i)\quad i\leq n+1$$
    $$y(i)=0 \quad i>n+1$$
    ANTYPRZYKLAD: $(\R, d_{dysk})$. Zbior gesty $A\subseteq \R$ musi kroic sie nipusto z kazdym singletoenm, wiec
    $$\forall\;x\quad A\cap \{x\}\neq\emptyset\iff A=\R$$
    BARDZIEJ SUBTELLNY ANTYPRZYKLAD: $(\R^2, d_{centrum})$. Intuicja podpowiada, ze $\Q\times\Q$ byloby geste i wtedy to bylby przeliczalny, ale kula ktora lezy na prostej $y=\pi x$ wymyka sie temu zbiorowi.\\
    FAKT: Jesli mamy przestrzen metryczna, to gestosc mozemy opisac $A\subseteq X$ jest gest, jesli dla kazdej kuli itnieje cos z tego zbioru blizej x niz kula
    $$\forall\;x\in X\;\forall\;\varepsilon>0\;\exists\;a\in A\quad d(x,a)<\varepsilon$$
    dowodzi <3\\
    $\implies$: zalozmy, ze $\exists\;x\quad\exists\;\varepsilon$ ze jest zle, czyli
    $$\exists\;x\quad B_\varepsilon(x)\cap A=\emptyset$$
    czyli nie mozemy byc gesci\\
    $\impliedby$ : wezmy jakis zbior otwraty $U\underset{otw}\subseteq X$, czyli mozemy zalozyc, ze jets taka kula:
    $$\exists\;B_r(x)\subseteq U$$
    i wowczasj z wlasnosci z faktu
    $$\exists\;a\in A\quad d(x,a)<r$$
    $$A\cap B_r(x,a)=\emptyset$$
    POWROT DO METRYKI CENTRUM\\
    Rozwazmy okrag i robimy kule promieniscie i jest ich $\cont$ wiele
    $$S^1=\{x\;:\;d(x, \langle0,0\rangle =1)\}$$
    \pmazidlo
        \draw[gray] (2, 0)--(2, 4);
        \draw[gray](0, 2)--(4, 2);
        \draw[emp] (2, 2) circle (1);
        \draw [def] (3, 3)--(2.5,2.5);
    \kmazidlo
    Przestrzen supremum jest osrodkowa, bo wielomiany tworza ciag gesty.
\end{document}