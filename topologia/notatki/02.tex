\documentclass{article}

\usepackage{../../notatka}

\begin{document}\ttfamily
\section*{PODPRZESTRZENIE METRYCZNEEEE}
    \begin{center}\large
        \color{def}PODPRZESTRZEN \color{txt}$(X, d)$ to $(A,d)$, $A\subseteq X$\smallskip\\
        \emph{formalnie $(A, d)$ nie jest metryka - musimy obciac $d_{\obet A\times A}$}
    \end{center}\bigskip
    \color{emp}\large PRZYKLAD\normalsize\color{txt}\medskip\\
    Mamy prosta $\R$ z metryka euklidesowa. Rozwazmy na niej \color{def}zbior $[0,1]$\color{txt}. Jesli zastanowimy sie nad kulami w tej podprzestrzeni, to mozemy otrzymac \color{emp}kule\color{txt}:
    \begin{center}\begin{tikzpicture}
        \draw[gray, thick] (0, 0)--(5, 0);
        \draw[def, ultra thick] (1, 0)--(4, 0);
        \filldraw[def] (1, 0) circle (0.07);
        \filldraw[def] (4, 0) circle (0.07);
        \filldraw[emp] (1, 0.5) circle (0.07);
        \draw[emp, very thick] (2.03, 0.5) circle (0.07);
        \draw[emp, ultra thick] (2, 0.5)--(1, 0.5);
        \node at (1.3, 0.5) {$\color{emp}\bullet$};
        \node at (1, -0.3) {0};
        \node at (4, -0.3) {1};
    \end{tikzpicture}\end{center}
    I ta kula jest otwarta, bo dla tej podprzestrzeni nie istnieja punkty mniejsze od 0.\bigskip\\
    Na $\R^2$ z metryka centrum wybieramy okrag o promieniu $\frac12$ i srodku w $(0,0)$. Taka podprzestrzen jest bardzo podobna do przestrzeni dyksretnej - kazde dwa punkty, ktore nie sa tym samym punktem, sa od siebie odlegle o 1. 
    \pmazidlo
        \draw[gray, thick] (2, 0)--(2, 4);
        \draw[gray, thick] (0, 2)--(4,2);
        \draw[acc, very thick] (2, 2) circle (1.2);
    \kmazidlo\bigskip\podz{gr}\bigskip\\
    Dwie przestrzenie metryczne: $(X, d)$ i $(Y,\rho)$, i funkcja z jednej w druga:
            $$f:X\to Y$$
    jest ciagla jesli (\color{def}warunek Cauchyego\color{txt}):
        $$\forall\;x\in X\;\forall\;\varepsilon>0\;\exists\;\delta>0\;\forall\;y\quad d(x,y)<\delta\implies \rho(f(x), f(y))<\varepsilon.\bigskip$$
    \begin{center}\large
        Jesli mamy $(X,d)$, $(Y,\rho)$ oraz funkcje $f:X\to Y$, wowczas\smallskip\\
        \color{def}1. \color{txt}$f$ jest funkcja ciagla\smallskip\\
        \color{def}2. \color{txt}(\color{acc}zbieznosc wg. Heinego\color{txt}): mamy $(x_n)$ - ciag z $X$, \\taki, ze $\lim x_n=x$, to $\lim f(x_n)=f(x)$ \\
        (\emph{ciag wartosci zbiega do wartosci granicy})\smallskip\\
        \color{def}3. \color{txt}$f^{-1}[U]$ jest otwarty, dla kazdego otwartego $U\subseteq Y$
    \end{center}\bigskip

    Pokazemy implikacje : $\texttt{3.}\implies\texttt{1.}$.\medskip\\
    Mamy funkcje $f:X\to Y$, i mamy sprawdzic, czy jest ciagla w sensie Cauchyego (z 1., warunek ciaglosci wyzej). Dla dowolnego $x\in X,\varepsilon>0$ mamy dobrac $\delta$ tak, zeby warunek ciaglosci byl spelniony, majac do dyspozycji tylko to, ze przeciwobrazy zbiorow otwartych sa otwarte.\smallskip\\
    Czyli chce pokazac, ze jesli bedziemy brali cos z kuli o promieniu $\delta$, to bedzie do tego nalezec wszystko w kuli o promieniu $\varepsilon$ i do tego chce korzystac z otwartosci przeciwobrazow zbiorow otwartych.\smallskip\\
    Wartosci musze byc w kuli o srodku w $f(x)$ i promieniu $\varepsilon$:
        $$U=B_\varepsilon(f(x)).$$
    Z zalozenia 3. jesli wezmiemy dowolny punkt $u$ ze zbioru $f^{-1}[U]$, to on siedzi w tym zbiorze wraz z pewna kula. Wybierzmy $u=x\in f^{-1}[U]$, bo $f(x)\in U$. Z definicji zbioru otwartego:
        $$\exists\;\delta>0\quad B_\delta{(x)}\subseteq f^{-1}[U].$$
    Jesli weze dowolne $y$ z kuli $B_\delta$, to jak naloze $y$ $f$, bedzie oon blizej $x$ niz $\varepsilon$, czyli $f(y)\in B_\epsilon(f(y))$
\subsection*{HOMEOMORFIZMY}
    \begin{center}\large
        \color{def}HOMEOMORFIZM \color{txt}($X\underset{hom}{\cong}Y$) nazywamy taka \\funkcje $f:(X,d)\to(y,\rho)$, ktora:\medskip\\
        $f$ jest 1-1 i na i ciagla oraz \smallskip\\
        $f^{-1}$ jest ciagla.\medskip\\
        \emph{$X$ jest homeomorfizmem z $Y$, jesli istnieje homomorfizm}
    \end{center}\bigskip
    \large \color{emp}PRZYKLADY\color{txt}\normalsize\bigskip\\
    $[0,1]\cong [0,2]$, wezmy funkcje $f(x)=2$ - jest ciagla bijekcja i funkcja odwrotna jest ciagla (najprostszy przyklad)\medskip\\
    $(\R^2, d_{euk})\cong(\R^2, d_{miast})$ dla funkcji $f(x,y)=\langle x,y\rangle$, czyli dla identycznosci 34:05
\end{document}