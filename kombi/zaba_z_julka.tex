\documentclass{article}

\usepackage{../notatka}
\usepackage[utf8]{inputenc}
\usepackage[T1]{fontenc}
\usepackage[polish]{babel}
\usepackage{array}
\usepackage{microtype}
\usepackage{makecell}
%\usepackage{showframe}
%\usepackage[nomathsymbols, OT4]{polski}
\selectlanguage{polish}

%\renewcommand*\ShowFrameColor{\color{gr}}

\title{\ttfamily {\color{emp}ŻABA}\medskip\\ \normalsize {\color{dygresyja}rozwiązanie robocze}}
\author{{\color{emp}Julka Walczuk} i W. Jakimowicz}
\date{}

\begin{document}\ttfamily
\maketitle\bigskip

Rozważamy, na ile sposobów żaba może przeskoczyć na przeciwny wierzchołek ośmiokąta \\foremnego w $2n$ krokach. \smallskip\\
Najpierw, pogrupujemy kroki w pary - wtedy żaba będzie ruszać się między $4$ wierzchoł-\\kami, nazwijmy je: $N,\; E,\; S,\; W$. Niech $N$ będzie naszym werzchołkiem startowym, a $S$ - \\końcowym.\medskip\\
Możliwe ruchy żaby to
$$(1, 1)\quad (-1, -1)\quad (1, -1)\quad (-1, 1),$$
gdzie $-1$ oznacza skok w prawo, a $1$ - w lewo (ale z kierunkami czasami są kontrowersje, \\więc zostają liczby).

\pmazidlo
\draw[white, thick] (0,0) circle (3);
\draw[dygresyja, ultra thick, ->] (0, 2.6) arc (90:-65:2.6);
\filldraw[color=emp, fill=back, ultra thick] (0, 3) circle (0.6);
\filldraw[color=emp, fill=back, ultra thick] (0, -3) circle (0.6);
\filldraw[color=emp, fill=back, ultra thick] (-3, 0) circle (0.6);
\filldraw[color=emp, fill=back, ultra thick] (3, 0) circle (0.6);
\filldraw[color=def, fill=back, ultra thick] (2.12, 2.12) circle (0.2);
\filldraw[color=def, fill=back, ultra thick] (2.12, -2.12) circle (0.2);
\filldraw[color=def, fill=back, ultra thick] (-2.12, 2.12) circle (0.2);
\filldraw[color=def, fill=back, ultra thick] (-2.12, -2.12) circle (0.2);
\node at (0, 3) {\Large\color{emp}N};
\node at (3, 0) {\Large\color{emp}E};
\node at (0, -3) {\Large\color{emp}S};
\node at (-3, 0) {\Large\color{emp}W};
\kmazidlo

Oznaczmy $S(n)$ jako liczbę sposobów na jakie do $S$ możemy dojść z wierzchołka $N$ w $2n$ kro-\\kach, $E(n)$ - liczbę sposobów żeby dojść do $E$ i analogicznie dla $W(n)$ oraz $N(n)$.

\subsection*{\color{emp}ŻABA BEZ KÓŁEK \color{txt}\normalsize(\emph{dla nas łatwiejsza})}

Szukamy wzoru na $S(n)$. Do wierzchołka $S$ możemy dojść dodając do dojścia do $E(n-1)$ skok $(1, 1)$ lub do dojścia $W(n-1)$ skok $(-1, -1)$:
$$S(n)=E(n-1)+W(n-1),$$
ale zauważamy, że te sposoby są symetryczne, więc do celów obliczeniowych możemy zapi-\\sać:
$$S(n)=2\cdot E(n-1).\quad\left(\kawa\right)$$

Znajdźmy teraz wzór na $E(n)$. Do wierzchołka $E$ możemy doskoczyć z wierzchołka $N$ lub z \\wierzchołka $E$ na dwa sposoby (dwa skoki 'zerowe'):
$$E(n)=N(n-1)+2\cdot E(n-1).\quad\left({\color{def}\kotecek}\right)$$

Do wierzchołka $N$ możemy dojść skacząc z wierzchołków $W$ oraz $E$ lub wykonując zerową \\parę skoków z $N$:
$$N(n)=E(n-1)+W(n-1)+2\cdot N(n-1)$$
$$N(n)=2\cdot E(n-1)+2\cdot N(n-1)\quad \left({\color{def}\kwiatuszek}\right)$$

Podstawiamy $\left({\color{def}\kwiatuszek}\right)$ do $\left({\color{def}\kotecek}\right)$:
$$E(n)=2(E(n-2)+N(n-2))+2\cdot E(n-1)\quad \left({\color{def}\baranek}\right)$$
I teraz z $\left({\color{def}\kotecek}\right)$ mamy
$$N(n-1)=E(n)-2\cdot E(n-1)$$
i wstawiamy to do $\left({\color{def}\baranek}\right)$
$$E(n)=2(E(n-2)+E(n-1)-2\cdot E(n-2))+2\cdot E(n-1)$$
$$E(n)=4\cdot E(n-1)-2\cdot E(n-2)$$

Rozwiązujemy tę rekurencję tak jak na wykładzie:
\begin{align*}
    q^n&=4\cdot q^{n-1}-2\cdot q^{n-2}\\
    q^2&=4\cdot q-2\\
    2&=q^2-4\cdot q+4\\
    2&=(q-2)^2\\
    q&=2\pm\sqrt{2}
\end{align*}

Teraz podstawiamy do wzorku z wykładu:
\begin{align*}
    &\begin{cases}
        E(0)=c_1+c_2\\
        E(1)=c_1(2+\sqrt{2})+c_2(2-\sqrt{2})
    \end{cases}\\
    &\begin{cases}
        c_1=-c_2\\
        1=c_1(2+\sqrt{2})-c_1(2-\sqrt{2})
    \end{cases}\\
    &\begin{cases}
        c_1=\frac{\sqrt{2}}2\\
        c_2=-\frac{\sqrt{2}}2
    \end{cases}
\end{align*}

W takim razie
$$2\cdot E(n)={\sqrt2}((2+\sqrt2)^n-(2-\sqrt2)^n)$$
więc podstawiąjąc do $\left(\kawa\right)$
$$S(n)=2\cdot E(n)$$
$$S(n)={\sqrt2}((2+\sqrt2)^n-(2-\sqrt2)^n)$$
\kondow

\subsection*{\color{emp}ŻABA Z KÓŁKAMI \emph{\normalsize\color{txt}(za trudne dla nas)}}

Pomysł jest podobny do rozwiązania wyżej, z tym, że możemy przejść dalej z $S$, więc \\wzorki dla poszczególnych wierzchołków to:
\begin{align*}
    N(n) &= 2\cdot N(n-1) + 2\cdot E(n-1) \quad({\color{def}\heartsuit})\\
    E(n) &= W(n) = 2\cdot E(n-1) + N(n-1) + S(n-1)\quad ({\color{def}\diamondsuit})\\
    S(n) &= 2\cdot S(n-1) + 2\cdot E(n-1)\quad({\color{def}\clubsuit})
\end{align*}\bigskip

{\color{back}nie lubie kombi}\bigskip\\
{\color{back}nie lubie algebry}\bigskip\\

\begin{center}\large
    {\large\color{emp}Lemat {\color{def}D}:} $N(n)-S(n)=2^n$
\end{center}\medskip
Wykażemy, korzystając z zasady indukcji matematycznej, że różnica między $S(n)$ i $N(n)$ \\jest zawsze $n$-tą potęgą liczby 2.\medskip\\
Dla $n=1$
\begin{align*}
    N(1)&=2\\
    S(1)&=0\\
    N(1)-S(1)&=2-0=2=2^1
\end{align*}

Zakładamy, że dla pewnego $n\geq 1$ zachodzi:
$$N(n)-S(n)=2^n.$$
Pokażemy, że wówczas
$$N(n+1)-S(n+1)=2^{n+1}$$
\begin{align*}
    L&=N(n+1)-S(n+1)=\underbrace{2\cdot N(n)+2\cdot E(n)}_{({\color{def}\heartsuit})} -\underbrace{2\cdot S(n)-2\cdot E(n)}_{({\color{def}\clubsuit})}=\\
    &=2\cdot N(n) +2\cdot S(n) = 2(N(n)+S(n))\overset{{\color{def}ind}}=2\cdot 2^n=2^{n+1}=P
\end{align*}
\begin{flushright}\emph{\scriptsize\color{def}i smiga}\\\scaleobj{0.2}{\tikz\randuck;}\end{flushright}

Przekształćmy teraz wzór $({\color{def}\diamondsuit})$
\begin{align*}
    E(n) &= 2\cdot E(n-1)+N(n-1)+S(n-1)= \underbrace{N(n)-2\cdot N(n-1)}_{({\color{def}\heartsuit})}+N(n-1)+S(n-1)=\\
    &=N(n)-N(n-1)+S(n-1) \overset{{\color{def}D}}= 2^{n} + S(n) - 2^{n-1}-S(n-1)+S(n-1)=\\
    &=S(n)+2^{n-1}
\end{align*}

Wstawiamy teraz do $({\color{def}\clubsuit})$
\begin{align*}
    S(n) = 2\cdot S(n-1) + 2\cdot (S(n-1) + 2^{n-2}) = 4\cdot S(n-1) + 2^{n-1}
\end{align*}

Trywialne rozwiązanie rekurencji zostawiamy dociekliwemu czytelnikowi.
\kondow

\end{document}