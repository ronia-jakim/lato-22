\documentclass{article}

\usepackage{../../notatka}

\begin{document}\ttfamily
\subsection*{Zad 1. Funkcja calkowalna w sensie Riemanna $f$ rozni sie od funckji $g$ w jednym punkcie z przedzialu $[a,b]$. Pokazac, ze $g$ jest calkowalna w sensie Riemanna i $\int\limits_a^bf(x)dx=\int_a^bg(x)dx$.}
    Niech $\rho\in[a,b]$ bedzie jedynym punktem takim, ze $f(\rho)\neq g(\rho)$. Rozwazmy dwa przypadki:\medskip\\
    \indent {\color{acc}1. $\rho\in\{a, b\}$.} \smallskip\\
    Wowczas, mozemy zapisac przedzial $[a,b]$ jako sume przedzialow
    $$[a,b]=[a, \;a+r]\cup[a+r,\; b-r]\cup[b-r,\; b]$$
    dla pewnego $r>0$. Poniewaz $\rho$ jest pojedzynczym punktem, mozemy wybrac dowolnie male $r$, tak, ze otrzymujemy
    \begin{align*}
        \int\limits_a^bg(x)dx&=\int\limits_a^{a+r}g(x)dx+\int\limits_{a+r}^{b-r}g(x)dx+\int\limits_{b-r}^{b}g(x)dx=\\
        &=\int\limits_a^{a+r}g(x)dx+\int\limits_{a+r}^{b-r}f(x)dx+\int\limits_{b-r}^{b}g(x)dx\xrightarrow{r\to0}0+\int\limits_a^bf(x)dx+0=\int\limits_a^bf(x)dx
    \end{align*}
    \indent {\color{acc}2. $\rho\in (a, b)$:}\smallskip\\
    Ustalmy $r>0$ takie, ze:
    $$[a,b]=\left[a, \;\rho-\frac r2\right]\cup\left[\rho-\frac r2,\; \rho+\frac r2\right]\cup\left[\rho+\frac r2,\; b\right]$$
    W takim razie mozemy rozpisac calke
    $$\int\limits_a^bg(x)dx=\int\limits_a^{\rho-\frac r2}g(x)dx+\int\limits_{\rho-\frac r2}^{\rho+\frac r2}g(x)dx+\int\limits_{\rho+\frac r2}^bg(x)dx$$
    Poniewaz funkcja $g$ rozni sie od funkcji $f$ tylko w punkcie $\rho$, to
    $$\int\limits_a^{\rho-\frac r2}g(x)dx=\int\limits_a^{\rho-\frac r2}f(x)dx\;\land\;\int\limits_{\rho+\frac r2}^bg(x)dx=\int\limits_{\rho+\frac r2}^bf(x)dx.$$
    W takim razie
    $$\int\limits_a^bg(x)dx=\int\limits_a^{\rho-\frac r2}f(x)dx+\int\limits_{\rho-\frac r2}^{\rho+\frac r2}g(x)dx+\int\limits_{\rho+\frac r2}^bf(x)dx.$$
    W trakcie dzielenia $[a,b]$ na mniejsze przedzialy, punkt $\rho$ znalazl sie w przedziale 
    a poniewaz tylko jeden punkt $\rho$ jest punktem gdzie te dwie funkcje sie roznia, mozemy ograniczac przedzial na ktorym te funkcje sie roznia, czyli dla $\rho\to0$
    \begin{align*}
        \int\limits_a^bg(x)dx&=\int\limits_a^{\rho-0}g(x)dx+\int\limits_{\rho-0}^{\rho+0}g(x)dx+\int\limits_{\rho+0}^bg(x)dx=\\
        &=\int\limits_a^{\rho}g(x)dx+\int\limits_{\rho}^{\rho}g(x)dx+\int\limits_{\rho}^bg(x)dx=\\
        &=\int\limits_a^\rho f(x)dx+0+\int\limits_\rho^bf(x)dx=\\
        \int\limits_a^bg(x)dx&=\int\limits_a^bf(x)dx
    \end{align*}
\subsection*{Zad 2. Funkcja calkowalna w sensie Riemanna $f$ rozni siie od funkcji$g$ w skonczenie wielu punktach przedzialu $[a,b]$. Pokazac, ze $g$ jest calkowalna w sensie Riemanna i $\int\limits_a^b f(x)dx=\int\limits_a^bg(x)dx$. Mozna skorzystac z poprzedniego zadania.}
    Niech $\rho_1, ..., \rho_n$ beda wszystkimi punktami przedzialu $[a,b]$ na ktorych funkcja $g$ przyjmuje wartosci rozne od $f$.Mozemy wiec podzielic $[a,b]$ na mniejsze przedzialy takie, ze:
    $$[a,b]=[a, \rho_1]\cup[\rho_1, \rho_2]\cup...\cup[\rho_n, b]$$
    Z pierwszego pierwszego podpunktu poprzedniego zadania mozna latwo zauwazyc, ze suma calek $g(x)$ na kazdym z tych przedzialow jest rowna $\int\limits_a^bf(x)dx$
    \kondow
\subsection*{Zad 3. Dla pewnego podzialu $P$ przedzialu $[a,b]$ spelniony jest warunek $L(P,f)=U(P,f)$. Roztrzygnac, czy z tego wynika, ze $f$ jest calkowalna w sensie Riemanna.}
    TAK.\bigskip\\
    Funkcja $f$ jest calkowalna w sensie Riemanna, jesli
    $$\underset{P}\sup\; L(P, f)=\underset{P}\inf\; U(P, f)=\int\limits_a^bf(x)dx$$
    Jesli funkcja, ktora spelnia warunek $L(P,f)=U(P,f)$ nie jest calkowalna w sensie Riemanna, to wowczas\medskip\\
    \indent {\color{acc}1. Istnieje taki podzial $P_2$, ze $L(P_2,f)> U(P_2,f)$}, co szybko prowadzi do absurdu. Pole pod wykresem nie moze byc ograniczone od dolu przez liczbe wieksza niz od gory.\medskip\\
    \indent {\color{acc}2. Istnieje taki podziel $P_2$, ze $L(P_2,f)< U(P_2, f)$}, ale wtedy $L(P,f)>L(P_2, f)\neq \underset{P}\sup\;L(P,f)$ oraz $U(P,f)>U(P_2, f)\neq \underset{P}\inf\;U(P,f)$. W takim razie albo
    $$\underset{P}\sup\;L(P,f)=L(P,f)=U(P,f)=\underset{P}\inf\;U(P,f),$$
    albo
    $$\underset{P}\sup\;L(P,f)>L(P,f)=U(P,f)<\underset{P}\inf\;U(P,f),$$
    co jest wypadkiem rownoznacznym z 1.
    \kondow
\subsection*{Zad 4. Funkcja $f$ jest calkowala osobno na przedzialach $[a,c]$ i $[c, b]$. Pokazac, ze $f$ jest calkowalna na przedziale $[a,b]$.}
hyyyh ja z tego do tej pory korzystalam XD

\subsection*{Zad 5. Rozstrzygnij, czy dana funkcja jest calkowalna w sensie Riemanna. Jesli jest, to oblicz jej calke po zadanym przedziale.}
{\color{emp}a. $f(x)=\begin{cases}x\quad\quad0\leq x\leq1\\-x^2\quad 1<x\leq2\end{cases}$}
\begin{center}\scaleobj{0.7}{
    \begin{tikzpicture}
    \begin{axis}[xmax=3, xmin=0, axis lines = middle, ymax=3, ymin=-5, samples=100]
        \addplot[emp, ultra thick, domain={0:1}] (x, x);
        \addplot[emp, ultra thick, domain={1:3}] (x, -1*x*x);
    \end{axis}
    \end{tikzpicture}}\end{center}
    Poniewaz interesuje nas pole miedzy wykresem funckji a osia OX, to mozemy odbic druga czesc funkcji wzgledem osi OX nie zmieniajac wartosci calki na rozwazanym przedziale.
    \begin{center}\scaleobj{0.7}{
        \begin{tikzpicture}
        \begin{axis}[xmax=3, xmin=0, axis lines = middle, ymax=5, ymin=-1, samples=100]
            \addplot[emp, ultra thick, domain={0:1}] (x, x);
            \addplot[emp, ultra thick, domain={1:3}] (x, x*x);
        \end{axis}
        \end{tikzpicture}}
    \end{center}
\subsection*{Zad 6. Funckja $f$ jest calkowalna na przedziale $[a,b]$. Uzywajac definicji calki Riemanna uzasadnic, ze dla ustalonego $c\in\R$ funkcja $f_c(x)=f(x-c)$ jest calkowalna w sensie Riemanna na przedziale $[a+c,\;b+c]$ oraz $\int\limits_a^bf(x)dx=\int\limits_{{c+a}}^{c+b}f_c(x)dx$}
    Z definicji Riemanna wiemy, ze na $f$ zachodzi:
    $$m(b-a)\leq \underset{P}\sup\;L(P, f)=\int\limits_a^bf(x)dx=\underset{P}\inf\;U(P,f)\leq M(b-a)$$
    dla kazdego $x\in[a,b]$ oraz $m\leq f(x)\leq M$.\\
    Przypuscmy, ze dla funkcji $f_c$ oraz $x\in[a+c, b+c]$ i $m_c\leq f_c(x)\leq M_c$
    $$m_c(b+c-a-c)\leq\underset{P}\sup\;L(P,f_c)\leq\underset{P}\inf\;U(P,f_c)\leq M_c(b+c-a-c),$$
    czyli
    $$m_c(b-a)\leq\underset{P}\sup\;L(P,f_c)\leq\underset{P}\inf\;U(P,f_c)\leq M_c(b-a).\quad(\kawa)$$
    Poniewaz $f_c(x)=f(x-c)$, to dla $x\in[c+a, c+b]$ $f_c$ przyjmuje te same wartosci co $f$, czyli mozemy stwierdzic nierownosc:
    $$m=m_c\leq f_c(x)\leq M_c=M.$$
    W takim razie nierownosc (\kawa) mozemy zapisac:
    $$m(b-a)\leq\underset{P}\sup\;L(P,f_c)\leq\underset{P}\inf\;U(P,f_c)\leq M(b-a),$$
    a poniewaz $f_c$ na przedziae $[a+c, b+c]$ przyjmuje nie tylko najwieksza i najmniejsza wartosc taka sama jak $f$ na $[a, b]$, ale tez wszystkie inne wartosci sa takie same, to
    $$m(b-a)\leq\underset{P}\sup\;L(P,f_c)=\underset{P}\sup\;(P,f)=\underset{P}\inf\;U(P,f)=\underset{P}\inf\;U(P,f_c)\leq M(b-a),$$
    tak wiec
    $$\int\limits_a^bf(x)dx=\underset{P}\sup\;(P,f)=\underset{P}\inf\;U(P,f)=\underset{P}\sup\;L(P, f_c)=\underset{P}\inf\;U(P, f_c)=\int\limits_{a+c}^{b+c}f_c(x)dx$$
\subsection*{Zad 10. Oblicz calke $\int\limits_{-\frac\pi2}^\frac\pi2 x\sin^2(x^3)\cos(x^3)dx$}
    Zastanowmy sie jak wygladaja poszczegolne czynniki calkowanej funkcji. Funkcja $\color{emp}f(x)=x$ jest nieprzysta:
    \begin{center}\scaleobj{0.7}{
        \begin{tikzpicture}
        \begin{axis}[xmax=4, xmin=-4, axis lines = middle, ymax=5, ymin=-5, samples=100]
            \addplot[emp, ultra thick, domain={-3.2:3.2}] (x, x);
        \end{axis}
        \end{tikzpicture}}
    \end{center}
    Natomiast $\color{def}\sin^2(x)$ orax $\color{acc}\cos(x)$ sa funkcjami parzystymi
    \begin{center}\scaleobj{0.7}{
        \begin{tikzpicture}
        \begin{axis}[xmax=4, xmin=-4, axis lines = middle, ymax=5, ymin=-5, samples=100]
            \addplot[def, ultra thick, domain={-3.2:3.2}] {sin(deg(x))*sin(deg(x)};
            \addplot[acc, ultra thick, domain={-3.2:3.2}] {cos(deg(x))};
        \end{axis}
        \end{tikzpicture}}
    \end{center}
    W takim razie, funkcja $g(x)=x\sin^2(x^3)\cos(x^3)$ jest funkcja nieparzysta - jej calka na przedziale symetrycznym wzgledem osi OY jest rowna 0.
    \kondow

\subsection*{Zad 13. Obliczyc}
\begin{align*}
    \lim\limits_{t\to\infty}\frac1{\sqrt{t^2+1}}\int\limits_0^t(\arcsin(y))^2dy&=
\end{align*}

\subsection*{Zad 14. Obliczyc calki stosujac calkowanie przez podstawianie. Dla ulatwienia, podstawianie jest podane}
{\Large\color{tit}a. $\int\limits_0^12x(x^2+2)^{2022}dx$, $u=x^2$}\bigskip\\
\begin{align*}
    \int\limits_0^12x(x^2+2)^{2022}dx&=\int\limits_0^1u^{2022}du=\\
    &=[\frac1{2023}u^{2023}]_0^1=\\
    &=[{(x^2+2)^{2023}\over2023}]_0^1=\\
    &=\frac{3^{2023}}{2023}
\end{align*}
\end{document}