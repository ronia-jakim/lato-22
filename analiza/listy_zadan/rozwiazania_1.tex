\documentclass{article}

\usepackage[utf8]{inputenc}
\usepackage{tikz}
\usepackage{amsmath}
\usepackage{mathtools}
\usepackage{amsfonts}
\usepackage{amssymb}
\usepackage{sectsty}
\usepackage{xcolor}
\usepackage[paperwidth=180mm, paperheight=290mm, left=10mm, top=10mm, bottom=10mm, right=10mm, margin=10mm]{geometry}
\usepackage{ragged2e}
\usepackage{listings}
\usepackage[hidelinks]{hyperref}

\definecolor{def}{RGB}{255, 150, 89}
\definecolor{tit}{RGB}{217, 84, 80}
\definecolor{emp}{RGB}{150, 206, 180}
\definecolor{acc}{RGB}{255, 234, 150}
\definecolor{txt}{RGB}{249, 232, 232}
\definecolor{back}{RGB}{22, 22, 22}

\subsectionfont{\ttfamily}

\DeclareFontFamily{\encodingdefault}{\ttdefault}{
  \hyphenchar\font=\defaulthyphenchar
  \fontdimen2\font=0.33333em
  \fontdimen3\font=0.16667em
  \fontdimen4\font=0.11111em
  \fontdimen7\font=0.11111em
}

\newcommand{\R}{\mathbb{R}}
\newcommand{\N}{\mathbb{N}}
\newcommand{\Q}{\mathbb{Q}}
\newcommand{\Z}{\mathbb{Z}}
\newcommand{\C}{\mathbb{C}}
\newcommand{\cont}{\mathfrak{c}}

\geometry{a4paper, textwidth=155mm, textheight=267mm, left=15mm, top=15mm, right=15mm, marginparwidth=0mm}
\setlength\parindent{15pt}

\pagestyle{empty}

\begin{document}\pagecolor{back}\color{txt}\ttfamily
\subsection*{\color{tit}ZAD 1. \color{txt}Funckja $f$ jest nieskonczenie wiele razy rozniczkowalna w otoczeniu punktu 0 i dla pewnego $n\in\N$ spelnia}
$$\lim\limits_{x\to0}{f(x)\over x^n}=0$$
\subsection*{Pokaz, ze $f^{(k)}(0)=0$ dla $0\leq k\leq n$. (Mozna na przyklad zastosowac wzor Taylora albo indukcje).}
???
\subsection*{\color{tit}ZAD 2.}
  Bezposredni wniosek z twierdzenia Rola. Stosujemy tw. Lagrange i pomiedzy dowolnymi dwoma pierwiastkami w ciagu dostajemy miejsce, gdzie pochodna sie zeruje. Takich przedzialow znajdziemy $k-1$, to pochodna ma co najmniej $k-1$ pierwiastkow
\subsection*{\color{tit}ZAD 3.}
  blad w tresci - nieparzyste pochodne maja sie zerowac a nie parzyste
  $$g(x)=f(x^2)$$
  indukcyjnie:
  $$g'(x)=(f(x^2))'=f'(x^2)2x$$
  $$g'(0)=0$$
  $$n\geq1$$
  $$g^{(2n+1)}(0)=0$$
  podstawic to do rozwiniecia taylora odpowiedniego rzedu
\subsection*{\color{tit}ZAD 4}
  Zalozmy nie wprost, ze $p(x)$ ma wiecej niz $k+2$ pierwiastkow.\\
  Policmy $p^{(k+1)}(x)$, wszystkie czynniki poza $x^n$ sie wyzeruja, a z drugiego zadania wynika, ze taki wielomian nie moze miec wiecej niz?????? \emph{analiza numeryczna, twierdzenie rola}
\end{document}