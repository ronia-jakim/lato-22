\documentclass{article}

\usepackage{notatka}

\begin{document}\ttfamily
{\large\color{acc}Znajdz jednorodny uklad rownan liniowych zlozony z\\
a) dwoch\\
b)trzech\\
c) czterech\\
rownan, ktorego zbior rozwiazan to $\Lin{(1,\;4,\;-1,\;2,\;-1)^T,\;(1,\;13,\;-1,\;2,\;9)^T,\;(2,\;7,\;-8,\;4,\;-5)^T}$.}\bigskip\\
Popatrzmy na te wektorych ktorych otoczka liniowa jest rozwiazaniem naszego szukanego rownania. Sprawdzmy, czy sa one liniowo zalezne:
\begin{align*}\alpha
    \begin{pmatrix}
        1\\4\\-1\\2\\-1
    \end{pmatrix}
    +\beta\begin{pmatrix}
        1\\13\\-1\\2\\9
    \end{pmatrix}
    +\gamma\begin{pmatrix}
        2\\7\\-8\\4\\-5
    \end{pmatrix}=0\\
    \alpha+\beta=-2\gamma\\
    \alpha+\beta=-8\gamma \implies \gamma = 0\\
    9\beta=\alpha\\
    \alpha=-\beta\implies \alpha=\beta=0
\end{align*}
czyyyli sa liniowo niezalezne.\medskip\\
Szukamy teraz takiej macierzy $A$, ze
$$A\begin{pmatrix}
    1\\4\\-1\\2\\-1
\end{pmatrix}=A\begin{pmatrix}
    1\\13\\-1\\2\\9
\end{pmatrix}=A\begin{pmatrix}
    2\\7\\-8\\4\\-5
\end{pmatrix}=0$$
Czyli dla kazdego wiersza macierzy $A$ otrzymujemy $A_iX_j=0$, gdzie $X_j$ to jeden z tych trzech wektorow ktore mamy zadane, a $A_i$ to jeden z wierszy naszej macierzy. Transponujac dostajemy $X_j^TA_i^T$. Czyli nasze wiersze macierzy musza byc rozwiazaniami
$$BY=0,$$
gdzie 
$$B=\begin{pmatrix}\begin{pmatrix}
    1\\4\\-1\\2\\-1
\end{pmatrix},\begin{pmatrix}
    1\\13\\-1\\2\\9
\end{pmatrix},\begin{pmatrix}
    2\\7\\-8\\4\\-5
\end{pmatrix}\end{pmatrix}^T$$ 
(\emph{jakie to obrzydliwe}). Wiemy teraz, ze te nasze trzy wektorki (ktorych nie bede znowu kopiowac), sa baza $\ker F_A$, czyli fundamentalnym ukladem rozwiazan $AX=0$.\medskip\\
Po tym wstepie czemu smigamy transpozycja (nadal dosc sketchy i niewiarygodnym), mozemy przejsc do rozwiazywania rownania jednorodnego
\begin{align*}
    \begin{pmatrix}
        1&&4&&-1&&2&&-1\\
        1&&13&&-1&&2&&9\\
        2&&7&&-8&&4&&-5
    \end{pmatrix}
    \begin{pmatrix}
        x_1\\x_2\\x_3\\x_4\\x_5
    \end{pmatrix}=0
\end{align*}
Gaussik <3
\begin{align*}
    \begin{pmatrix}
        1&&4&&-1&&2&&-1\\
        0&&9&&0&&0&&10\\
        0&&-1&&-6&&0&&-3
    \end{pmatrix}\\
    \begin{pmatrix}
        1&&4&&-1&&2&&-1\\
        0&&-1&&-6&&0&&-3\\
        0&&9&&0&&0&&10
    \end{pmatrix}\\
    \begin{pmatrix}
        1&&4&&-1&&2&&-1\\
        0&&-1&&-6&&0&&-3\\
        0&&0&&-54&&0&&-1
    \end{pmatrix}
\end{align*}
nie bede sie ponizac do rozwiazania tego\bigskip\\

\end{document}