\documentclass{article}

\usepackage{../../../notatka}


\begin{document}\ttfamily
\section*{SZEREGI FORMALNE (i nieformalne)}
$$f(x)=\sum\limits_{n=0}^\infty a_nx^n=a_0+a_1x+...$$
W analizie $f$ okreslona na $(-\R, \R)$, gdzie $\R=\frac{1}{\lim\sup \sqrt[n]{a_n}}$ - cos z promieniem zbieznosci.\medskip\\
Dodawanie nam smiga normalnie
$$\sum\limits_{n=0}^\infty a_nx^+\sum\limits_{n=0}^\infty b_nx^n$$
Mnozenie
$$(\sum\limits_{n=0}^\infty a_nx^n)(\sum\limits_{n=0}^\infty b_nx^n)=\sum\limits_{n=0}^\infty c_nx^n,\quad gdzie\; c_n=\sum\limits_{i=0}^n a_ib_{n-i}$$
Wezmy teraz
$$1+x+x^2+...=\frac1{1-x}$$
Mozemy to pokazac analitycznie, a mozemy zrobic hop siup
$$(1-x)+x(1-x)+x^2(1-x)+...=1$$
$$1-x+x-x^2+x^2-x^3+...=1+0=1$$
Przyklad\medskip\\
$$\frac1{\sum\limits_{n=0}^\infty a_nx^n}= \sum\limits_{n=0}^\infty b_nx^n,\quad a_n\neq0$$
To znaczy, ze
$$1=(\sum\limits_{n=0}^\infty a_nx^n)(\sum\limits_{n=0}^\infty b_nx^n)$$
no i smiga, nie chce mi sie dzisiaj pisac\medskip\\
\begin{center}\large
{\color{def} FUNKCJA TWORZACA CIAGU} $a_0, a_1, ...$ to $f(x)=\sum\limits_{n=0}^\infty a_nx^n$
\end{center}
Manipulowanie ciagami czasami pozwala nam rozwiazywac rekurencje.\medskip\\
Zaczniemy od Fibonacciego\bigskip\\
$a_0=a_1=1$
$$a_n=a_{n-1}+a_{n-2}$$
wezmy funkcje tworzaca
$$t(x)=\sum\limits_{n=0}^\infty a_nx^n=a_0+a_1+...=1+x+(a_1+a_0)x^2+(a_2+a_1)c^3+...=$$
$$=1+x+(a_1x^2+a_2x^3+...)+(a_0x^2+a_1x^3+...)=1+x+x(t(x)-1)+x^2(t(x))$$
Mamy rownanie uwiklane, chce je rozwiazac z punktu widzenia $t(x)$.\medskip\\
$$t(x)=\frac1{1-x-x^2}=\frac1{(p_1-x)(p_2-x)}= zapisac sobie jako roznice ulamkow$$
Czyli zapisujemy to sobie
$$\frac A {1-cx}+\frac B{1- dx}=A(1+cx+c^2x^2+...)+B(1+dx+d^2x^2+...)$$
czyli nasze $a_n=Ac^n+Bd^n$. Wydaje sie byc bardziej zawile niz oryginalny pomysl tego ziomka z Pizy, ale ma wiecej zastosowan.\bigskip\\
PRZYKLAD - liczby Catalona\\
Rozwazmy funkcje tworzaca
$$g(x)=\sum\limits_{n=1}^\infty c_n x^n$$
Potrzebuje $g^2(x)$
$$g^2(x)=g(x)g(x)=(c_1x+c_2x^2+...)(c_1x+c_2x^2+...)=(c_1^2x^2)+(c_2c_1+c_1c_2)x^3+...$$
$$=c_1^2x^2+c_3x^3+c_4x^4+...=g(x)-x$$
W takim razie mamy wniosek
$$g^2(x)-g(x)+x=0$$
$$g(x)=\frac{1\pm\sqrt{1-4x}}2$$
interesuje nas rozwiazanie gdzie dla malych wartosci $x$ mamy zeczy dodatniie, wiec
$$g(x)=\frac{1-\sqrt{1-4x}}2$$

\end{document}