\documentclass{article}

\usepackage{../../../notatka}

\begin{document}\ttfamily
\subsection*{Zad 1. W brydza gra czterech graczy; na poczatku kazdy dostaje 13 kart z talii 52 kart.\\
Ile jest sposobow potasowania talii 52 kart? Ile roznych ukladow kart moze dostac gracz? Na ile sposobow mozna rodac talie?}
    1. Jest $52!$ sposobow potasowania kart.\bigskip\\
    2. Kazdy gracz dostaje zbior 13-elementowy ze zbioru 52-elementowego, a taki zbior mozna wybrac na ${52\choose13}$ sposobow\bigskip\\
    3. Dla pierwszego gracza talie mozemy wybrac na ${52\choose 13}$ sposobow, dla drugiego nie mozemy juz wykorzystac tych 13 rozdanych kart, wiec on moze dostac swoja talie na ${39\choose13}$ sposobow, gracz 3 moze dostac swoja talie na $26\choose 13$ sposobow, a dla 4 zostaje dokladnie $13$ kart. Tak wiec ilosc sposobow na ktore mozna rozdac 13 na 4 graczy to
        $${52\choose13}\cdot{39\choose13}\cdot{26\choose13}={52!\over(13!)^4}$$
\subsection*{Zad 2. Ile roznych dodatnich dzielnikow ma liczba $3^4\cdot5^2\cdot7^3\cdot11$, a ile liczba 620?}
    3 do dzielnika mozemy wybrac 0, 1, 2, 3 lub 4 razy - mamy 5 sposobow wyboru 3\smallskip\\
    5 do dzielnika mozemy wybrac - 0, 1, lub 2 razy - 3 sposoby wyboru potegi 5\smallskip\\
    7 - 0, 1, 2 lub 3 razy - 4 sposoby wyboru potegi 7,\smallskip\\
    natomiast 11 mozemy wziac lub jej nie brac\medskip\\
    W takim razie roznych dzielnikow mamy $5\cdot3\cdot4\cdot2=120$\bigskip\\
    $$620=2^7\cdot5$$
    w takim razie dzielnikow jest $8\cdot 2=16$
\subsection*{Zad 3. Ile zer na koncu ma liczba 50!?}
    Wystarczy znalezc liczbe $5$ w rozkladzie na czynniki pierwsze liczby $50!$, czyli:
    $$\lfloor{50\over5}\rfloor + \lfloor{50\over25}\rfloor=12$$
\subsection*{Zad 4. Ile liczb wiekszych od 5400 ma rozne cyfry, wsrod ktorych nie wystepuja 2 i 7?}
NIEDOKLADNIE PRZECZYTALAM ZADANKO\\
    Na pierwszym miejscu moze pojawic 6, 8 lub 9, wtedy na drugim miejscu zostaje nam 7 cyfr, na 3 - 6, a na 4 - 5 cyfr.
    $$3\cdot7\cdot6\cdot5$$
    Moze sie tez zdazyc, ze na pierwszym miejscu pojawia sie 5, a na drugim 6, 7, 8 lub 9, takich liczb jest
    $$1\cdot4\cdot6\cdot5$$
    Czyli ogolem liczb wiekszych od 5400 o roznych cyfrach jest
    $$21\cdot30+4\cdot30=24\cdot30=720$$
\subsection*{Zad 5}
    $$|X|=5^4$$
    $$|A|=5!$$
    $$|B|=5^3\cdot3$$
    $$|A\cap B|=2\cdot4\cdot3\cdot2$$
\subsection*{Zad 8}
    na ile sposobow mozemy wybrac pare kolejnych liczb
    $$(n-2)$$
    i tworzy sie wokol niej buffor jedno w przod i jedno w tyl\\
    trzecia liczbe mozemy wybrac na $n-3$ sposobow i wstawic ja mozemy na $3$ miejsca
    $${(n-3)(n-2)\over3\cdot2}$$
    nad tym jeszcze pomyslec
\subsection*{Zad 9. }
    nic prosciej sie nie da
\subsection*{Zad 10. }
    8! ustawien - patrzymy po ich wspolrzednych
\subsection*{Zad 11.}
    
\end{document}