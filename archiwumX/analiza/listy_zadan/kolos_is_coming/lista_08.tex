\documentclass{article}

\usepackage{../../../notatka}

\begin{document}\ttfamily
{\large\color{def}1. OBLICZYC POLE OBSZARU OGRANICZONEGO PRZEZ WYKRESY ROWNAN}\bigskip

\indent {\color{acc}(a) $y=x^3$, $y=x^\frac13$}\medskip\\
\emph{juz wewnetrznie cierpie}

Najpierw sprawdzamy gdzie te dwie funkcje sie przecinaja, czyli rozwiazujemy rownanie
$$x^3=x^\frac13$$
$$x^9=x$$
$$x(x^8-1) = 0$$
czyli
$$x=-1\quad x=0\quad x=1.$$
Mozemy wiec rozdzielic pole na dwie czesci:
$$\int\limits_{-1}^0|x^3-\sqrt[3]{x}|dx+\int\limits_0^1|x^3-\sqrt[3]{x}|dx,$$
ale latwo zauwazyc, ze
$$\int\limits_{-1}^0|x^3-\sqrt[3]{x}|dx=\int\limits_0^1|x^3-\sqrt[3]{x}|dx,$$
wiec nasze pole to bedzie
$$2\int\limits_{0}^1 |x^3-\sqrt{x}|dx = 2|\int\limits_0^1(x^3-\sqrt[3]{x})dx|=2|\frac{x^4}4-\frac34x^{\frac43}|_0^1=2\cdot \frac12=1$$

\end{document}