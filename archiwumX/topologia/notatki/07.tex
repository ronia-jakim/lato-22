\documentclass{article}

\usepackage{../../notatka}

\begin{document}\ttfamily
\section*{COS}
Twierdzenie $C[0,1]$ z metryka supremum jest spojna.\\
\dowod
Latwiej jest udowodnic, ze cos jest lukowo spojne (wlasnosc silniejsza). Wezmy dowolna funkcje $f\in C[0,1]$ i polaczmy ja lukiem z najprostsza funkcja $g$, czyli ta stale rowna 0. Wystarczy laczyc funkcje lukami z nia bo jak mam dwa luki, to ich suma tez jest lukiem.\bigskip\\
$F:[0,1]\to C[0,1]$\\
$F(0) = g$\\
$F(1)=f$\\
to sie sprowadza do znalezienia
$$G:[0,1]\times[0,1]\to \R$$
$$G(x,t)=F(x)$$
Startujemy of funkcji zerowej i mamy dojsc do funkcji $f$ w sposob ciagly. Wygodnie o tym myslec jako o czasiie, czyli $g$ plynnie zmienia sie w $f$
$$\forall\;x\quad G(x, 0) = 0$$
$$\forall\;x\quad G(x, 1) = f(x)$$
$$\forall\;t\quad G(x, t) \in C[0,1]$$
$$F(t)=G(x, t)$$
$G$ jest homotopia.
$$G(x, t)= tf(x)$$
wlozenie na cwiczeniach
\section*{PRODUKTY}
$(X_i)_{i\in I}$ - rodzina indeksowana przestrzeni topologicznych.
$$\prod_{i\in I} X_i \ni f$$
$$f:I\to \bigcup\limits_{i\in I} X_i$$
$$f(i)\in X_i$$
Jak zdefiniowac topologie?\\
Topologia produktowa (trzy kreseczki) najslabsza topologia taka, ze wszystkie rzuty sa ciagle\bigskip\\
$$i\in I\quad p_i\prod\limits_{i\in I} X_i\to X_i$$
$$p_i(f)=f(i)$$
niech warunek wstepny zeby te rzuty byly ciagle jest\bigskip\\
$$p_i^{-1}[V] \;sa\;otwarte\;w\;prod$$
gdzie $V$ jest otwartym podzbiorem $X_i$\\
$$p_i^{-1}[V]=\{f\in X\;:\;f(i)\in V\}$$
to jak bramki z wczesniej\bigskip\\
Baza topologii produktowej sa zbiory psotaci
$$p_{i_1}^{-1}[V_{i_1}]\cap ...\cap p_{i_n}^{-1}[V_{i_n}]$$
$$X_0\times X_1\times U_2\times X_3\times...$$
baza jest produkt tego prawie wszystkiego, gdzie tylko na skonczenie wielu miejscach nie ma pelnego $X$.\\
PRZYKLADY
$\R\times\R$ powyzsza definicja daje nam $\R^2$\\
$\R\times$strzalka - $\R\times\R$ $\{U\times V\;:\;U otwarte euklidesowo, V otwarte strzalkowo\}$\\
$\{0, 1\}\times\{0, 1\}\times...$ przeliczalnie wiele razy, czyli $\{0,1\}^\N$ ustalamy skonczenie wiele osi i dostajemy baze na kostke cantora\\
$[0, 1]\times [0, 1]\times...$ przeliczalnie wiele razy, wtedy mamy $[0,1]^\N$ i dostajemy kostke hilberta\\
$\prod\limits_{i\in I}X_i,\; X_i=\R$ i to jest $\R^\R$ i topologia jest $\{f\in\R^\R\;:\;f(x_0)\in U_0,..., f(x_k)\in U_k\}$, ale mozemy wziac sobie $R^{[0,1]} \supseteq C[0,1]$ top prod to top zb punkt\bigskip\\
Twierdzenie Tichonowa - jesli mamy rodzine przestrzeni topologicznych $(X_i)_{i\in I}$ zwartych to produkt jest zwarty\\
\dowod
$I=\N$
najprostszy nieskonczony przypadek\\
Zalozmy, ze istnieje takie zle pokrycie U bez podpokrycia skonczonego. Skonstruujemy pewien $x\in X$\\
Indukcja <3\\
Zaczynamy od $x_1$ takiego, ze
$$\forall \;V\subseteq X_1\quad V\times X_2\times X_3\times... nie jest pokrywany skonczenie wieloma elementami$$
gdyby nie, to
$$\forall\;x\in X_1\;\exists\;V_x\ni x\quad v\times X_2\times X_3\times... pokrywa sie skonczenie wieloma elementami U$$
$$\{V_x\;:\;x\in X_1\}$$
jest pokryciem $X_1$
$$V_{x_1}\cup...V_{x_n}=X_1$$
do $V_{x_1}$ dobieram sobie skonczone podpokrycie $U_1\supseteq U$ ktore jest skonczone takie, ze
$$V_{x_1}\times X_2\times... \subseteq \bigcup U_1... (\kawa)$$
teraz trzeba sie przyjrzec $U_1\cup U_2\cup...$ - to jest ba pewno skonczona podrodzina U i jest pokryciem, bo sumujac tego potworka nalezacrgo do Vx1 x X2 x V3 x... nalezy mi do sumy popokrycia
$$\{V_{x_k}\times X_2\times...\;:\;k\leq n\} jest pokryciem X$$
wiec z warunku (\kawa) mamy, ze $U_1\cup U_2\cup..$ rowniez (tylko tu sa dwa rozne rodzaje U XD)\\
$x_2$ takie, ze $\forall\;U\subseteq X_1, V\subseteq X_2 \quad U\times V\times X_3.. nie jest skonczonym pokryciem elementw U$

\end{document} 