\section{ROZMAITOŚCI}
\subsection{RELACJA RÓWNOWAŻNOŚCI}
$$X=[0,1]^2$$
$$\parl 0, y\parr \sim \parl 1,y\parr$$
\pmazidlo
    \draw[white, ultra thick] (0, 0)--(2, 0)--(2, 2)--(0, 2)--cycle;
    \draw[def, ultra thick] (-0.2, 0.9)--(0, 1.1)--(0.2, 0.9);
    \draw[def, ultra thick] (1.8, 0.9) -- (2, 1.1) -- (2.2, 0.9);
\kmazidlo
Tutaj zlepiamy dwa przeciwne boki prostokąta i otrzymujemy tubę.\smallskip\\
\podz{gr}\smallskip\\
$$\parl 0,y\parr\sim\parl 1,1-y\parr$$
\pmazidlo
    \draw[white, ultra thick] (0, 0)--(2, 0)--(2, 2)--(0, 2)--cycle;
    \draw[def, ultra thick] (-0.2, 0.9)--(0, 1.1)--(0.2, 0.9);
    \draw[def, ultra thick] (1.8, 1.1) -- (2, 0.9) -- (2.2, 1.1);
\kmazidlo
Powstaje nam wstęga Mobiusa.\smallskip\\
\podz{gr}\smallskip\\
$$parl x, 0\parr\sim\parl x, 1\parr\cup\parl 0, y\parr\sim\parl 1, y\parr$$
\pmazidlo
    \draw[white, ultra thick] (0, 0)--(2, 0)--(2, 2)--(0, 2)--cycle;
    \draw[def, ultra thick] (-0.2, 0.9)--(0, 1.1)--(0.2, 0.9);
    \draw[def, ultra thick] (1.8, 0.9) -- (2, 1.1) -- (2.2, 0.9);
    \draw[emp, ultra thick] (0.7, 1.8) -- (0.9, 2) -- (0.7, 2.2);
    \draw[emp, ultra thick] (1.1, 1.8) -- (1.3, 2) -- (1.1, 2.2);
    \draw[emp, ultra thick] (0.7, -0.2) -- (0.9, 0) -- (0.7, 0.2);
    \draw[emp, ultra thick] (1.1, -0.2) -- (1.3, 0) -- (1.1, 0.2);
\kmazidlo
Zlepiamy górę i dół oraz prawo i lewo - dostajemy torus.\smallskip\\
\podz{gr}\smallskip\\
$$\parl x, 0\parr\sim\parl 1-x, 1\parr\cup\parl 0, y\parr\sim\parl 1, y\parr$$
\pmazidlo
    \draw[white, ultra thick] (0, 0)--(2, 0)--(2, 2)--(0, 2)--cycle;
    \draw[def, ultra thick] (-0.2, 0.9)--(0, 1.1)--(0.2, 0.9);
    \draw[def, ultra thick] (1.8, 0.9) -- (2, 1.1) -- (2.2, 0.9);
    \draw[emp, ultra thick] (0.7, 1.8) -- (0.9, 2) -- (0.7, 2.2);
    \draw[emp, ultra thick] (1.1, 1.8) -- (1.3, 2) -- (1.1, 2.2);
    \draw[emp, ultra thick] (1.3, -0.2) -- (1.1, 0) -- (1.3, 0.2);
    \draw[emp, ultra thick] (0.9, -0.2) -- (0.7, 0) -- (0.9, 0.2);
\kmazidlo
Jak to się zrobi na przemian strzałki na górze i dole to dostajemy torus z obrotem, \\czyli {\color{acc}butelkę Kleina}.\smallskip\\
\podz{gr}\smallskip\\
$$\parl x, 0\parr\sim\parl 1-x, 1\parr\cup\parl 0, y\parr\sim\parl 1, 1-y\parr$$
\pmazidlo
    \draw[white, ultra thick] (0, 0)--(2, 0)--(2, 2)--(0, 2)--cycle;
    \draw[def, ultra thick] (-0.2, 0.9)--(0, 1.1)--(0.2, 0.9);
    \draw[def, ultra thick] (1.8, 1.1) -- (2, 0.9) -- (2.2, 1.1);
    \draw[emp, ultra thick] (0.7, 1.8) -- (0.9, 2) -- (0.7, 2.2);
    \draw[emp, ultra thick] (1.1, 1.8) -- (1.3, 2) -- (1.1, 2.2);
    \draw[emp, ultra thick] (1.3, -0.2) -- (1.1, 0) -- (1.3, 0.2);
    \draw[emp, ultra thick] (0.9, -0.2) -- (0.7, 0) -- (0.9, 0.2);
\kmazidlo
Na przemian wszystkie strzałki i dostajemy {\color{acc}płaszczyznę rzutową}.

\subsection{ROZMAITOŚĆ}
\begin{center}\large
    {\color{def}N - ROZMAITOŚĆ} to przestrzeń topologiczna, \\łukowo spójna, lokalnie homeomorficzna z $\R^n$ (to znaczy, \\że $(\forall\;x\in X)(\exists\;U\underset{otw}{\ni} x)\;U\cong \R^n$)
\end{center}
kula - przykład, rura (z końcem) - antyprzykład\bigskip\\
{\color{acc}Czym się różni sfera od torusa?} \medskip\\
Wyobraźmy sobie pętelkę na spherze, jeśli będziemy ją ściskać, to zrobimy supełek. Na-\\tomiast jeśli na torusie weźmiemy pętelkę ale taką oplatającą go, to tego nie \\możemy ścisnąć do supełka.\bigskip
\begin{center}\large
    {\color{def}PĘTLA} to funkcja\smallskip\\
    $p:[0,1]\to X$\smallskip\\
    która jest ciągła i $p(0)=p(1)$
\end{center}\bigskip
Funkcje stałe też są pętlami.\bigskip
\begin{center}\large
    Przestrzeń topologiczna jest {\color{def}JEDNOSPÓJNA}, \\gdy jest łukowo spójna i dla każdej pętli istnieje \\punkt, z którym jest ona homotopijnie równoważna ({\color{emp}jest ściągalna do punktu}).
\end{center}
jakaś dygresja\bigskip
\begin{center}\large
    $X\cong Y$,\\
    jeśli $X$ jest jednospójna, to $Y$ jest też jednospójna.
\end{center}
Weźmy koło bez brzegu i pół okręgu z otwartymi końcami. Obie te przestrzenie są jedno-\\spójne, ale jeśli z koła wyjmiemy jeden punkt, to przestaje ono być jednospójne, a \\jeśli z pół okregu wyjmiemy, to on nadal jest jednospójny.

\subsection{PRZESTRZEŃ ŚCIĄGALNA}
\begin{center}\large
    Przestrzeń topologiczna jest {\color{def}ŚCIĄGALNA,} \\jeżeli identyczność jest homotopijna z pewną funkcją stała\smallskip\\
    $id:X\to X\quad id(x) = x$\smallskip\\
    $f:X\to X\quad f(x)=a$
\end{center}\bigskip
Na przykład dysk jest ściągalny:
$$H(x, t)=t\cdot a+(1-t)x,$$
identyczność jest homotopijnie spójna z funckją $f(x)=a$. Sfera nie jest homotopijnie \\spójna.\bigskip\\
\podz{gr}\bigskip
\begin{center}\large
    {\color{def}TWIERDZENIE BROUWERA}\smallskip\\
    jeśli istnieje ciągła
    $f:D^n\to D^n,$
    gdzie $D^n$ to dysk $n$-wymiarowy, to\smallskip\\
    $(\exists\;x)\;f(x)=x$,\smallskip\\
    czyli istnieje punkt stały.
\end{center}
\dowod
Dla $n=2$.\medskip\\
Wyobraźmy sobie, że mamy ciągłą funkcję
$$f:D^2\to D^2,$$
któa nie ma punktu stałego
$$(\forall\;x)\;f(x)\neq x.$$
Korzystając z niej konstruujemy drugą funkcję
$$r:D^2\to S^1,$$
gdzie $S^1$ to brzeg $D^2$. Prowadzimy prostą przez $x$ i jego obraz $f(x)$, po czym przypisujemy $r(x)$ jako \\punkt przecięcia tej prostej i $S^1$. Co możemy o $r$ powiedzieć?\medskip\\
\indent - $r$ jest ciągła\smallskip\\
\indent - $r_{\obet S^1}= id_{\obet S^1}$\medskip\\
Rozważmy funkcję:
$$H:D^2\times[0,1]\to S^1$$
$$H(x, t)=r(tx)$$
$H$ jest ciągłe, $H(x, 0) = r(0)$, gdzie $r(0)$ bedzie środkiem dysku $H(x,1)=r(x) = x$. Czyli dos-\\taliśmy, że okrąg jest ściągalny, ale on nie jest więc mamy sprzeczność
\kondow