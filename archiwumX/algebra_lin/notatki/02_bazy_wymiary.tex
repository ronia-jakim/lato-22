\documentclass{article}

\usepackage{../../notatka}

\begin{document}\ttfamily
\section*{BEZY I WYMIARY}
    \begin{center}\large
    \color{def}BAZA \color{txt}przestrzeni liniowej $V$ nazywamy \\
        taki podzbior $B\subseteq V$, ktory:
    \end{center}
    $$\color{emp}\texttt{1. }B\texttt{ jest lnz i }\Lin{B}=V$$
    Czyli $B$ rozpina cala przestrzen $V$.

    $$\color{emp}\texttt{2. }\forall\;v\in V\setminus B\quad B\cup\{v\}\texttt{ jest lz}$$
    Wynika z poprzedniego zalozenia oraz tego, ze $B\cup\{v\}$ jest liniowo zalezny jesli $v\in V\setminus B$. \color{tit}CWICZENIA\color{txt}

    $$\color{emp}\texttt{3. }B\texttt{ jest max lnz}$$
    Jesli $B$ daloby sie powiekszyc do jakiegos liniowo niezaleznego zbioru istotnie wiekszego $A$, to moglibysmy wziac jeden element $a\in A\setminus B$ i wowczas $B\cup\{a\}$ jest liniowo zalezny, wiec mamy sprzecznosc.

    $$\color{emp}\texttt{4. }\forall\;v\in V\quad v\texttt{ zapisuje sie jednoznaczie jako } \sum\limits_{b\in B}\alpha_bb$$
    Wezmy $v\in V$:\smallskip\\
    \indent jesli $v\in B$ to oznacza, ze sam siebie zapisuje,\smallskip\\
    \indent jesli $v\notin B$, to wowczas z dwoch poprzednich twierdzen wiemy, ze $B\cup\{v\}$ jest liniowo zalezne. To znaczy, ze \emph{\color{acc}pewna nietrywialna kombinacja liniowa wektorow z $B\cup\{v\}$ jest zerowa}:
        $$\alpha\cdot  v+\sum\limits_{b\in B}\alpha_bb=0.$$
    Gdyby $\alpha=0$, to wowczas wszystkie $\alpha_b=0$. Czyli kombinacja liniowa wektorow z $B\cup\{v\}$ jest 0 tylko wtedy, gdy wszystkie wspolczynniki sa zerowe, a to oznaczaloby, ze $B\cup\{v\}$ jest lnz - \emph{sprzecznosc}.\smallskip\\
    W takim razie $\color{acc}\alpha\neq0$, wiec:
        $$\alpha\cdot v=-\sum\limits_{b\in B}\alpha_bb$$
        $$v=\sum\limits_{b\in B}(-\alpha^{-1}\alpha_b)b.$$
    Pokazalismy, ze $v$ mozna zapisac jako kombinacje liniowa wektorow z $B$. \color{acc}Zalozmy, ze istnieja dwie takie kombinacje liniowe\color{txt}:
        $$v=\sum\limits_{b\in B}\alpha_bb$$
        $$v=\sum\limits_{b\in B}\beta_bb.$$
    Odejmujac obie strony rownania dostajemy:
        $$\sum\limits_{b\in B}(\alpha_b-\beta_b)b=0.$$
    Skoro $B$ jest lnz, to wszystkie $\alpha_b-\beta_b=0$, a wiec $\alpha_b=\beta_b$.\bigskip\\
    Pozostaje nam udowodnic implikacje $\color{emp}4.\implies1.$\smallskip\\
    4. mowi, ze kazdy wektor $v\in V$ zapisuje sie jednoznacznie jako kombinacja liniowa elementow $B$. Z tego wynika, ze 
        $$\Lin{B}=V,$$ 
    a skoro $B$ jest lnz, to w szczegolnosci \color{acc}wektor 0 zapisuje sie jednoznacznie\color{txt}:
        $$\sum\limits_{b\in B}\alpha_bb=\vzer=\sum\limits_{b\in B}0\cdot b=0$$
    Z jednoznacznosci zapisu wektorow mamy dla kazdego $\alpha_b=0$, w takim razie $B$ jest lnz.\bigskip\\
    \podz{gr}\bigskip\\
    
    \color{emp}\large PRZYKLADY\color{txt}\normalsize:\medskip\\

    Baza $K^n$ jest zbior $\{e_1, e_2, ..., e_n\}$, takich, ze na $k$-tej pozycji wektor $e_k$ ma 1, a na pozostalch 0 (czyli zbior weresorow).\bigskip\\

    Jesli $A$ jest skonczony, to baza $K^A$ jest zbior funkcji postaci 
        $$f_a(x)=\begin{cases}1\quad x=a\\0\quad x\neq a.\end{cases}$$
    Ten zbior jest liniowo niezalezny, jezeli $\sum\limits_{a\in A}\alpha_af_a=\vzer$. Dla kazdego $b\in A$ mamy
        $$\sum\alpha_af_a(b)=0,$$
    bo ta funkcja zawsze daje 0 poza $f_a(a)$, wiec zbior jest liniowo niezalezny.\smallskip\\
    Wezmy $g\in K^A$. Wowczas mozemy te funckje zapisac jako 
        $$g=\sum\limits_{a\in A} \underbrace{f(a)}_{\in K}\cdot f_a$$
    Wtedy 
        $$g(b)= \sum f(a)\cdot f_a(b),$$ 
    ktore faktycznie tyle wynosi, bo prawie wszystko sie zeruje poza tym jednym wyrazem gdzie jest 1 i tam mamy $g_a(b)$.\medskip\\

    Jesli \color{acc}$A$ jest nieskonczone\color{txt}, to 
        $$\{f_a\;:\;a\in A\},$$
    jest lnz, ale nie rozpina calego zbioru. Na przyklad \emph{funkcja stala ktora zawsze przyjmuje 1} nie moze byc zapisana jako kombinacja liniowa wektorow z $\{f_a\;:\;a\in A\}$.\medskip\\

    W zbiorze wszystkich wielomianow o wspolczynnikach z X, $W[X]$, mamy baze $\{1, X, X^2, X^3, ...\}$. \medskip\\
    Jesli nasze wielomiany maja co najwyzej okreslony stopien $n$, to wtedy baza zbiory $K_n[X]$ jest rowna $\{1, X, X^2, X^3, ..., X^n\}$.
    \bigskip\\\podz{tit}\bigskip\\

    \begin{center}
        \large\color{def}LEMAT KURATOWKIEGO-ZORNA \color{txt}(LKZ) - jezeli mamy zbior czesciowo \\
        uporzadkowany $(P, \leq)$ taki, ze $P\neq \emptyset$ i kazdy lancuch w $P$ \\
        ma ograniczenie gorne, to wtedy $P$ ma element maksymalny.
    \end{center}\bigskip

    \begin{center}\large
        \color{def}TWIERDZENIE O ISTNIENIU BAZY \color{txt}- kazda przestrzen liniowa ma baze.
    \end{center}
    Ustalmy dowolna przestrzen liniowa $V$ nad cialem $K$. Chcemy zastosowac lemat K-Z. Niech $P=\{\texttt{liniowo niezalzezne podzbiory }V\}$ iuporzadkowane przez $\leq\subseteq$. Na pewno $P\neq\emptyset$, bo $\emptyset\in P$.\smallskip\\
    Wezmy $L\leq P$, ktory jest lancuchem. Wtedy $l^*=\bigcup L=\{v\;:\;\exists\;l\in L\quad v\in l\}$ jest ograniczeniem gornym. Wystarczy sprawdzic, ze $l^8\in P$. Wezmy dowolny uklad $v_1, ..., v_n\subset l^*$ roznych wektorow. Chcemy sprawdzic, czy jest on lnz. Kazdy $v_k\in l_k\in L$, ale poniewaz $L$ jest lancuchem, to
    $$\exists\;k_0\;\forall\;k\quad l_{k_0} \supseteq l_k$$
    Wtedy $v_1, ..., v_n\in l_{k_0}\in P$, wiec jest lnz.\smallskip\\
    Z LK-Z $P$ ma element maksymalny, czyli $V$ ma baze.\bigskip\\
    Jezeli $V$ jest pzestrzenia liniowa i mamy jej podzbiory$N\subseteq G\subseteq V$ takii, ze $N$ jest lnz, a $\Lin{G}=V$ ($G$ rozpina przestrzen $V$), to wtedy istnieje baza dla $V$ taka, ze $N\subseteq N$ i $b\subseteq G$.\medskip\\
    Rozwazamy $P=\{A\subseteq G\;:\;N\subseteq A\;\land\;A\texttt{ jest lnz}\}$. $P\neq\empty$, bo $N\in P$. Drugie zalozenie LK-Z sprawdzamy analogicznie do poprzedniego dowodu. Stad dostajemy analogicznie maksymalny liniowo niezalezny podzbior $B\subseteq G$, ktory jest nadzbiorem $N$. Zostaje sprawdzic, ze on jest baza, czyli rozpina $V$.\smallskip\\
    Poniewaz $B$ jest max lnz w $G$. W takim razie $\forall\;g\in G\quad g\in\Lin{B}$, czyli $G\subseteq\subseteq\Lin{B}$. Skoro $\Lin{G}=V$, to $\Lin{G}=V\subseteq\Lin{\Lin{B}}=\Lin{B}$.\bigskip\\
    Jezeli $V$ jest przestrzenia liniowa, to wtedy $\forall\;N\subseteq V \texttt{lnz}\;\exists\; b\supseteq N$ oraz $\forall\;G\subseteq V\quad\Lin{G}=V\;\exists\;B\subseteq G$\bigskip\\
    \color{tit}CWICZENIA \color{txt}$v_a, ..., v_k$ - ln i $v_{k+1}$ nie jest kombinacja lin $v_1, ..., v_k$, to wtedy $v_1, ..., v_{k+1}$ jest lnz\medskip\\
    Zalozmy, ze $V=\Lin{v_1, .., v_k}$ i zdefiniujmy rekurencyjnie podzbiory:
    $$B_0=\empty\quad B_{k+1}=\begin{cases}B_k\quad v_{k+1}\in\Lin{B_k}\\B_k\cup v_{k+1}\end{cases}$$
    Wtedy $B_n$ jest baza $V$.\medskip\\
    Dowod: $v_k\in\Lin{{B_k}}\subseteq\Lin{B_n}$ bo w innym przypadku dorzucamy go w kroku rekurencyjnym.To teraz wiemy, ze $\Lin{B_n}\supseteq\Lin{v_1, ..., v_n}$, czyli $B_n$ rozpina $V$.\\
    Pokazujemy, ze $B_n$ jest lnz przez indukcje:\\
    $B_0$ jest lnz\\
    Jezeli $B_k$ jest lnz, to wtedy\\
    a. jesli $V_{k+1}\in\Lin{B_k}$, to wtedy $B_{k+1}=B_k$ i jest lnz\\
    b. jesli $v_{k+1}\notin \Lin{B_k}$, to wtedy $B_{k+1}$ jest liniowo niezalezny. 
\subsection*{LEMAT STEINITZA}
Jesli $B$ jest baza $V$, a $a_1, ..., a_n\in V$ sa lnz, to\\
$B$ ma przynajmniej $n$ elementow\\
$B$ ma $c_1, ..., c_n\in B$ takie, ze $(B\setminus\{c_1, ..., c_n\}\cup\{a_1, ..., a_n\})$ jest baza.\medskip\\
Wniozek to twierdzenie o wymiarze - kazde dwie bazy $V$ maja tyle samo elementow.\\
Dowod tylko kiedy jedna z baz jest skonczona.\\
Niech $B_1, B_2$ to skonczone bazy $V$. Z tw. dla $B_1$ i ciagu$\{a_1, ..., a_n\}=B_2$ dostajemy $|B_1|\geq n=|B_2|$. Symetrycznie, $|B_2|\geq |B_1|$. W takim razie, $|B_1|=|B_2|$. \\
WYMIAR przestrzeni liniowej $V$ ($\dim V$) to moc dowolnej bazy $V$. \\
Na przyklad
$$\dim K^n=n$$
$$\dim_\C\C^n=n\quad \dim_\R\C^n=2n$$
$$\dim_\Q\C=2^{\aleph_0}=\cont$$
Jesli $B$ jest baza $V$ i jakis wektor $a=\sum\limits_{b\in B}\alpha_b b$, to wtedy dla $c\in B$ taie, ze $\alpha_c\neq 0$, to mozemy wyrzucic $c$ i dodac $a$ i dostajemy baze $V$\\
Z zalozenia mozemy wrzucic $c$ na druga strone:
$$c=\alpha_c^{-1}(a-\sum\limits_{c\in B\setminus\{c\}}\alpha_bb) \implies c\in \Lin{(B\setminus\{c\})\cup\{a\}}\implies B\subseteq\Lin{(B\setminus\{c\})\cup\{a\}}=V$$
Teraz pokazujemy lnz:
$$\beta_a\cdot a+\sum\limits_{b\in B\setminus\{c\}}
\beta_bb=0$$
Za $a$ popdstawiamy sume
$$\beta_a\cdot \sum\limits_{b\in B}\alpha_b b+\sum\limits_{b\in B\setminus\{c\}}\beta_bb=\beta_a\alpha_cc+\sum\limits_{b\in B\setminus\{c\}}(\beta_b+\beta_a\alpha_b)b=0$$
Jest to kombinacja liniowa elementow $B$. Wszystkie te wspolczynniki sa rowne 0, wiec $\beta_a\alpha_c=0$, wiec $\beta_a=0\;\lor\;\alpha_c=0$, ale w zalozeniu mielismy, ze $\alpha_c\neq 0$, skad mamy, ze $\beta_a=0$, ale Wowczas
$$0=0c+\sum\limits_{b\in B\setminus\{c\}}(0\alpha_b+\beta_b)b$$
$$0=\sum\limits_{b\in B\setminus\{c\}}\beta_b$$
wiec wszystkie $\beta_a=0$.\\
DOWOD LEMATU STEINITZA\\
$B$ - baza, $a_1, ..., a_n$ jest lnz. Szukamy $c_1, ..., c_n$ tak ze $(B\setminus\{c_1, ..., c_n\})\cup\{a_1, ..., a_n\}$ jest baza.\\
Dowood indukcyjnie
$B$ jest baza i $a_1\in V$, czyli $0\neq a_1=\sum\limits_{b\in B}\alpha_b\cdot b\implies \exists\;c_1\in B$ takie, ze $\alpha_{c_1}\neq0$. Co sugeruje, ze istnieje $B_1 = (B\setminus\{c_1\})\cup\{a_1\}$.\\
Wydaje sie, ze mozemy teraz powtorzyc ten argument, ale to mogloby sie nie sprawdzic, bo moze wybralisy ten sam wektor co w pierwszym kroki.\\
Wezmy $a_2=\sum\limits_{b\in B_1}\alpha_bb=\alpha_{a_1}a_1+\sum\limits_{b\in B_1\setminus\{a_1\}}\alpha_bb$ i wtedy ktorys ze wspolczynnikow jest niezerowy, wiec mozemy wziac jakis element $c_2\in B_1\setminus\{a_1\}=B\setminus\{c_1\}$. W szczegolnosci $c_1\neq c_1$.\\
Zalozmy, ze mamy $c_1, ..., c_k\subseteq B$ parami rozne, takie, ze $B_k>))2\setminus\{c_1, ..., c_k\}\cup\{a_1, ..., a_k\}$l ktora jest baza.\\
Teraz zauwazamy, ze $a_{k+1}\in\Lin{B_k}=\sum\limits_\{b\in B_k\}\alpha_bb=\alpha_{a_1}a_1+...+\alpha_{a_k}a_k+\sum\limits_{b\in B_k}\alpha_bb$, czyli jais element tej sumy jest niezerowy.\\
Wezmy $c_{k+1}\in B_k'$ taki, ze $\alpha_{k+1}\neq0$ i z twierdzenia
$$B_{k+1}=(B_k'\setminus \{c_{k+1}\})\cup\{a_{k+1}\}=B\setminus\{c_1, c_2, ..., c_{n+1}\}\cup\{a_1, ..., a_{n+1}\}$$
ten zbior jest baza.\\
$c_{k+1}\neq c_1, ..., c_k$.\\
$B_n$ dziala, czyli jest baza.
\end{document}