\documentclass{article}

\usepackage[utf8]{inputenc}
\usepackage{tikz}
\usepackage{amsmath}
\usepackage{mathtools}
\usepackage{amsfonts}
\usepackage{amssymb}
\usepackage{sectsty}
\usepackage{xcolor}
\usepackage[paperwidth=180mm, paperheight=290mm, left=10mm, top=10mm, bottom=10mm, right=10mm, margin=10mm]{geometry}
\usepackage{ragged2e}
\usepackage{listings}
\usepackage[hidelinks]{hyperref}

\definecolor{def}{RGB}{255, 150, 89}
\definecolor{tit}{RGB}{217, 84, 80}
\definecolor{emp}{RGB}{150, 206, 180}
\definecolor{acc}{RGB}{255, 234, 150}
\definecolor{txt}{RGB}{249, 232, 232}
\definecolor{back}{RGB}{22, 22, 22}

\sectionfont{\color{tit}\fontsize{20.74}{35}\ttfamily}
\subsectionfont{\color{tit}\ttfamily}

\DeclareFontFamily{\encodingdefault}{\ttdefault}{
  \hyphenchar\font=\defaulthyphenchar
  \fontdimen2\font=0.33333em
  \fontdimen3\font=0.16667em
  \fontdimen4\font=0.11111em
  \fontdimen7\font=0.11111em
}

\newcommand{\R}{\mathbb{R}}
\newcommand{\N}{\mathbb{N}}
\newcommand{\Q}{\mathbb{Q}}
\newcommand{\Z}{\mathbb{Z}}
\newcommand{\C}{\mathbb{C}}
\newcommand{\cont}{\mathfrak{c}}

\geometry{a4paper, textwidth=155mm, textheight=267mm, left=15mm, top=15mm, right=15mm, marginparwidth=0mm}
\setlength\parindent{15pt}

\pagestyle{empty}

\begin{document}\pagecolor{back}\color{txt}\ttfamily
\section*{POWTORECZKA?}
\subsection*{PODSTAWOWE POJECIA}
    \begin{center}
    \color{def}CIALO \color{txt}- zbior $+, \cdot, 0, 1\in K$, gdzie\smallskip\par
        +, $\cdot$ sa przemienne, laczne i rozdzielne\smallskip\par
        $\forall\;x\quad0+x=1\cdot x=x$\smallskip\par
        $\forall\;x\;\exists\;-x\quad x+(-x)=0$\smallskip\par
        $x\neq0$: $\exists\;x^{-1}\quad x\cdot x^{-1}=1$\smallskip\par
        $0\neq1$\medskip\\
    \color{emp}Jesli istnieja $-x$ oraz $x^{-1}$, to sa one jedyne.\color{txt}\medskip\\
    \end{center}
    $\R,\C,\Q$ sa cialami, ale $\Z$ nie jest cialem (nie ma elementu odwrotnego do 2).\smallskip\\
    Cialo jest \color{acc}zamkniete na dodawanie i mnozenie\color{txt}, czyli wynikiem tych dzialan na zbiorze $K$ jest element ze zbioru $K$.\medskip\\
    $\{0, 1, 2, 3, 4\}$ z dodawaniem i mnozeniem $\mod5$ jest cialem, czyli jest element odwrotny dla wszystkich liczb:
    $$2*3=4*4=1$$
    Mozna zdefiniowac cialo rozszerzone o pierwiastek:
    $$d\in\Q\to\Q[\sqrt{d}]\{a+b\sqrt{d}\;:\;a,b\in\Q\}$$
    Cialo $K$ rozszezmy o $x$: $K[x]$ nie jest cialem, bo $x^{-1}$ nie istnieje.
    \begin{center}\color{def}PRZESTRZEN LINIOWA \color{txt}nad $K$ to zbior $V$:\\
        $+:\;V\times V\to V$\\
        $\cdot:\;K\times V\to V$\\
        $0\in V$\smallskip\\
        takie, ze:\smallskip\par
        +, $\cdot$ sa laczne, przemienne i rozdzielne\smallskip\par
        0 jest tylko jedno\smallskip\par
        zachodzi lacznosc mnozenia dla $\underset{V}{\cdot}$ $\underset{K}{\cdot}$: $(\alpha\underset{K}{\cdot}\beta)\underset{V}{\cdot}v=\alpha\underset{V}{\cdot}(\beta\underset{V}{\cdot}v)$\smallskip\\
    \end{center}
    Przestrzeniami liniowymi sa m.in. $\R^2$ i $\R^3$ nad $\R$, $\C^2$ i $\C^3$ nad $\C$.\medskip\\
    Jesli $K$ jest cialem, do dla dowolnego $n\in\N$ zbior $K^n$ jest przestrzenia liniowa nad $K$\smallskip\\
    Jesli $A$ jest dowolnym zbiorem, to $K^A =\{f:A\to K\}$ tez jest przestrzenia liniowa nad K\smallskip\\
    $K_n[X]$ - zbior wielomianow o stopniu $\leq n$ i wspolczynnikach z $K$ jest przestrzenia liniowa nad $K$\smallskip\\
    $C(\R)$ - zbior wszystkich funkcji ciaglych $\R\to\R$ jest przestrzenia liniowa nad $\R$.
    
\end{document}