\documentclass{article}

\usepackage[utf8]{inputenc}
\usepackage{tikz}
\usepackage{amsmath}
\usepackage{mathtools}
\usepackage{amsfonts}
\usepackage{amssymb}
\usepackage{sectsty}
\usepackage{xcolor}
\usepackage[paperwidth=180mm, paperheight=290mm, left=10mm, top=10mm, bottom=10mm, right=10mm, margin=10mm]{geometry}
\usepackage{ragged2e}
\usepackage{listings}

\definecolor{def}{RGB}{255, 173, 97}
\definecolor{tit}{RGB}{217, 84, 80}
\definecolor{emp}{RGB}{92, 204, 152}
\definecolor{acc}{RGB}{255, 238, 173}
\definecolor{txt}{RGB}{249, 232, 232}
\definecolor{back}{RGB}{22, 22, 22}

\sectionfont{\color{tit}\fontsize{20.74}{35}\ttfamily}
\subsectionfont{\color{tit}\fontsize{17.28}{17.28}\ttfamily}

\DeclareFontFamily{\encodingdefault}{\ttdefault}{%
  \hyphenchar\font=\defaulthyphenchar
  \fontdimen2\font=0.33333em
  \fontdimen3\font=0.16667em
  \fontdimen4\font=0.11111em
  \fontdimen7\font=0.11111em
}

\newcommand{\R}{\mathbb{R}}
\newcommand{\N}{\mathbb{N}}
\newcommand{\Q}{\mathbb{Q}}
\newcommand{\Z}{\mathbb{Z}}
\newcommand{\C}{\mathbb{C}}
\newcommand{\cont}{\mathfrak{c}}

\geometry{a4paper, textwidth=155mm, textheight=267mm, left=15mm, top=15mm, right=15mm, marginparwidth=0mm}
\setlength\parindent{15pt}

\pagestyle{empty}

\begin{document}\pagecolor{back}\color{txt}\ttfamily
\section*{PODSTAWOWE POJECIA ALGEBRY LINIOWEJ}
\subsection*{CIALO}
  \begin{center}
    \color{def}CIALO \color{txt}to zbior K z dwoma dzialniami, \\
    dodawaniem i mnozeniem, i ich elementami neutralnymi ($0,1\in K$)\smallskip\\
    dodawanie i mnozenie to funkcje $+:K\times K\to K$
  \end{center}
  \color{def}WLASNOSCI CIAL:\color{txt}\smallskip\par
    1. dodawanie i mnozenie sa laczne, przemienne i rozdzielne\smallskip\par
    2. istnieja elementy neutralne: $0+x=1\cdot x=x$\smallskip\par
    3. dla kazdego elementu ciala istnieje element przeciwny: $\forall\;x\;\exists\;(-x)\quad x+(-x)=0$\smallskip\par
    4. dla kazdego $x\neq0$ istnieje element odwrotny: $\forall\;x\neq0\;\exists\;x^{-1}\quad x\cdot x^{-1}=1$\smallskip\par
    5. $0\neq1$ - wyklucza zbior jednoelementowys\medskip\\
  Jesli istnieja odpowiednie $-x, \;x^{-1}$, to sa one jedyne - \color{tit}dowod na cwiczeniach\color{txt}\bigskip\\
  \color{emp}PRZYKLADY:\color{txt}\smallskip\\
  $\color{acc}\R, \;\C,\;\Q$ sa cialami, natomiast $\Z$ nie jest cialem (nie ma elementu odwrotnego do 2, \color{tit}pierscienie\color{txt})\medskip\\
  Kazdy podzbior \color{acc}$K\subseteq\C$, ktory jest zamkniety na \color{txt}dodawanie, mnozenie oraz dla kazdego elementu $K$ mozna znalezc w $K$ element do niego przeciwny i odwrotny, tez jest cialem.\medskip\\
  $\{0, 1, 2, 3, 4\}$ z dodawaniem i mnozeniem modulo 5 jest cialem: jest element neutralny: $2\cdot3=4\cdot4=1$\smallskip\\
  $\{0, 1,..., p-1\}$, gdzie $p$ jest licza pierwsza jest cialem (\color{tit}dowod z algorytmu euklidesa\color{txt})\medskip\\
  Dla kazdej liczby naturalnej $n$ i dla kazdej liczby pierwszej $p$ jest cialo, ktore ma dokladnie $p^n$ elementow i sa to wszystkie ciala skonczone.\medskip\\
  Dla dowolnego $d\in K$ mozemy zdefiniowac $\color{acc}\Q[d]=\{a+b\cdot d\;:\;a,b\in\Q\}$\medskip\\
  Jesli $K$ jest cialem, to mozemy rozpatrzec zbior wszystkich wielomianow o wspolczynnikach w $K$: $K[X]$ i nie jest cialem (nie istnieje $X^{-1}$).\smallskip\\
  Mozemy rozpatrzyc tez zbior wiekszy, \color{def}\emph{cialo funkcji wymiernych} $K(X)$\color{txt}, czyli formalne ilorazy wspolczynnikow z$K$, tyle ze w mianowniku nie moze pojawic sie 0:
  $$K(X)=\{\frac{p}{q}\;:\;p,q\in K[X], \Q\in 0\}$$
  Jak dowodzic twierdzenia:
  $$\forall x\in K\quad 0\cdot a= 0$$
  \begin{align*}
    0\cdot a &= (0+0)\cdot a = 0\cdot a + 0\cdot a | + (- 0\cdot a)\\
    0\cdot a + (-0\cdot a) &= 0\cdot a + 0\cdot a+(-0\cdot a)\\
    0 = 0\cdot a + 0 &= 0\cdot a
  \end{align*}
\subsection*{PRZESTRZEN LINIOWA}
  \begin{center}
    \color{def}PRZESTRZEN LINIOWA nad K \color{txt}to zbior $V$ z dzialaniem dodawaniem i mnozeniem:\smallskip\\
    $+:V\times V\to V$\smallskip\\
    $\cdot:K\times V\to V$\smallskip\\
    $0\in V$
  \end{center}
  \color{def}WLASNOSCI:\color{txt}\smallskip\par
    + i $\cdot$ spelniaja oczywiste wlasnosci\smallskip\par
    Lacznosc mieszana dla mnozenia:
    $$(\alpha \underset{K}{\cdot}\beta)\underset{V}{\cdot}\gamma = \alpha\underset{V}{\cdot}(\beta\underset{V}{\cdot}\gamma)$$\par
    Rozdzielnosc mnozenia wzgledem dodawania:
    $$\alpha\underset{V}{\cdot}(u\underset{K}{+}w) = \alpha\underset{V}{\cdot}u\underset{V}{+}\alpha\underset{V}{\cdot}w$$
    $$(\alpha\underset{V}{+}\beta)\underset{V}{\cdot}u = \alpha\underset{V}{\cdot}u\underset{V}{+}\beta\underset{V}{\cdot}v$$ \\ \\
  \color{emp}PRZYKLADY:\color{txt}\smallskip\\
    $\R^2$, $\R^3$ to przestrzenie liniowe nad $\R$\medskip\\
    Dla kazdego iloczynu kartezjanskiego ciala, iloczyn ten jest cialem. Bardziej ogolnie mozna to ujac, ze jesli $A$ jest dowolnym zbiorem, a $K^A$ jest zbiorem wszystkich funkcji z $A$ w $K$, to $K^A$ jest przestrzenia liniowa nad $K$\medskip\\
    $K[X]$ to zbior wielomianow o wspolczynnikach z $K$, to jest on przestrzenia liniowa nad $K$. Tak samo $K_n[X]$ (wielomiany co najwyzej stopnia $n$) rowniez sa przestrzenia liniowa.\medskip\\
    $C(\R)$ to zbior wszystkich funkcji ciaglych $f:\R\to\R$ i jest on przestrzenia liniowa nad $\R$\bigskip\\
  Jesli przemnozymy dowolny wetor przez 0, to dostaniemy \color{acc}wektor zerowy\color{txt}:
  $$0\cdot v=\overset{\to}{0}$$
  Dla kazdego wektora z $V$ i kazdego skalara z $K$ istnieje dokladnie jeden wektor $w$ taki, ze:
  $$\forall\;v\in V\;\forall\;a\in K\;\exists!w\in V\quad a\cdot v+w=0$$
  Wezmy $v=-a^{-1}\cdot w$. Chcemy udowodnic rownanie
  $$a\cdot v+w=0$$
  $$a\cdot (-a^{-1}\cdot w)+w=0$$
  $$(-1\cdot 1)\cdot w+w=0$$
  $$(-1+1)\cdot w=0$$
  $$0\cdot w=0$$
  Z tego wynika, ze $(-1)\cdot w=-w$ oraz $-(v+w)=(-v)+(-w)$.\bigskip\\
  \color{def}LEMAT \color{txt}jesli $V$ jest przestrzenia liniowa, a $W\subseteq V$, takim, ze $W\neq\emptyset$ oraz
  $$\forall\;a\in K\;\forall\;w\in W\quad a\cdot w\in W$$
  $$\forall\;w_1,w_2\in W\quad w_1+w_2\in W,$$
  to \color{acc}$W$ jest przestrzenia liniowa\color{txt}. Jest to odpowiednik twierdzenia dla cial.\medskip\\
  \color{emp}DOWOD:\color{txt}\smallskip\\
  Wlasciwosci dodawania i odejmowania przenosza sie automatycznie. Zostaje sprawdzic, ze
  $$0\in W$$
  $$\forall\;w\in W\;\exists\;-w\in W$$
  Poniewaz $W\neq\emptyset$, stad istnieje jakies $w\in W$. Wowczas, 
  $$0\cdot w=\overset{\to}{0}$$
  z tego, ze $W$ jest zamkniete na mnozenie przez skalary. Wiec pokazalismy, ze $0\in W$. Tak samo, skoro mozemy przemnozyc $w\in W$ przez kazdy skalar i otrzymac element $W$, Wowczas
  $$(-1)\cdot w =-w\in W$$
  Podzbior, ktorego istnienie udowodnilismy wyzej, nazywamy \color{def}PODPRZESTRZENIA V \color{txt} i oznaczamy
  $$\color{acc}W\leq V$$
  \color{emp}PRZYKLADY:\color{txt}\medskip\\
  Proste przechodzace przez 0 w $K^2$ sa podprzestrzenia. \color{tit}MOGE OBRAZECZKI PROSTYCH ZROBIC\color{txt}\smallskip\\
  \color{def}PROSTA \color{txt}- podprzestrzen rozpieta przez jeden wektor, czyli bierzemy jeden wektor i patrzymy na wszystkie jego skalarne nierownosci.\medskip\\
  W ogolnosci, $n>m\implies K^n\geq K^m$.\medskip\\
  $C^1(\R)\leq C(\R)$ zbior funkcji rozniczkowalnych jest podprzestrzenia zbioru funkcji ciaglych. Ten z kolei jest podprzestrzenia zbioru wszystkich funkcji z $\R$ w $\R$ ($C(\R)\leq\R^\R$)\medskip\\
  Zbioro funkcji z $\R$ w $\R$ zbiegajacych do dowolnego $x_0$ to tez jest podprzestrzenia:
  $$\{f:\R\to\R\;:\;\lim\limits_{x\to x_0}f(x)=0\}\leq \R^\R$$
  Mozemy tez przekroic dwie podprzestrzenie.\medskip\\
  Zbior ciagow spelniajacych rekurencje:
  $$\{(a_n)_{n\in\N}\;:\;\forall\;n\quad a_{n+2}=a_n+a_{n+1}\}\leq \R^\N$$
  jest podprzestrzenia zbioru wszystkich ciagowo indeksach w $\N$ i wyrazach $\R$
  \emph{na dzisiaj chyba koniec, jutro rano kolejne 4h wykladow!}
\end{document}