\documentclass{article}

\usepackage{../../notatka}

\begin{document}\pagecolor{back}\color{txt}\ttfamily
\section*{PODSTAWOWE POJECIA ALGEBRY LINIOWEJ}
\subsection*{CIALO}
  \begin{center}
    \color{def}CIALO \color{txt}to zbior K z dwoma dzialniami, \\
    dodawaniem i mnozeniem, i ich elementami neutralnymi ($0,1\in K$)\smallskip\\
    dodawanie i mnozenie to funkcje $+:K\times K\to K$
  \end{center}
  \color{def}WLASNOSCI CIAL:\color{txt}\smallskip\par
    1. dodawanie i mnozenie sa laczne, przemienne i rozdzielne\smallskip\par
    2. istnieja elementy neutralne: $0+x=1\cdot x=x$\smallskip\par
    3. dla kazdego elementu ciala istnieje element przeciwny: $\forall\;x\;\exists\;(-x)\quad x+(-x)=0$\smallskip\par
    4. dla kazdego $x\neq0$ istnieje element odwrotny: $\forall\;x\neq0\;\exists\;x^{-1}\quad x\cdot x^{-1}=1$\smallskip\par
    5. $0\neq1$ - wyklucza zbior jednoelementowys\medskip\\
  Jesli istnieja odpowiednie $-x, \;x^{-1}$, to sa one jedyne - \color{tit}dowod na cwiczeniach\color{txt}\bigskip\\
  \color{emp}PRZYKLADY:\color{txt}\smallskip\\
  $\color{acc}\R, \;\C,\;\Q$ sa cialami, natomiast $\Z$ nie jest cialem (nie ma elementu odwrotnego do 2, \color{tit}pierscienie\color{txt})\medskip\\
  Kazdy podzbior \color{acc}$K\subseteq\C$, ktory jest zamkniety na \color{txt}dodawanie, mnozenie oraz dla kazdego elementu $K$ mozna znalezc w $K$ element do niego przeciwny i odwrotny, tez jest cialem.\medskip\\
  $\{0, 1, 2, 3, 4\}$ z dodawaniem i mnozeniem modulo 5 jest cialem: jest element neutralny: $2\cdot3=4\cdot4=1$\smallskip\\
  $\{0, 1,..., p-1\}$, gdzie $p$ jest licza pierwsza jest cialem (\color{tit}dowod z algorytmu euklidesa\color{txt})\medskip\\
  Dla kazdej liczby naturalnej $n$ i dla kazdej liczby pierwszej $p$ jest cialo, ktore ma dokladnie $p^n$ elementow i sa to wszystkie ciala skonczone.\medskip\\
  Dla dowolnego $d\in K$ mozemy zdefiniowac $\color{acc}\Q[d]=\{a+b\cdot d\;:\;a,b\in\Q\}$\medskip\\
  Jesli $K$ jest cialem, to mozemy rozpatrzec zbior wszystkich wielomianow o wspolczynnikach w $K$: $K[X]$ i nie jest cialem (nie istnieje $X^{-1}$).\smallskip\\
  Mozemy rozpatrzyc tez zbior wiekszy, \color{def}\emph{cialo funkcji wymiernych} $K(X)$\color{txt}, czyli formalne ilorazy wspolczynnikow z$K$, tyle ze w mianowniku nie moze pojawic sie 0:
  $$K(X)=\{\frac{p}{q}\;:\;p,q\in K[X], \Q\in 0\}$$
  Jak dowodzic twierdzenia:
  $$\forall x\in K\quad 0\cdot a= 0$$
  \begin{align*}
    0\cdot a &= (0+0)\cdot a = 0\cdot a + 0\cdot a | + (- 0\cdot a)\\
    0\cdot a + (-0\cdot a) &= 0\cdot a + 0\cdot a+(-0\cdot a)\\
    0 = 0\cdot a + 0 &= 0\cdot a
  \end{align*}
\subsection*{PRZESTRZEN LINIOWA}
  \begin{center}
    \color{def}PRZESTRZEN LINIOWA nad K \color{txt}to zbior $V$ z dzialaniem dodawaniem i mnozeniem:\smallskip\\
    $+:V\times V\to V$\smallskip\\
    $\cdot:K\times V\to V$\smallskip\\
    $0\in V$
  \end{center}
  \color{def}WLASNOSCI:\color{txt}\smallskip\par
    + i $\cdot$ spelniaja oczywiste wlasnosci\smallskip\par
    Lacznosc mieszana dla mnozenia:
    $$(\alpha \underset{K}{\cdot}\beta)\underset{V}{\cdot}\gamma = \alpha\underset{V}{\cdot}(\beta\underset{V}{\cdot}\gamma)$$\par
    Rozdzielnosc mnozenia wzgledem dodawania:
    $$\alpha\underset{V}{\cdot}(u\underset{K}{+}w) = \alpha\underset{V}{\cdot}u\underset{V}{+}\alpha\underset{V}{\cdot}w$$
    $$(\alpha\underset{V}{+}\beta)\underset{V}{\cdot}u = \alpha\underset{V}{\cdot}u\underset{V}{+}\beta\underset{V}{\cdot}v$$ \\ \\
  \color{emp}PRZYKLADY:\color{txt}\smallskip\\
    $\R^2$, $\R^3$ to przestrzenie liniowe nad $\R$\medskip\\
    Dla kazdego iloczynu kartezjanskiego ciala, iloczyn ten jest cialem. Bardziej ogolnie mozna to ujac, ze jesli $A$ jest dowolnym zbiorem, a $K^A$ jest zbiorem wszystkich funkcji z $A$ w $K$, to $K^A$ jest przestrzenia liniowa nad $K$\medskip\\
    $K[X]$ to zbior wielomianow o wspolczynnikach z $K$, to jest on przestrzenia liniowa nad $K$. Tak samo $K_n[X]$ (wielomiany co najwyzej stopnia $n$) rowniez sa przestrzenia liniowa.\medskip\\
    $C(\R)$ to zbior wszystkich funkcji ciaglych $f:\R\to\R$ i jest on przestrzenia liniowa nad $\R$\bigskip\\
  Jesli przemnozymy dowolny wetor przez 0, to dostaniemy \color{acc}wektor zerowy\color{txt}:
  $$0\cdot v=\overset{\to}{0}$$
  Dla kazdego wektora z $V$ i kazdego skalara z $K$ istnieje dokladnie jeden wektor $w$ taki, ze:
  $$\forall\;v\in V\;\forall\;a\in K\;\exists!w\in V\quad a\cdot v+w=0$$
  Wezmy $v=-a^{-1}\cdot w$. Chcemy udowodnic rownanie
  $$a\cdot v+w=0$$
  $$a\cdot (-a^{-1}\cdot w)+w=0$$
  $$(-1\cdot 1)\cdot w+w=0$$
  $$(-1+1)\cdot w=0$$
  $$0\cdot w=0$$
  Z tego wynika, ze $(-1)\cdot w=-w$ oraz $-(v+w)=(-v)+(-w)$.\bigskip\\\color{txt}\dotfill\bigskip\\
  \color{def}LEMAT \color{txt}jesli $V$ jest przestrzenia liniowa, a $W\subseteq V$, takim, ze $W\neq\emptyset$ oraz
  $$\forall\;a\in K\;\forall\;w\in W\quad a\cdot w\in W$$
  $$\forall\;w_1,w_2\in W\quad w_1+w_2\in W,$$
  to \color{emp}$W$ jest przestrzenia liniowa\color{txt}. Jest to \color{acc}odpowiednik twierdzenia dla cial\color{txt}.\medskip\\
  \color{emp}DOWOD:\color{txt}\smallskip\\
  Wlasciwosci dodawania i odejmowania przenosza sie automatycznie. Zostaje sprawdzic, ze
  $$\texttt{1. }0\in W$$
  $$\texttt{2. }\forall\;w\in W\;\exists\;-w\in W$$
  \color{tit}1. \color{txt}Poniewaz $W\neq\emptyset$, stad istnieje jakies $w\in W$. Wowczas, 
  $$0\cdot w=\overset{\to}{0}$$
  z tego, ze $W$ jest zamkniete na mnozenie przez skalary. Wiec pokazalismy, ze $0\in W$. \smallskip\\
  \color{tit}2. \color{txt}Tak samo, skoro mozemy przemnozyc $w\in W$ przez kazdy skalar i otrzymac element $W$, Wowczas
  $$(-1)\cdot w =-w\in W$$
  Podzbior $W\subseteq V$, ktorego istnienie udowodnilismy wyzej, nazywamy \color{def}PODPRZESTRZENIA V \color{txt} i oznaczamy
  $$\color{acc}W\leq V$$

  \podz{gr}\medskip\\

  \color{emp}PRZYKLADY:\color{txt}\medskip\\
  Proste przechodzace przez 0 w $K^2$ sa podprzestrzenia. Niech $K=F_2=\{0,1\}$ ($K$ to cialo dwuelementowe)
  \begin{center}\begin{tikzpicture}
    \filldraw[color=white, fill=white, thick] (0, 0) circle (0.1);
    \filldraw[color=white, fill=white, thick] (0, 1.5) circle (0.1);
    \filldraw[color=white, fill=white, thick] (1.5, 0) circle (0.1);
    \filldraw[color=white, fill=white, thick] (1.5, 1.5) circle (0.1);
    \draw[emp, ultra thick] (0, 0.75) ellipse (0.4 and 1.2);
    \draw[def, ultra thick] (0.75, 0) ellipse (1.2 and 0.4);
  \end{tikzpicture}\end{center}
  Tak samo proste przechodzace przez 0 w $K^3$ sa przestrzeniami. Na przyklad dla $K=F_3=\{0,1,2\}$
  \begin{center}\begin{tikzpicture}
    \filldraw[color=white, fill=white, thick] (0, 0) circle (0.1);
    \filldraw[color=white, fill=white, thick] (0, 1.5) circle (0.1);
    \filldraw[color=white, fill=white, thick] (0, 3)  circle (0.1);
    \filldraw[color=white, fill=white, thick] (1.5, 0) circle (0.1);
    \filldraw[color=white, fill=white, thick] (1.5, 1.5) circle (0.1);
    \filldraw[color=white, fill=white, thick] (1.5, 3) circle (0.1);
    \filldraw[color=white, fill=white, thick] (3, 1.5) circle (0.1);
    \filldraw[color=white, fill=white, thick] (3, 3)  circle (0.1);
    \filldraw[color=white, fill=white, thick] (3, 0)  circle (0.1);
    \draw[color=emp, ultra thick] (0, 1.5) ellipse (0.5 and 2);
    \draw[color=def, ultra thick] (1.5, 0) ellipse (2 and 0.5);
    \draw[color=acc, ultra thick, rotate around={135:(1.5, 1.5)}] (1.5, 1.5) ellipse (0.5 and 2.5);
    \draw[color=tit, ultra thick] (0, 0) circle (0.6);
    \draw[color=tit, ultra thick] (1.5, 3) circle (0.6);
    \draw[color=tit, ultra thick] (3, 1.5) circle (0.6);
  \end{tikzpicture}\end{center}\bigskip
  \emph{\color{def}PROSTA \color{txt}- podprzestrzen rozpieta przez jeden wektor, czyli bierzemy jeden wektor i patrzymy na wszystkie jego skalarne nierownosci.}\medskip\\
  \color{emp}W ogolnosci, $n>m\implies K^n\geq K^m$.\color{txt}\medskip\\
  $C^1(\R)\leq C(\R)$ zbior funkcji rozniczkowalnych jest podprzestrzenia zbioru funkcji ciaglych. Ten z kolei jest podprzestrzenia zbioru wszystkich funkcji z $\R$ w $\R$ ($C(\R)\leq\R^\R$):
  $$\color{acc}C^1(\R)\leq C(\R)\leq \R^\R.$$
  Zbior funkcji z $\R$ w $\R$ zbiegajacych do dowolnego $x_0$ to tez jest podprzestrzenia:
  $$\{f:\R\to\R\;:\;\lim\limits_{x\to x_0}f(x)=0\}\leq \R^\R$$
  Mozemy tez przekroic dwie podprzestrzenie. Na przyklad wszystkie funkcje rozniczkowane, ktore daza do 0.\medskip\\
  Zbior ciagow spelniajacych rekurencje:
  $$\{(a_n)_{n\in\N}\;:\;\forall\;n\quad a_{n+2}=a_n+a_{n+1}\}\leq \R^\N$$
  jest podprzestrzenia zbioru wszystkich ciagow o indeksach w $\N$ i wyrazach $\R$\bigskip\\
  \color{txt}\dotfill\bigskip
  \begin{center}
    \color{def}LEMAT: \color{txt}dla dwoch podprzestrzeni $W_1,W_2\leq V$ zachodzi:\smallskip\\
    1. $W_1\cap W_2\leq V$\smallskip\\
    2. $W=W_1+W_2=\{w_1+w_2\;:\; w_1\in W_1, w_2\in W_2\}$ \\
    \emph{czyli \color{acc}suma kompleksowa \color{txt}podprzestrzeni jest podprzestrzenia}
  \end{center}
  \color{tit}1. \color{txt}lematu zostanie udowodniona \color{tit}NA CWICZENIACH\color{txt}.\medskip\\
  \color{tit}2. \color{txt}Niepustosc jest oczywista. Chcemy sprawdzic, czy ten zbior jest zmakniety na dzialania.\smallskip\\
  Zmakniecie na mnozenie przez skalary:
  $$a\in K,\quad w_1+w_2\in W$$
  $$a\cdot(w_1+w_2)=\underbrace{a\cdot w_1}_{\in W_1}+\underbrace{a\cdot w_2}_{\in W_2}\in W_1+W_2=W.$$
  Zamkniecie na dodawanie:
  $$(w_1+w_2), (w_1', w_2')\in W, \quad w_1, w_1'\in W_1,\; w_2, w_2'\in W_2$$
  $$(w_1+w_2)+(w_1'+w_2')=w_1+w_2+w_1'+w_2'=\underbrace{(w_1+w_1')}_{\in W_1}+\underbrace{(w_2+w_2')}_{\in W_2}\in W\bigskip$$
  1. $W_1\leq V\;\land\; W_1\leq W_2\implies W_2\leq V $\smallskip\\
  2. $W_1, W_2\leq V\;\land \; W_1\subseteq W_2\implies W_1\leq W_2$\medskip\\
  \color{tit}CWICZENIA\color{txt}\bigskip\\\podz{acc}\bigskip
\subsection*{KOMBINACJA LINIOWA}
  \begin{center}
    Dla pewnej przestrzeni liniowej $V$ i zbioru $A\subseteq V$ \color{def}OTOCZKA LINIOWA \color{txt}$A$ \\to najmniejsza podprzestrzen $V$, ktora zaiwera $A$\smallskip\\
    $\texttt{Lin}(A)=\{v=\sum\limits_{k=1}^n\alpha_kv_k\;:\;\alpha_k\in K\;\land \; v_k\in A\}$
  \end{center}
  \color{emp}DOWOD\color{txt}: \medskip\\
  $\sum\limits_{k=1}^n\alpha_kv_k\in\texttt{Lin}(A)$ mozna pokazac korzystajac z prostej \color{tit}indukcji\color{txt}. \smallskip\\
  Wystarczy pokazac, ze zbior takich wektorow jest podprzestrzenia.
  $$\texttt{Lin}(A)\neq\emptyset$$
  Bo pusta suma jest rowna zero (czyli wektor zerowy)
  $$\texttt{Lin}(\emptyset)=\sum\limits_{k=1}^0\alpha_kv_k=0.$$
  Wezmiemy dwa wektory bedace sumami wektorow w $A$:
  $$\sum\limits_{k=1}^n\alpha_kv_k+\sum\limits_{l=1}^m\beta_lw_l.$$
  Rozpiszmy to:
  \begin{align*}
    \color{acc}\gamma_1, ..., \gamma_n\quad&\color{acc}\gamma_{n+1}, ..., \gamma_{n+m}\\
    \alpha_1, ..., \alpha_n\quad &\beta_1,\;\; ..., \;\;\beta_m\\
    v_1, ..., v_n\quad &w_1,\;\; ..., \;\;w_m\\
    \color{acc}u_1, ..., u_n\quad &\color{acc}u_{n+1}, ..., u_{n+m}
  \end{align*}
  Z tego widac, ze
  $$\sum\limits_{k=1}^n\alpha_kv_k+\sum\limits_{l=1}^m\beta_lw_l=\sum\limits_{j=1}^{n+m}\gamma_ju_j.$$
  Czyli $\texttt{Lin}(A)$ jest zamkniety na dodawanie.\smallskip\\
  Zamkniecie na mnozenie, przy pomocy sumy kompleksowej:
  $$\alpha\sum\limits_{k=1}^n\alpha_kv_k=\sum\limits_{k=1}^n(\alpha\cdot\alpha_k)v_k$$
  Z powyzszego rozumowania wynika, ze
  $$\color{def}\texttt{Lin}(A)=\bigcap\;\{W\leq V\;:\;A\subseteq W\}$$
  \color{emp}DOWOD\color{txt}:\medskip\\
  Z definicji $\Lin{A}\subseteq W$, czyli
  $$\Lin{A}\subseteq\bigcap\{W\leq V\;:\;A\subseteq W\},$$
  a otoczka liniowa nalezy do tej rodziny podprzestrzeni:
  $$\Lin{A}\in \{W\leq V\;:\;A\subseteq W\}$$
  wiec zawiera jego przekroj
  $$\texttt{Lin}(A)=\bigcap\;\{W\leq V\;:\;A\subseteq W\}$$
  \begin{center}
    \color{def}KOMBINACJA LINIOWA \color{txt}wektorow $v_1, ..., v_n$ to element $\Lin{v_1, ..., v_n}$, czyli wektor postaci\smallskip\\
    $\sum\limits_{k=1}^n\alpha_kv_k$
  \end{center}\bigskip
  \color{emp}PRZYKLADY\color{txt}:\medskip\\
  prosta rozpieta przez niezerowy wektor $v$:
  $$\Lin{v}=\{\alpha\cdot v\;:\;\alpha\in K\}$$
  kombinacja punktow nalezacych do hiperboli na $\R^2$:
  $$\Lin{\{{x\choose0}\in\R^2\;:\;xy=1\}}=\R^2$$
  kombinacja liniowa wszystkich punktow na plaszczyznie:
  $$\Lin{\{\begin{pmatrix}x\\y\\z\end{pmatrix}\in\R^3\;:\;x+y+z=1\}}=\R^3$$
  \podz{gr}
    $$\color{def}A\subseteq B\implies \Lin{A}\subseteq\Lin{B}$$
    \color{emp}DOWOD\color{txt}: $A\subseteq B\subseteq\Lin{B}$ i $\Lin{B}\leq V$, wiec $\Lin{A}\leq\Lin{B}$
    $$\color{def}\Lin{\Lin{A}}=\Lin{A}$$
    \color{emp}DOWOD\color{txt}: $\Lin{A}\leq V$, wiec jest najmniejsza podprzestrzena zawierajaca $\Lin{A}$
    $$\color{def}b\in\Lin{A}\iff\Lin{A}=\Lin{A\cup\{b\}}$$
    \color{emp}DOWOD\color{txt}: $b\in\Lin{A}$, wiec $A\subseteq A\cup\{b\}\subseteq\Lin{A\cup\{b\}}$ i $A\subseteq\Lin{A}\subseteq\Lin{A\cup\{b\}}$. \smallskip\\
    Z drugiej strony, wiemy, ze $A\cup\{b\}\subseteq\Lin{A}$, czyli $\Lin{A\cup\{b\}}\subseteq\Lin{\Lin{A}}=\Lin{A}$. \smallskip\\
    Dostajemy $\Lin{A\cup\{b\}}\subseteq\Lin{A}$ oraz $\Lin{A}\subseteq\Lin{A\cup\{b\}}.$ Mamy inkluzje w obie strony, takze $\Lin{A\cup\{b\}}=\Lin{A}$.
  \subsection*{LINIOWO NIEZALEZNE}
  \begin{center}
    Mowimy, ze wektory $v_1, ..., v_n$ sa \color{def}LINIOWO NIEZALEZNE \color{txt}(lnz), gdy\smallskip\\
    $\sum\alpha_kv_k=0\implies \;\forall\;k\quad \alpha_k=0$\medskip\\
    \emph{\color{acc}Zbior $A\subseteq V$ jest lnz, gdy kazdy (skonczony) zbior roznych wektorow z $A$ jest lnz.}
  \end{center}
\end{document}